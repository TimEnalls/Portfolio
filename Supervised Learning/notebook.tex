
% Default to the notebook output style

    


% Inherit from the specified cell style.




    
\documentclass[11pt]{article}

    
    
    \usepackage[T1]{fontenc}
    % Nicer default font (+ math font) than Computer Modern for most use cases
    \usepackage{mathpazo}

    % Basic figure setup, for now with no caption control since it's done
    % automatically by Pandoc (which extracts ![](path) syntax from Markdown).
    \usepackage{graphicx}
    % We will generate all images so they have a width \maxwidth. This means
    % that they will get their normal width if they fit onto the page, but
    % are scaled down if they would overflow the margins.
    \makeatletter
    \def\maxwidth{\ifdim\Gin@nat@width>\linewidth\linewidth
    \else\Gin@nat@width\fi}
    \makeatother
    \let\Oldincludegraphics\includegraphics
    % Set max figure width to be 80% of text width, for now hardcoded.
    \renewcommand{\includegraphics}[1]{\Oldincludegraphics[width=.8\maxwidth]{#1}}
    % Ensure that by default, figures have no caption (until we provide a
    % proper Figure object with a Caption API and a way to capture that
    % in the conversion process - todo).
    \usepackage{caption}
    \DeclareCaptionLabelFormat{nolabel}{}
    \captionsetup{labelformat=nolabel}

    \usepackage{adjustbox} % Used to constrain images to a maximum size 
    \usepackage{xcolor} % Allow colors to be defined
    \usepackage{enumerate} % Needed for markdown enumerations to work
    \usepackage{geometry} % Used to adjust the document margins
    \usepackage{amsmath} % Equations
    \usepackage{amssymb} % Equations
    \usepackage{textcomp} % defines textquotesingle
    % Hack from http://tex.stackexchange.com/a/47451/13684:
    \AtBeginDocument{%
        \def\PYZsq{\textquotesingle}% Upright quotes in Pygmentized code
    }
    \usepackage{upquote} % Upright quotes for verbatim code
    \usepackage{eurosym} % defines \euro
    \usepackage[mathletters]{ucs} % Extended unicode (utf-8) support
    \usepackage[utf8x]{inputenc} % Allow utf-8 characters in the tex document
    \usepackage{fancyvrb} % verbatim replacement that allows latex
    \usepackage{grffile} % extends the file name processing of package graphics 
                         % to support a larger range 
    % The hyperref package gives us a pdf with properly built
    % internal navigation ('pdf bookmarks' for the table of contents,
    % internal cross-reference links, web links for URLs, etc.)
    \usepackage{hyperref}
    \usepackage{longtable} % longtable support required by pandoc >1.10
    \usepackage{booktabs}  % table support for pandoc > 1.12.2
    \usepackage[inline]{enumitem} % IRkernel/repr support (it uses the enumerate* environment)
    \usepackage[normalem]{ulem} % ulem is needed to support strikethroughs (\sout)
                                % normalem makes italics be italics, not underlines
    

    
    
    % Colors for the hyperref package
    \definecolor{urlcolor}{rgb}{0,.145,.698}
    \definecolor{linkcolor}{rgb}{.71,0.21,0.01}
    \definecolor{citecolor}{rgb}{.12,.54,.11}

    % ANSI colors
    \definecolor{ansi-black}{HTML}{3E424D}
    \definecolor{ansi-black-intense}{HTML}{282C36}
    \definecolor{ansi-red}{HTML}{E75C58}
    \definecolor{ansi-red-intense}{HTML}{B22B31}
    \definecolor{ansi-green}{HTML}{00A250}
    \definecolor{ansi-green-intense}{HTML}{007427}
    \definecolor{ansi-yellow}{HTML}{DDB62B}
    \definecolor{ansi-yellow-intense}{HTML}{B27D12}
    \definecolor{ansi-blue}{HTML}{208FFB}
    \definecolor{ansi-blue-intense}{HTML}{0065CA}
    \definecolor{ansi-magenta}{HTML}{D160C4}
    \definecolor{ansi-magenta-intense}{HTML}{A03196}
    \definecolor{ansi-cyan}{HTML}{60C6C8}
    \definecolor{ansi-cyan-intense}{HTML}{258F8F}
    \definecolor{ansi-white}{HTML}{C5C1B4}
    \definecolor{ansi-white-intense}{HTML}{A1A6B2}

    % commands and environments needed by pandoc snippets
    % extracted from the output of `pandoc -s`
    \providecommand{\tightlist}{%
      \setlength{\itemsep}{0pt}\setlength{\parskip}{0pt}}
    \DefineVerbatimEnvironment{Highlighting}{Verbatim}{commandchars=\\\{\}}
    % Add ',fontsize=\small' for more characters per line
    \newenvironment{Shaded}{}{}
    \newcommand{\KeywordTok}[1]{\textcolor[rgb]{0.00,0.44,0.13}{\textbf{{#1}}}}
    \newcommand{\DataTypeTok}[1]{\textcolor[rgb]{0.56,0.13,0.00}{{#1}}}
    \newcommand{\DecValTok}[1]{\textcolor[rgb]{0.25,0.63,0.44}{{#1}}}
    \newcommand{\BaseNTok}[1]{\textcolor[rgb]{0.25,0.63,0.44}{{#1}}}
    \newcommand{\FloatTok}[1]{\textcolor[rgb]{0.25,0.63,0.44}{{#1}}}
    \newcommand{\CharTok}[1]{\textcolor[rgb]{0.25,0.44,0.63}{{#1}}}
    \newcommand{\StringTok}[1]{\textcolor[rgb]{0.25,0.44,0.63}{{#1}}}
    \newcommand{\CommentTok}[1]{\textcolor[rgb]{0.38,0.63,0.69}{\textit{{#1}}}}
    \newcommand{\OtherTok}[1]{\textcolor[rgb]{0.00,0.44,0.13}{{#1}}}
    \newcommand{\AlertTok}[1]{\textcolor[rgb]{1.00,0.00,0.00}{\textbf{{#1}}}}
    \newcommand{\FunctionTok}[1]{\textcolor[rgb]{0.02,0.16,0.49}{{#1}}}
    \newcommand{\RegionMarkerTok}[1]{{#1}}
    \newcommand{\ErrorTok}[1]{\textcolor[rgb]{1.00,0.00,0.00}{\textbf{{#1}}}}
    \newcommand{\NormalTok}[1]{{#1}}
    
    % Additional commands for more recent versions of Pandoc
    \newcommand{\ConstantTok}[1]{\textcolor[rgb]{0.53,0.00,0.00}{{#1}}}
    \newcommand{\SpecialCharTok}[1]{\textcolor[rgb]{0.25,0.44,0.63}{{#1}}}
    \newcommand{\VerbatimStringTok}[1]{\textcolor[rgb]{0.25,0.44,0.63}{{#1}}}
    \newcommand{\SpecialStringTok}[1]{\textcolor[rgb]{0.73,0.40,0.53}{{#1}}}
    \newcommand{\ImportTok}[1]{{#1}}
    \newcommand{\DocumentationTok}[1]{\textcolor[rgb]{0.73,0.13,0.13}{\textit{{#1}}}}
    \newcommand{\AnnotationTok}[1]{\textcolor[rgb]{0.38,0.63,0.69}{\textbf{\textit{{#1}}}}}
    \newcommand{\CommentVarTok}[1]{\textcolor[rgb]{0.38,0.63,0.69}{\textbf{\textit{{#1}}}}}
    \newcommand{\VariableTok}[1]{\textcolor[rgb]{0.10,0.09,0.49}{{#1}}}
    \newcommand{\ControlFlowTok}[1]{\textcolor[rgb]{0.00,0.44,0.13}{\textbf{{#1}}}}
    \newcommand{\OperatorTok}[1]{\textcolor[rgb]{0.40,0.40,0.40}{{#1}}}
    \newcommand{\BuiltInTok}[1]{{#1}}
    \newcommand{\ExtensionTok}[1]{{#1}}
    \newcommand{\PreprocessorTok}[1]{\textcolor[rgb]{0.74,0.48,0.00}{{#1}}}
    \newcommand{\AttributeTok}[1]{\textcolor[rgb]{0.49,0.56,0.16}{{#1}}}
    \newcommand{\InformationTok}[1]{\textcolor[rgb]{0.38,0.63,0.69}{\textbf{\textit{{#1}}}}}
    \newcommand{\WarningTok}[1]{\textcolor[rgb]{0.38,0.63,0.69}{\textbf{\textit{{#1}}}}}
    
    
    % Define a nice break command that doesn't care if a line doesn't already
    % exist.
    \def\br{\hspace*{\fill} \\* }
    % Math Jax compatability definitions
    \def\gt{>}
    \def\lt{<}
    % Document parameters
    \title{finding\_donors}
    
    
    

    % Pygments definitions
    
\makeatletter
\def\PY@reset{\let\PY@it=\relax \let\PY@bf=\relax%
    \let\PY@ul=\relax \let\PY@tc=\relax%
    \let\PY@bc=\relax \let\PY@ff=\relax}
\def\PY@tok#1{\csname PY@tok@#1\endcsname}
\def\PY@toks#1+{\ifx\relax#1\empty\else%
    \PY@tok{#1}\expandafter\PY@toks\fi}
\def\PY@do#1{\PY@bc{\PY@tc{\PY@ul{%
    \PY@it{\PY@bf{\PY@ff{#1}}}}}}}
\def\PY#1#2{\PY@reset\PY@toks#1+\relax+\PY@do{#2}}

\expandafter\def\csname PY@tok@w\endcsname{\def\PY@tc##1{\textcolor[rgb]{0.73,0.73,0.73}{##1}}}
\expandafter\def\csname PY@tok@c\endcsname{\let\PY@it=\textit\def\PY@tc##1{\textcolor[rgb]{0.25,0.50,0.50}{##1}}}
\expandafter\def\csname PY@tok@cp\endcsname{\def\PY@tc##1{\textcolor[rgb]{0.74,0.48,0.00}{##1}}}
\expandafter\def\csname PY@tok@k\endcsname{\let\PY@bf=\textbf\def\PY@tc##1{\textcolor[rgb]{0.00,0.50,0.00}{##1}}}
\expandafter\def\csname PY@tok@kp\endcsname{\def\PY@tc##1{\textcolor[rgb]{0.00,0.50,0.00}{##1}}}
\expandafter\def\csname PY@tok@kt\endcsname{\def\PY@tc##1{\textcolor[rgb]{0.69,0.00,0.25}{##1}}}
\expandafter\def\csname PY@tok@o\endcsname{\def\PY@tc##1{\textcolor[rgb]{0.40,0.40,0.40}{##1}}}
\expandafter\def\csname PY@tok@ow\endcsname{\let\PY@bf=\textbf\def\PY@tc##1{\textcolor[rgb]{0.67,0.13,1.00}{##1}}}
\expandafter\def\csname PY@tok@nb\endcsname{\def\PY@tc##1{\textcolor[rgb]{0.00,0.50,0.00}{##1}}}
\expandafter\def\csname PY@tok@nf\endcsname{\def\PY@tc##1{\textcolor[rgb]{0.00,0.00,1.00}{##1}}}
\expandafter\def\csname PY@tok@nc\endcsname{\let\PY@bf=\textbf\def\PY@tc##1{\textcolor[rgb]{0.00,0.00,1.00}{##1}}}
\expandafter\def\csname PY@tok@nn\endcsname{\let\PY@bf=\textbf\def\PY@tc##1{\textcolor[rgb]{0.00,0.00,1.00}{##1}}}
\expandafter\def\csname PY@tok@ne\endcsname{\let\PY@bf=\textbf\def\PY@tc##1{\textcolor[rgb]{0.82,0.25,0.23}{##1}}}
\expandafter\def\csname PY@tok@nv\endcsname{\def\PY@tc##1{\textcolor[rgb]{0.10,0.09,0.49}{##1}}}
\expandafter\def\csname PY@tok@no\endcsname{\def\PY@tc##1{\textcolor[rgb]{0.53,0.00,0.00}{##1}}}
\expandafter\def\csname PY@tok@nl\endcsname{\def\PY@tc##1{\textcolor[rgb]{0.63,0.63,0.00}{##1}}}
\expandafter\def\csname PY@tok@ni\endcsname{\let\PY@bf=\textbf\def\PY@tc##1{\textcolor[rgb]{0.60,0.60,0.60}{##1}}}
\expandafter\def\csname PY@tok@na\endcsname{\def\PY@tc##1{\textcolor[rgb]{0.49,0.56,0.16}{##1}}}
\expandafter\def\csname PY@tok@nt\endcsname{\let\PY@bf=\textbf\def\PY@tc##1{\textcolor[rgb]{0.00,0.50,0.00}{##1}}}
\expandafter\def\csname PY@tok@nd\endcsname{\def\PY@tc##1{\textcolor[rgb]{0.67,0.13,1.00}{##1}}}
\expandafter\def\csname PY@tok@s\endcsname{\def\PY@tc##1{\textcolor[rgb]{0.73,0.13,0.13}{##1}}}
\expandafter\def\csname PY@tok@sd\endcsname{\let\PY@it=\textit\def\PY@tc##1{\textcolor[rgb]{0.73,0.13,0.13}{##1}}}
\expandafter\def\csname PY@tok@si\endcsname{\let\PY@bf=\textbf\def\PY@tc##1{\textcolor[rgb]{0.73,0.40,0.53}{##1}}}
\expandafter\def\csname PY@tok@se\endcsname{\let\PY@bf=\textbf\def\PY@tc##1{\textcolor[rgb]{0.73,0.40,0.13}{##1}}}
\expandafter\def\csname PY@tok@sr\endcsname{\def\PY@tc##1{\textcolor[rgb]{0.73,0.40,0.53}{##1}}}
\expandafter\def\csname PY@tok@ss\endcsname{\def\PY@tc##1{\textcolor[rgb]{0.10,0.09,0.49}{##1}}}
\expandafter\def\csname PY@tok@sx\endcsname{\def\PY@tc##1{\textcolor[rgb]{0.00,0.50,0.00}{##1}}}
\expandafter\def\csname PY@tok@m\endcsname{\def\PY@tc##1{\textcolor[rgb]{0.40,0.40,0.40}{##1}}}
\expandafter\def\csname PY@tok@gh\endcsname{\let\PY@bf=\textbf\def\PY@tc##1{\textcolor[rgb]{0.00,0.00,0.50}{##1}}}
\expandafter\def\csname PY@tok@gu\endcsname{\let\PY@bf=\textbf\def\PY@tc##1{\textcolor[rgb]{0.50,0.00,0.50}{##1}}}
\expandafter\def\csname PY@tok@gd\endcsname{\def\PY@tc##1{\textcolor[rgb]{0.63,0.00,0.00}{##1}}}
\expandafter\def\csname PY@tok@gi\endcsname{\def\PY@tc##1{\textcolor[rgb]{0.00,0.63,0.00}{##1}}}
\expandafter\def\csname PY@tok@gr\endcsname{\def\PY@tc##1{\textcolor[rgb]{1.00,0.00,0.00}{##1}}}
\expandafter\def\csname PY@tok@ge\endcsname{\let\PY@it=\textit}
\expandafter\def\csname PY@tok@gs\endcsname{\let\PY@bf=\textbf}
\expandafter\def\csname PY@tok@gp\endcsname{\let\PY@bf=\textbf\def\PY@tc##1{\textcolor[rgb]{0.00,0.00,0.50}{##1}}}
\expandafter\def\csname PY@tok@go\endcsname{\def\PY@tc##1{\textcolor[rgb]{0.53,0.53,0.53}{##1}}}
\expandafter\def\csname PY@tok@gt\endcsname{\def\PY@tc##1{\textcolor[rgb]{0.00,0.27,0.87}{##1}}}
\expandafter\def\csname PY@tok@err\endcsname{\def\PY@bc##1{\setlength{\fboxsep}{0pt}\fcolorbox[rgb]{1.00,0.00,0.00}{1,1,1}{\strut ##1}}}
\expandafter\def\csname PY@tok@kc\endcsname{\let\PY@bf=\textbf\def\PY@tc##1{\textcolor[rgb]{0.00,0.50,0.00}{##1}}}
\expandafter\def\csname PY@tok@kd\endcsname{\let\PY@bf=\textbf\def\PY@tc##1{\textcolor[rgb]{0.00,0.50,0.00}{##1}}}
\expandafter\def\csname PY@tok@kn\endcsname{\let\PY@bf=\textbf\def\PY@tc##1{\textcolor[rgb]{0.00,0.50,0.00}{##1}}}
\expandafter\def\csname PY@tok@kr\endcsname{\let\PY@bf=\textbf\def\PY@tc##1{\textcolor[rgb]{0.00,0.50,0.00}{##1}}}
\expandafter\def\csname PY@tok@bp\endcsname{\def\PY@tc##1{\textcolor[rgb]{0.00,0.50,0.00}{##1}}}
\expandafter\def\csname PY@tok@fm\endcsname{\def\PY@tc##1{\textcolor[rgb]{0.00,0.00,1.00}{##1}}}
\expandafter\def\csname PY@tok@vc\endcsname{\def\PY@tc##1{\textcolor[rgb]{0.10,0.09,0.49}{##1}}}
\expandafter\def\csname PY@tok@vg\endcsname{\def\PY@tc##1{\textcolor[rgb]{0.10,0.09,0.49}{##1}}}
\expandafter\def\csname PY@tok@vi\endcsname{\def\PY@tc##1{\textcolor[rgb]{0.10,0.09,0.49}{##1}}}
\expandafter\def\csname PY@tok@vm\endcsname{\def\PY@tc##1{\textcolor[rgb]{0.10,0.09,0.49}{##1}}}
\expandafter\def\csname PY@tok@sa\endcsname{\def\PY@tc##1{\textcolor[rgb]{0.73,0.13,0.13}{##1}}}
\expandafter\def\csname PY@tok@sb\endcsname{\def\PY@tc##1{\textcolor[rgb]{0.73,0.13,0.13}{##1}}}
\expandafter\def\csname PY@tok@sc\endcsname{\def\PY@tc##1{\textcolor[rgb]{0.73,0.13,0.13}{##1}}}
\expandafter\def\csname PY@tok@dl\endcsname{\def\PY@tc##1{\textcolor[rgb]{0.73,0.13,0.13}{##1}}}
\expandafter\def\csname PY@tok@s2\endcsname{\def\PY@tc##1{\textcolor[rgb]{0.73,0.13,0.13}{##1}}}
\expandafter\def\csname PY@tok@sh\endcsname{\def\PY@tc##1{\textcolor[rgb]{0.73,0.13,0.13}{##1}}}
\expandafter\def\csname PY@tok@s1\endcsname{\def\PY@tc##1{\textcolor[rgb]{0.73,0.13,0.13}{##1}}}
\expandafter\def\csname PY@tok@mb\endcsname{\def\PY@tc##1{\textcolor[rgb]{0.40,0.40,0.40}{##1}}}
\expandafter\def\csname PY@tok@mf\endcsname{\def\PY@tc##1{\textcolor[rgb]{0.40,0.40,0.40}{##1}}}
\expandafter\def\csname PY@tok@mh\endcsname{\def\PY@tc##1{\textcolor[rgb]{0.40,0.40,0.40}{##1}}}
\expandafter\def\csname PY@tok@mi\endcsname{\def\PY@tc##1{\textcolor[rgb]{0.40,0.40,0.40}{##1}}}
\expandafter\def\csname PY@tok@il\endcsname{\def\PY@tc##1{\textcolor[rgb]{0.40,0.40,0.40}{##1}}}
\expandafter\def\csname PY@tok@mo\endcsname{\def\PY@tc##1{\textcolor[rgb]{0.40,0.40,0.40}{##1}}}
\expandafter\def\csname PY@tok@ch\endcsname{\let\PY@it=\textit\def\PY@tc##1{\textcolor[rgb]{0.25,0.50,0.50}{##1}}}
\expandafter\def\csname PY@tok@cm\endcsname{\let\PY@it=\textit\def\PY@tc##1{\textcolor[rgb]{0.25,0.50,0.50}{##1}}}
\expandafter\def\csname PY@tok@cpf\endcsname{\let\PY@it=\textit\def\PY@tc##1{\textcolor[rgb]{0.25,0.50,0.50}{##1}}}
\expandafter\def\csname PY@tok@c1\endcsname{\let\PY@it=\textit\def\PY@tc##1{\textcolor[rgb]{0.25,0.50,0.50}{##1}}}
\expandafter\def\csname PY@tok@cs\endcsname{\let\PY@it=\textit\def\PY@tc##1{\textcolor[rgb]{0.25,0.50,0.50}{##1}}}

\def\PYZbs{\char`\\}
\def\PYZus{\char`\_}
\def\PYZob{\char`\{}
\def\PYZcb{\char`\}}
\def\PYZca{\char`\^}
\def\PYZam{\char`\&}
\def\PYZlt{\char`\<}
\def\PYZgt{\char`\>}
\def\PYZsh{\char`\#}
\def\PYZpc{\char`\%}
\def\PYZdl{\char`\$}
\def\PYZhy{\char`\-}
\def\PYZsq{\char`\'}
\def\PYZdq{\char`\"}
\def\PYZti{\char`\~}
% for compatibility with earlier versions
\def\PYZat{@}
\def\PYZlb{[}
\def\PYZrb{]}
\makeatother


    % Exact colors from NB
    \definecolor{incolor}{rgb}{0.0, 0.0, 0.5}
    \definecolor{outcolor}{rgb}{0.545, 0.0, 0.0}



    
    % Prevent overflowing lines due to hard-to-break entities
    \sloppy 
    % Setup hyperref package
    \hypersetup{
      breaklinks=true,  % so long urls are correctly broken across lines
      colorlinks=true,
      urlcolor=urlcolor,
      linkcolor=linkcolor,
      citecolor=citecolor,
      }
    % Slightly bigger margins than the latex defaults
    
    \geometry{verbose,tmargin=1in,bmargin=1in,lmargin=1in,rmargin=1in}
    
    

    \begin{document}
    
    
    \maketitle
    
    

    
    \hypertarget{supervised-learning}{%
\subsection{Supervised Learning}\label{supervised-learning}}

\hypertarget{project-finding-donors-for-charityml}{%
\subsection{\texorpdfstring{Project: Finding Donors for
\emph{CharityML}}{Project: Finding Donors for CharityML}}\label{project-finding-donors-for-charityml}}

    In this notebook, some template code has already been provided for you,
and it will be your job to implement the additional functionality
necessary to successfully complete this project. Sections that begin
with \textbf{`Implementation'} in the header indicate that the following
block of code will require additional functionality which you must
provide. Instructions will be provided for each section and the
specifics of the implementation are marked in the code block with a
\texttt{\textquotesingle{}TODO\textquotesingle{}} statement. Please be
sure to read the instructions carefully!

In addition to implementing code, there will be questions that you must
answer which relate to the project and your implementation. Each section
where you will answer a question is preceded by a \textbf{`Question X'}
header. Carefully read each question and provide thorough answers in the
following text boxes that begin with \textbf{`Answer:'}. Your project
submission will be evaluated based on your answers to each of the
questions and the implementation you provide.

\begin{quote}
\textbf{Note:} Please specify WHICH VERSION OF PYTHON you are using when
submitting this notebook. Code and Markdown cells can be executed using
the \textbf{Shift + Enter} keyboard shortcut. In addition, Markdown
cells can be edited by typically double-clicking the cell to enter edit
mode.
\end{quote}

    \hypertarget{getting-started}{%
\subsection{Getting Started}\label{getting-started}}

In this project, you will employ several supervised algorithms of your
choice to accurately model individuals' income using data collected from
the 1994 U.S. Census. You will then choose the best candidate algorithm
from preliminary results and further optimize this algorithm to best
model the data. Your goal with this implementation is to construct a
model that accurately predicts whether an individual makes more than
\$50,000. This sort of task can arise in a non-profit setting, where
organizations survive on donations. Understanding an individual's income
can help a non-profit better understand how large of a donation to
request, or whether or not they should reach out to begin with. While it
can be difficult to determine an individual's general income bracket
directly from public sources, we can (as we will see) infer this value
from other publically available features.

The dataset for this project originates from the
\href{https://archive.ics.uci.edu/ml/datasets/Census+Income}{UCI Machine
Learning Repository}. The datset was donated by Ron Kohavi and Barry
Becker, after being published in the article \emph{``Scaling Up the
Accuracy of Naive-Bayes Classifiers: A Decision-Tree Hybrid''}. You can
find the article by Ron Kohavi
\href{https://www.aaai.org/Papers/KDD/1996/KDD96-033.pdf}{online}. The
data we investigate here consists of small changes to the original
dataset, such as removing the
\texttt{\textquotesingle{}fnlwgt\textquotesingle{}} feature and records
with missing or ill-formatted entries.

    \begin{center}\rule{0.5\linewidth}{\linethickness}\end{center}

\hypertarget{exploring-the-data}{%
\subsection{Exploring the Data}\label{exploring-the-data}}

Run the code cell below to load necessary Python libraries and load the
census data. Note that the last column from this dataset,
\texttt{\textquotesingle{}income\textquotesingle{}}, will be our target
label (whether an individual makes more than, or at most, \$50,000
annually). All other columns are features about each individual in the
census database.

    \begin{Verbatim}[commandchars=\\\{\}]
{\color{incolor}In [{\color{incolor}29}]:} \PY{c+c1}{\PYZsh{} Import libraries necessary for this project}
         \PY{k+kn}{import} \PY{n+nn}{numpy} \PY{k}{as} \PY{n+nn}{np}
         \PY{k+kn}{import} \PY{n+nn}{pandas} \PY{k}{as} \PY{n+nn}{pd}
         \PY{k+kn}{from} \PY{n+nn}{time} \PY{k}{import} \PY{n}{time}
         \PY{k+kn}{from} \PY{n+nn}{IPython}\PY{n+nn}{.}\PY{n+nn}{display} \PY{k}{import} \PY{n}{display} \PY{c+c1}{\PYZsh{} Allows the use of display() for DataFrames}
         
         \PY{c+c1}{\PYZsh{} Import supplementary visualization code visuals.py}
         \PY{k+kn}{import} \PY{n+nn}{visuals} \PY{k}{as} \PY{n+nn}{vs}
         
         \PY{c+c1}{\PYZsh{} Pretty display for notebooks}
         \PY{o}{\PYZpc{}}\PY{k}{matplotlib} inline
         
         \PY{c+c1}{\PYZsh{} Load the Census dataset}
         \PY{n}{data} \PY{o}{=} \PY{n}{pd}\PY{o}{.}\PY{n}{read\PYZus{}csv}\PY{p}{(}\PY{l+s+s2}{\PYZdq{}}\PY{l+s+s2}{census.csv}\PY{l+s+s2}{\PYZdq{}}\PY{p}{)}
         
         \PY{c+c1}{\PYZsh{} Success \PYZhy{} Display the first record}
         \PY{n}{display}\PY{p}{(}\PY{n}{data}\PY{o}{.}\PY{n}{head}\PY{p}{(}\PY{n}{n}\PY{o}{=}\PY{l+m+mi}{1}\PY{p}{)}\PY{p}{)}
\end{Verbatim}


    
    \begin{verbatim}
   age   workclass education_level  education-num  marital-status  \
0   39   State-gov       Bachelors           13.0   Never-married   

      occupation    relationship    race    sex  capital-gain  capital-loss  \
0   Adm-clerical   Not-in-family   White   Male        2174.0           0.0   

   hours-per-week  native-country income  
0            40.0   United-States  <=50K  
    \end{verbatim}

    
    \hypertarget{implementation-data-exploration}{%
\subsubsection{Implementation: Data
Exploration}\label{implementation-data-exploration}}

A cursory investigation of the dataset will determine how many
individuals fit into either group, and will tell us about the percentage
of these individuals making more than \$50,000. In the code cell below,
you will need to compute the following: - The total number of records,
\texttt{\textquotesingle{}n\_records\textquotesingle{}} - The number of
individuals making more than \$50,000 annually,
\texttt{\textquotesingle{}n\_greater\_50k\textquotesingle{}}. - The
number of individuals making at most \$50,000 annually,
\texttt{\textquotesingle{}n\_at\_most\_50k\textquotesingle{}}. - The
percentage of individuals making more than \$50,000 annually,
\texttt{\textquotesingle{}greater\_percent\textquotesingle{}}.

** HINT: ** You may need to look at the table above to understand how
the \texttt{\textquotesingle{}income\textquotesingle{}} entries are
formatted.

    \begin{Verbatim}[commandchars=\\\{\}]
{\color{incolor}In [{\color{incolor}44}]:} \PY{c+c1}{\PYZsh{} TODO: Total number of records}
         \PY{n}{n\PYZus{}records} \PY{o}{=} \PY{n+nb}{len}\PY{p}{(}\PY{n}{data}\PY{p}{)}
         
         \PY{c+c1}{\PYZsh{} TODO: Number of records where individual\PYZsq{}s income is more than \PYZdl{}50,000}
         \PY{n}{n\PYZus{}greater\PYZus{}50k} \PY{o}{=} \PY{n}{data}\PY{p}{[}\PY{n}{data}\PY{p}{[}\PY{l+s+s1}{\PYZsq{}}\PY{l+s+s1}{income}\PY{l+s+s1}{\PYZsq{}}\PY{p}{]}\PY{o}{==}\PY{l+s+s1}{\PYZsq{}}\PY{l+s+s1}{\PYZgt{}50K}\PY{l+s+s1}{\PYZsq{}}\PY{p}{]}\PY{o}{.}\PY{n}{shape}\PY{p}{[}\PY{l+m+mi}{0}\PY{p}{]}
         
         \PY{c+c1}{\PYZsh{} TODO: Number of records where individual\PYZsq{}s income is at most \PYZdl{}50,000}
         \PY{n}{n\PYZus{}at\PYZus{}most\PYZus{}50k} \PY{o}{=} \PY{n}{data}\PY{p}{[}\PY{n}{data}\PY{p}{[}\PY{l+s+s1}{\PYZsq{}}\PY{l+s+s1}{income}\PY{l+s+s1}{\PYZsq{}}\PY{p}{]}\PY{o}{==}\PY{l+s+s1}{\PYZsq{}}\PY{l+s+s1}{\PYZlt{}=50K}\PY{l+s+s1}{\PYZsq{}}\PY{p}{]}\PY{o}{.}\PY{n}{shape}\PY{p}{[}\PY{l+m+mi}{0}\PY{p}{]}
         
         \PY{c+c1}{\PYZsh{} TODO: Percentage of individuals whose income is more than \PYZdl{}50,000}
         \PY{n}{greater\PYZus{}percent} \PY{o}{=} \PY{n}{n\PYZus{}greater\PYZus{}50k} \PY{o}{/} \PY{p}{(}\PY{n}{n\PYZus{}greater\PYZus{}50k} \PY{o}{+} \PY{n}{n\PYZus{}at\PYZus{}most\PYZus{}50k}\PY{p}{)} \PY{o}{*} \PY{l+m+mi}{100}
         
         \PY{c+c1}{\PYZsh{} Print the results}
         \PY{n+nb}{print}\PY{p}{(}\PY{l+s+s2}{\PYZdq{}}\PY{l+s+s2}{Total number of records: }\PY{l+s+si}{\PYZob{}\PYZcb{}}\PY{l+s+s2}{\PYZdq{}}\PY{o}{.}\PY{n}{format}\PY{p}{(}\PY{n}{n\PYZus{}records}\PY{p}{)}\PY{p}{)}
         \PY{n+nb}{print}\PY{p}{(}\PY{l+s+s2}{\PYZdq{}}\PY{l+s+s2}{Individuals making more than \PYZdl{}50,000: }\PY{l+s+si}{\PYZob{}\PYZcb{}}\PY{l+s+s2}{\PYZdq{}}\PY{o}{.}\PY{n}{format}\PY{p}{(}\PY{n}{n\PYZus{}greater\PYZus{}50k}\PY{p}{)}\PY{p}{)}
         \PY{n+nb}{print}\PY{p}{(}\PY{l+s+s2}{\PYZdq{}}\PY{l+s+s2}{Individuals making at most \PYZdl{}50,000: }\PY{l+s+si}{\PYZob{}\PYZcb{}}\PY{l+s+s2}{\PYZdq{}}\PY{o}{.}\PY{n}{format}\PY{p}{(}\PY{n}{n\PYZus{}at\PYZus{}most\PYZus{}50k}\PY{p}{)}\PY{p}{)}
         \PY{n+nb}{print}\PY{p}{(}\PY{l+s+s2}{\PYZdq{}}\PY{l+s+s2}{Percentage of individuals making more than \PYZdl{}50,000: }\PY{l+s+si}{\PYZob{}\PYZcb{}}\PY{l+s+s2}{\PYZpc{}}\PY{l+s+s2}{\PYZdq{}}\PY{o}{.}\PY{n}{format}\PY{p}{(}\PY{n}{greater\PYZus{}percent}\PY{p}{)}\PY{p}{)}
\end{Verbatim}


    \begin{Verbatim}[commandchars=\\\{\}]
Total number of records: 45222
Individuals making more than \$50,000: 11208
Individuals making at most \$50,000: 34014
Percentage of individuals making more than \$50,000: 24.78439697492371\%

    \end{Verbatim}

    ** Featureset Exploration **

\begin{itemize}
\tightlist
\item
  \textbf{age}: continuous.
\item
  \textbf{workclass}: Private, Self-emp-not-inc, Self-emp-inc,
  Federal-gov, Local-gov, State-gov, Without-pay, Never-worked.
\item
  \textbf{education}: Bachelors, Some-college, 11th, HS-grad,
  Prof-school, Assoc-acdm, Assoc-voc, 9th, 7th-8th, 12th, Masters,
  1st-4th, 10th, Doctorate, 5th-6th, Preschool.
\item
  \textbf{education-num}: continuous.
\item
  \textbf{marital-status}: Married-civ-spouse, Divorced, Never-married,
  Separated, Widowed, Married-spouse-absent, Married-AF-spouse.
\item
  \textbf{occupation}: Tech-support, Craft-repair, Other-service, Sales,
  Exec-managerial, Prof-specialty, Handlers-cleaners, Machine-op-inspct,
  Adm-clerical, Farming-fishing, Transport-moving, Priv-house-serv,
  Protective-serv, Armed-Forces.
\item
  \textbf{relationship}: Wife, Own-child, Husband, Not-in-family,
  Other-relative, Unmarried.
\item
  \textbf{race}: Black, White, Asian-Pac-Islander, Amer-Indian-Eskimo,
  Other.
\item
  \textbf{sex}: Female, Male.
\item
  \textbf{capital-gain}: continuous.
\item
  \textbf{capital-loss}: continuous.
\item
  \textbf{hours-per-week}: continuous.
\item
  \textbf{native-country}: United-States, Cambodia, England,
  Puerto-Rico, Canada, Germany, Outlying-US(Guam-USVI-etc), India,
  Japan, Greece, South, China, Cuba, Iran, Honduras, Philippines, Italy,
  Poland, Jamaica, Vietnam, Mexico, Portugal, Ireland, France,
  Dominican-Republic, Laos, Ecuador, Taiwan, Haiti, Columbia, Hungary,
  Guatemala, Nicaragua, Scotland, Thailand, Yugoslavia, El-Salvador,
  Trinadad\&Tobago, Peru, Hong, Holand-Netherlands.
\end{itemize}

    \begin{center}\rule{0.5\linewidth}{\linethickness}\end{center}

\hypertarget{preparing-the-data}{%
\subsection{Preparing the Data}\label{preparing-the-data}}

Before data can be used as input for machine learning algorithms, it
often must be cleaned, formatted, and restructured --- this is typically
known as \textbf{preprocessing}. Fortunately, for this dataset, there
are no invalid or missing entries we must deal with, however, there are
some qualities about certain features that must be adjusted. This
preprocessing can help tremendously with the outcome and predictive
power of nearly all learning algorithms.

    \hypertarget{transforming-skewed-continuous-features}{%
\subsubsection{Transforming Skewed Continuous
Features}\label{transforming-skewed-continuous-features}}

A dataset may sometimes contain at least one feature whose values tend
to lie near a single number, but will also have a non-trivial number of
vastly larger or smaller values than that single number. Algorithms can
be sensitive to such distributions of values and can underperform if the
range is not properly normalized. With the census dataset two features
fit this description: '\texttt{capital-gain\textquotesingle{}} and
\texttt{\textquotesingle{}capital-loss\textquotesingle{}}.

Run the code cell below to plot a histogram of these two features. Note
the range of the values present and how they are distributed.

    \begin{Verbatim}[commandchars=\\\{\}]
{\color{incolor}In [{\color{incolor}31}]:} \PY{c+c1}{\PYZsh{} Split the data into features and target label}
         \PY{n}{income\PYZus{}raw} \PY{o}{=} \PY{n}{data}\PY{p}{[}\PY{l+s+s1}{\PYZsq{}}\PY{l+s+s1}{income}\PY{l+s+s1}{\PYZsq{}}\PY{p}{]}
         \PY{n}{features\PYZus{}raw} \PY{o}{=} \PY{n}{data}\PY{o}{.}\PY{n}{drop}\PY{p}{(}\PY{l+s+s1}{\PYZsq{}}\PY{l+s+s1}{income}\PY{l+s+s1}{\PYZsq{}}\PY{p}{,} \PY{n}{axis} \PY{o}{=} \PY{l+m+mi}{1}\PY{p}{)}
         
         \PY{c+c1}{\PYZsh{} Visualize skewed continuous features of original data}
         \PY{n}{vs}\PY{o}{.}\PY{n}{distribution}\PY{p}{(}\PY{n}{data}\PY{p}{)}
\end{Verbatim}


    \begin{center}
    \adjustimage{max size={0.9\linewidth}{0.9\paperheight}}{output_10_0.png}
    \end{center}
    { \hspace*{\fill} \\}
    
    For highly-skewed feature distributions such as
\texttt{\textquotesingle{}capital-gain\textquotesingle{}} and
\texttt{\textquotesingle{}capital-loss\textquotesingle{}}, it is common
practice to apply a logarithmic transformation on the data so that the
very large and very small values do not negatively affect the
performance of a learning algorithm. Using a logarithmic transformation
significantly reduces the range of values caused by outliers. Care must
be taken when applying this transformation however: The logarithm of
\texttt{0} is undefined, so we must translate the values by a small
amount above \texttt{0} to apply the the logarithm successfully.

Run the code cell below to perform a transformation on the data and
visualize the results. Again, note the range of values and how they are
distributed.

    \begin{Verbatim}[commandchars=\\\{\}]
{\color{incolor}In [{\color{incolor}32}]:} \PY{c+c1}{\PYZsh{} Log\PYZhy{}transform the skewed features}
         \PY{n}{skewed} \PY{o}{=} \PY{p}{[}\PY{l+s+s1}{\PYZsq{}}\PY{l+s+s1}{capital\PYZhy{}gain}\PY{l+s+s1}{\PYZsq{}}\PY{p}{,} \PY{l+s+s1}{\PYZsq{}}\PY{l+s+s1}{capital\PYZhy{}loss}\PY{l+s+s1}{\PYZsq{}}\PY{p}{]}
         \PY{n}{features\PYZus{}log\PYZus{}transformed} \PY{o}{=} \PY{n}{pd}\PY{o}{.}\PY{n}{DataFrame}\PY{p}{(}\PY{n}{data} \PY{o}{=} \PY{n}{features\PYZus{}raw}\PY{p}{)}
         \PY{n}{features\PYZus{}log\PYZus{}transformed}\PY{p}{[}\PY{n}{skewed}\PY{p}{]} \PY{o}{=} \PY{n}{features\PYZus{}raw}\PY{p}{[}\PY{n}{skewed}\PY{p}{]}\PY{o}{.}\PY{n}{apply}\PY{p}{(}\PY{k}{lambda} \PY{n}{x}\PY{p}{:} \PY{n}{np}\PY{o}{.}\PY{n}{log}\PY{p}{(}\PY{n}{x} \PY{o}{+} \PY{l+m+mi}{1}\PY{p}{)}\PY{p}{)}
         
         \PY{c+c1}{\PYZsh{} Visualize the new log distributions}
         \PY{n}{vs}\PY{o}{.}\PY{n}{distribution}\PY{p}{(}\PY{n}{features\PYZus{}log\PYZus{}transformed}\PY{p}{,} \PY{n}{transformed} \PY{o}{=} \PY{k+kc}{True}\PY{p}{)}
\end{Verbatim}


    \begin{center}
    \adjustimage{max size={0.9\linewidth}{0.9\paperheight}}{output_12_0.png}
    \end{center}
    { \hspace*{\fill} \\}
    
    \hypertarget{normalizing-numerical-features}{%
\subsubsection{Normalizing Numerical
Features}\label{normalizing-numerical-features}}

In addition to performing transformations on features that are highly
skewed, it is often good practice to perform some type of scaling on
numerical features. Applying a scaling to the data does not change the
shape of each feature's distribution (such as
\texttt{\textquotesingle{}capital-gain\textquotesingle{}} or
\texttt{\textquotesingle{}capital-loss\textquotesingle{}} above);
however, normalization ensures that each feature is treated equally when
applying supervised learners. Note that once scaling is applied,
observing the data in its raw form will no longer have the same original
meaning, as exampled below.

Run the code cell below to normalize each numerical feature. We will use
\href{http://scikit-learn.org/stable/modules/generated/sklearn.preprocessing.MinMaxScaler.html}{\texttt{sklearn.preprocessing.MinMaxScaler}}
for this.

    \begin{Verbatim}[commandchars=\\\{\}]
{\color{incolor}In [{\color{incolor}33}]:} \PY{c+c1}{\PYZsh{} Import sklearn.preprocessing.StandardScaler}
         \PY{k+kn}{from} \PY{n+nn}{sklearn}\PY{n+nn}{.}\PY{n+nn}{preprocessing} \PY{k}{import} \PY{n}{MinMaxScaler}
         
         \PY{c+c1}{\PYZsh{} Initialize a scaler, then apply it to the features}
         \PY{n}{scaler} \PY{o}{=} \PY{n}{MinMaxScaler}\PY{p}{(}\PY{p}{)} \PY{c+c1}{\PYZsh{} default=(0, 1)}
         \PY{n}{numerical} \PY{o}{=} \PY{p}{[}\PY{l+s+s1}{\PYZsq{}}\PY{l+s+s1}{age}\PY{l+s+s1}{\PYZsq{}}\PY{p}{,} \PY{l+s+s1}{\PYZsq{}}\PY{l+s+s1}{education\PYZhy{}num}\PY{l+s+s1}{\PYZsq{}}\PY{p}{,} \PY{l+s+s1}{\PYZsq{}}\PY{l+s+s1}{capital\PYZhy{}gain}\PY{l+s+s1}{\PYZsq{}}\PY{p}{,} \PY{l+s+s1}{\PYZsq{}}\PY{l+s+s1}{capital\PYZhy{}loss}\PY{l+s+s1}{\PYZsq{}}\PY{p}{,} \PY{l+s+s1}{\PYZsq{}}\PY{l+s+s1}{hours\PYZhy{}per\PYZhy{}week}\PY{l+s+s1}{\PYZsq{}}\PY{p}{]}
         
         \PY{n}{features\PYZus{}log\PYZus{}minmax\PYZus{}transform} \PY{o}{=} \PY{n}{pd}\PY{o}{.}\PY{n}{DataFrame}\PY{p}{(}\PY{n}{data} \PY{o}{=} \PY{n}{features\PYZus{}log\PYZus{}transformed}\PY{p}{)}
         \PY{n}{features\PYZus{}log\PYZus{}minmax\PYZus{}transform}\PY{p}{[}\PY{n}{numerical}\PY{p}{]} \PY{o}{=} \PY{n}{scaler}\PY{o}{.}\PY{n}{fit\PYZus{}transform}\PY{p}{(}\PY{n}{features\PYZus{}log\PYZus{}transformed}\PY{p}{[}\PY{n}{numerical}\PY{p}{]}\PY{p}{)}
         
         \PY{c+c1}{\PYZsh{} Show an example of a record with scaling applied}
         \PY{n}{display}\PY{p}{(}\PY{n}{features\PYZus{}log\PYZus{}minmax\PYZus{}transform}\PY{o}{.}\PY{n}{head}\PY{p}{(}\PY{n}{n} \PY{o}{=} \PY{l+m+mi}{5}\PY{p}{)}\PY{p}{)}
\end{Verbatim}


    
    \begin{verbatim}
        age          workclass education_level  education-num  \
0  0.301370          State-gov       Bachelors       0.800000   
1  0.452055   Self-emp-not-inc       Bachelors       0.800000   
2  0.287671            Private         HS-grad       0.533333   
3  0.493151            Private            11th       0.400000   
4  0.150685            Private       Bachelors       0.800000   

        marital-status          occupation    relationship    race      sex  \
0        Never-married        Adm-clerical   Not-in-family   White     Male   
1   Married-civ-spouse     Exec-managerial         Husband   White     Male   
2             Divorced   Handlers-cleaners   Not-in-family   White     Male   
3   Married-civ-spouse   Handlers-cleaners         Husband   Black     Male   
4   Married-civ-spouse      Prof-specialty            Wife   Black   Female   

   capital-gain  capital-loss  hours-per-week  native-country  
0      0.667492           0.0        0.397959   United-States  
1      0.000000           0.0        0.122449   United-States  
2      0.000000           0.0        0.397959   United-States  
3      0.000000           0.0        0.397959   United-States  
4      0.000000           0.0        0.397959            Cuba  
    \end{verbatim}

    
    \hypertarget{implementation-data-preprocessing}{%
\subsubsection{Implementation: Data
Preprocessing}\label{implementation-data-preprocessing}}

From the table in \textbf{Exploring the Data} above, we can see there
are several features for each record that are non-numeric. Typically,
learning algorithms expect input to be numeric, which requires that
non-numeric features (called \emph{categorical variables}) be converted.
One popular way to convert categorical variables is by using the
\textbf{one-hot encoding} scheme. One-hot encoding creates a
\emph{``dummy''} variable for each possible category of each non-numeric
feature. For example, assume \texttt{someFeature} has three possible
entries: \texttt{A}, \texttt{B}, or \texttt{C}. We then encode this
feature into \texttt{someFeature\_A}, \texttt{someFeature\_B} and
\texttt{someFeature\_C}.

~~\textbar{} someFeature \textbar{} \textbar{} someFeature\_A \textbar{}
someFeature\_B \textbar{} someFeature\_C \textbar{}\\
:-: \textbar{} :-: \textbar{} \textbar{} :-: \textbar{} :-: \textbar{}
:-: \textbar{}\\
0 \textbar{} B \textbar{} \textbar{} 0 \textbar{} 1 \textbar{} 0
\textbar{}\\
1 \textbar{} C \textbar{} ----\textgreater{} one-hot encode
----\textgreater{} \textbar{} 0 \textbar{} 0 \textbar{} 1 \textbar{}\\
2 \textbar{} A \textbar{} \textbar{} 1 \textbar{} 0 \textbar{} 0
\textbar{}

Additionally, as with the non-numeric features, we need to convert the
non-numeric target label,
\texttt{\textquotesingle{}income\textquotesingle{}} to numerical values
for the learning algorithm to work. Since there are only two possible
categories for this label (``\textless{}=50K'' and
``\textgreater{}50K''), we can avoid using one-hot encoding and simply
encode these two categories as \texttt{0} and \texttt{1}, respectively.
In code cell below, you will need to implement the following: - Use
\href{http://pandas.pydata.org/pandas-docs/stable/generated/pandas.get_dummies.html?highlight=get_dummies\#pandas.get_dummies}{\texttt{pandas.get\_dummies()}}
to perform one-hot encoding on the
\texttt{\textquotesingle{}features\_log\_minmax\_transform\textquotesingle{}}
data. - Convert the target label
\texttt{\textquotesingle{}income\_raw\textquotesingle{}} to numerical
entries. - Set records with ``\textless{}=50K'' to \texttt{0} and
records with ``\textgreater{}50K'' to \texttt{1}.

    \begin{Verbatim}[commandchars=\\\{\}]
{\color{incolor}In [{\color{incolor}34}]:} \PY{c+c1}{\PYZsh{} TODO: One\PYZhy{}hot encode the \PYZsq{}features\PYZus{}log\PYZus{}minmax\PYZus{}transform\PYZsq{} data using pandas.get\PYZus{}dummies()}
         \PY{n}{features\PYZus{}final} \PY{o}{=} \PY{n}{pd}\PY{o}{.}\PY{n}{get\PYZus{}dummies}\PY{p}{(}\PY{n}{features\PYZus{}log\PYZus{}minmax\PYZus{}transform}\PY{p}{)}
         
         \PY{c+c1}{\PYZsh{} TODO: Encode the \PYZsq{}income\PYZus{}raw\PYZsq{} data to numerical values}
         \PY{n}{income} \PY{o}{=} \PY{n}{income\PYZus{}raw}\PY{o}{.}\PY{n}{replace}\PY{p}{(}\PY{p}{[}\PY{l+s+s1}{\PYZsq{}}\PY{l+s+s1}{\PYZgt{}50K}\PY{l+s+s1}{\PYZsq{}}\PY{p}{,} \PY{l+s+s1}{\PYZsq{}}\PY{l+s+s1}{\PYZlt{}=50K}\PY{l+s+s1}{\PYZsq{}}\PY{p}{]}\PY{p}{,} \PY{p}{[}\PY{l+m+mi}{1}\PY{p}{,} \PY{l+m+mi}{0}\PY{p}{]}\PY{p}{)}
         
         \PY{c+c1}{\PYZsh{} Print the number of features after one\PYZhy{}hot encoding}
         \PY{n}{encoded} \PY{o}{=} \PY{n+nb}{list}\PY{p}{(}\PY{n}{features\PYZus{}final}\PY{o}{.}\PY{n}{columns}\PY{p}{)}
         \PY{n+nb}{print}\PY{p}{(}\PY{l+s+s2}{\PYZdq{}}\PY{l+s+si}{\PYZob{}\PYZcb{}}\PY{l+s+s2}{ total features after one\PYZhy{}hot encoding.}\PY{l+s+s2}{\PYZdq{}}\PY{o}{.}\PY{n}{format}\PY{p}{(}\PY{n+nb}{len}\PY{p}{(}\PY{n}{encoded}\PY{p}{)}\PY{p}{)}\PY{p}{)}
         
         \PY{c+c1}{\PYZsh{} Uncomment the following line to see the encoded feature names}
         \PY{n+nb}{print} \PY{p}{(}\PY{n}{encoded}\PY{p}{)}
\end{Verbatim}


    \begin{Verbatim}[commandchars=\\\{\}]
103 total features after one-hot encoding.
['age', 'education-num', 'capital-gain', 'capital-loss', 'hours-per-week', 'workclass\_ Federal-gov', 'workclass\_ Local-gov', 'workclass\_ Private', 'workclass\_ Self-emp-inc', 'workclass\_ Self-emp-not-inc', 'workclass\_ State-gov', 'workclass\_ Without-pay', 'education\_level\_ 10th', 'education\_level\_ 11th', 'education\_level\_ 12th', 'education\_level\_ 1st-4th', 'education\_level\_ 5th-6th', 'education\_level\_ 7th-8th', 'education\_level\_ 9th', 'education\_level\_ Assoc-acdm', 'education\_level\_ Assoc-voc', 'education\_level\_ Bachelors', 'education\_level\_ Doctorate', 'education\_level\_ HS-grad', 'education\_level\_ Masters', 'education\_level\_ Preschool', 'education\_level\_ Prof-school', 'education\_level\_ Some-college', 'marital-status\_ Divorced', 'marital-status\_ Married-AF-spouse', 'marital-status\_ Married-civ-spouse', 'marital-status\_ Married-spouse-absent', 'marital-status\_ Never-married', 'marital-status\_ Separated', 'marital-status\_ Widowed', 'occupation\_ Adm-clerical', 'occupation\_ Armed-Forces', 'occupation\_ Craft-repair', 'occupation\_ Exec-managerial', 'occupation\_ Farming-fishing', 'occupation\_ Handlers-cleaners', 'occupation\_ Machine-op-inspct', 'occupation\_ Other-service', 'occupation\_ Priv-house-serv', 'occupation\_ Prof-specialty', 'occupation\_ Protective-serv', 'occupation\_ Sales', 'occupation\_ Tech-support', 'occupation\_ Transport-moving', 'relationship\_ Husband', 'relationship\_ Not-in-family', 'relationship\_ Other-relative', 'relationship\_ Own-child', 'relationship\_ Unmarried', 'relationship\_ Wife', 'race\_ Amer-Indian-Eskimo', 'race\_ Asian-Pac-Islander', 'race\_ Black', 'race\_ Other', 'race\_ White', 'sex\_ Female', 'sex\_ Male', 'native-country\_ Cambodia', 'native-country\_ Canada', 'native-country\_ China', 'native-country\_ Columbia', 'native-country\_ Cuba', 'native-country\_ Dominican-Republic', 'native-country\_ Ecuador', 'native-country\_ El-Salvador', 'native-country\_ England', 'native-country\_ France', 'native-country\_ Germany', 'native-country\_ Greece', 'native-country\_ Guatemala', 'native-country\_ Haiti', 'native-country\_ Holand-Netherlands', 'native-country\_ Honduras', 'native-country\_ Hong', 'native-country\_ Hungary', 'native-country\_ India', 'native-country\_ Iran', 'native-country\_ Ireland', 'native-country\_ Italy', 'native-country\_ Jamaica', 'native-country\_ Japan', 'native-country\_ Laos', 'native-country\_ Mexico', 'native-country\_ Nicaragua', 'native-country\_ Outlying-US(Guam-USVI-etc)', 'native-country\_ Peru', 'native-country\_ Philippines', 'native-country\_ Poland', 'native-country\_ Portugal', 'native-country\_ Puerto-Rico', 'native-country\_ Scotland', 'native-country\_ South', 'native-country\_ Taiwan', 'native-country\_ Thailand', 'native-country\_ Trinadad\&Tobago', 'native-country\_ United-States', 'native-country\_ Vietnam', 'native-country\_ Yugoslavia']

    \end{Verbatim}

    \hypertarget{shuffle-and-split-data}{%
\subsubsection{Shuffle and Split Data}\label{shuffle-and-split-data}}

Now all \emph{categorical variables} have been converted into numerical
features, and all numerical features have been normalized. As always, we
will now split the data (both features and their labels) into training
and test sets. 80\% of the data will be used for training and 20\% for
testing.

Run the code cell below to perform this split.

    \begin{Verbatim}[commandchars=\\\{\}]
{\color{incolor}In [{\color{incolor}35}]:} \PY{c+c1}{\PYZsh{} Import train\PYZus{}test\PYZus{}split}
         \PY{k+kn}{from} \PY{n+nn}{sklearn}\PY{n+nn}{.}\PY{n+nn}{model\PYZus{}selection} \PY{k}{import} \PY{n}{train\PYZus{}test\PYZus{}split}
         
         \PY{c+c1}{\PYZsh{} Split the \PYZsq{}features\PYZsq{} and \PYZsq{}income\PYZsq{} data into training and testing sets}
         \PY{n}{X\PYZus{}train}\PY{p}{,} \PY{n}{X\PYZus{}test}\PY{p}{,} \PY{n}{y\PYZus{}train}\PY{p}{,} \PY{n}{y\PYZus{}test} \PY{o}{=} \PY{n}{train\PYZus{}test\PYZus{}split}\PY{p}{(}\PY{n}{features\PYZus{}final}\PY{p}{,} 
                                                             \PY{n}{income}\PY{p}{,} 
                                                             \PY{n}{test\PYZus{}size} \PY{o}{=} \PY{l+m+mf}{0.2}\PY{p}{,} 
                                                             \PY{n}{random\PYZus{}state} \PY{o}{=} \PY{l+m+mi}{0}\PY{p}{)}
         
         \PY{c+c1}{\PYZsh{} Show the results of the split}
         \PY{n+nb}{print}\PY{p}{(}\PY{l+s+s2}{\PYZdq{}}\PY{l+s+s2}{Training set has }\PY{l+s+si}{\PYZob{}\PYZcb{}}\PY{l+s+s2}{ samples.}\PY{l+s+s2}{\PYZdq{}}\PY{o}{.}\PY{n}{format}\PY{p}{(}\PY{n}{X\PYZus{}train}\PY{o}{.}\PY{n}{shape}\PY{p}{[}\PY{l+m+mi}{0}\PY{p}{]}\PY{p}{)}\PY{p}{)}
         \PY{n+nb}{print}\PY{p}{(}\PY{l+s+s2}{\PYZdq{}}\PY{l+s+s2}{Testing set has }\PY{l+s+si}{\PYZob{}\PYZcb{}}\PY{l+s+s2}{ samples.}\PY{l+s+s2}{\PYZdq{}}\PY{o}{.}\PY{n}{format}\PY{p}{(}\PY{n}{X\PYZus{}test}\PY{o}{.}\PY{n}{shape}\PY{p}{[}\PY{l+m+mi}{0}\PY{p}{]}\PY{p}{)}\PY{p}{)}
\end{Verbatim}


    \begin{Verbatim}[commandchars=\\\{\}]
Training set has 36177 samples.
Testing set has 9045 samples.

    \end{Verbatim}

    \begin{center}\rule{0.5\linewidth}{\linethickness}\end{center}

\hypertarget{evaluating-model-performance}{%
\subsection{Evaluating Model
Performance}\label{evaluating-model-performance}}

In this section, we will investigate four different algorithms, and
determine which is best at modeling the data. Three of these algorithms
will be supervised learners of your choice, and the fourth algorithm is
known as a \emph{naive predictor}.

    \hypertarget{metrics-and-the-naive-predictor}{%
\subsubsection{Metrics and the Naive
Predictor}\label{metrics-and-the-naive-predictor}}

\emph{CharityML}, equipped with their research, knows individuals that
make more than \$50,000 are most likely to donate to their charity.
Because of this, \emph{CharityML} is particularly interested in
predicting who makes more than \$50,000 accurately. It would seem that
using \textbf{accuracy} as a metric for evaluating a particular model's
performace would be appropriate. Additionally, identifying someone that
\emph{does not} make more than \$50,000 as someone who does would be
detrimental to \emph{CharityML}, since they are looking to find
individuals willing to donate. Therefore, a model's ability to precisely
predict those that make more than \$50,000 is \emph{more important} than
the model's ability to \textbf{recall} those individuals. We can use
\textbf{F-beta score} as a metric that considers both precision and
recall:

\[ F_{\beta} = (1 + \beta^2) \cdot \frac{precision \cdot recall}{\left( \beta^2 \cdot precision \right) + recall} \]

In particular, when \(\beta = 0.5\), more emphasis is placed on
precision. This is called the \textbf{F\(_{0.5}\) score} (or F-score for
simplicity).

Looking at the distribution of classes (those who make at most \$50,000,
and those who make more), it's clear most individuals do not make more
than \$50,000. This can greatly affect \textbf{accuracy}, since we could
simply say \emph{``this person does not make more than \$50,000''} and
generally be right, without ever looking at the data! Making such a
statement would be called \textbf{naive}, since we have not considered
any information to substantiate the claim. It is always important to
consider the \emph{naive prediction} for your data, to help establish a
benchmark for whether a model is performing well. That been said, using
that prediction would be pointless: If we predicted all people made less
than \$50,000, \emph{CharityML} would identify no one as donors.

\hypertarget{note-recap-of-accuracy-precision-recall}{%
\paragraph{Note: Recap of accuracy, precision,
recall}\label{note-recap-of-accuracy-precision-recall}}

** Accuracy ** measures how often the classifier makes the correct
prediction. It's the ratio of the number of correct predictions to the
total number of predictions (the number of test data points).

** Precision ** tells us what proportion of messages we classified as
spam, actually were spam. It is a ratio of true positives(words
classified as spam, and which are actually spam) to all positives(all
words classified as spam, irrespective of whether that was the correct
classificatio), in other words it is the ratio of

\texttt{{[}True\ Positives/(True\ Positives\ +\ False\ Positives){]}}

** Recall(sensitivity)** tells us what proportion of messages that
actually were spam were classified by us as spam. It is a ratio of true
positives(words classified as spam, and which are actually spam) to all
the words that were actually spam, in other words it is the ratio of

\texttt{{[}True\ Positives/(True\ Positives\ +\ False\ Negatives){]}}

For classification problems that are skewed in their classification
distributions like in our case, for example if we had a 100 text
messages and only 2 were spam and the rest 98 weren't, accuracy by
itself is not a very good metric. We could classify 90 messages as not
spam(including the 2 that were spam but we classify them as not spam,
hence they would be false negatives) and 10 as spam(all 10 false
positives) and still get a reasonably good accuracy score. For such
cases, precision and recall come in very handy. These two metrics can be
combined to get the F1 score, which is weighted average(harmonic mean)
of the precision and recall scores. This score can range from 0 to 1,
with 1 being the best possible F1 score(we take the harmonic mean as we
are dealing with ratios).

    \hypertarget{question-1---naive-predictor-performace}{%
\subsubsection{Question 1 - Naive Predictor
Performace}\label{question-1---naive-predictor-performace}}

\begin{itemize}
\tightlist
\item
  If we chose a model that always predicted an individual made more than
  \$50,000, what would that model's accuracy and F-score be on this
  dataset? You must use the code cell below and assign your results to
  \texttt{\textquotesingle{}accuracy\textquotesingle{}} and
  \texttt{\textquotesingle{}fscore\textquotesingle{}} to be used later.
\end{itemize}

** Please note ** that the the purpose of generating a naive predictor
is simply to show what a base model without any intelligence would look
like. In the real world, ideally your base model would be either the
results of a previous model or could be based on a research paper upon
which you are looking to improve. When there is no benchmark model set,
getting a result better than random choice is a place you could start
from.

** HINT: **

\begin{itemize}
\tightlist
\item
  When we have a model that always predicts `1' (i.e.~the individual
  makes more than 50k) then our model will have no True Negatives(TN) or
  False Negatives(FN) as we are not making any negative(`0' value)
  predictions. Therefore our Accuracy in this case becomes the same as
  our Precision(True Positives/(True Positives + False Positives)) as
  every prediction that we have made with value `1' that should have `0'
  becomes a False Positive; therefore our denominator in this case is
  the total number of records we have in total.
\item
  Our Recall score(True Positives/(True Positives + False Negatives)) in
  this setting becomes 1 as we have no False Negatives.
\end{itemize}

    \begin{Verbatim}[commandchars=\\\{\}]
{\color{incolor}In [{\color{incolor}36}]:} \PY{l+s+sd}{\PYZsq{}\PYZsq{}\PYZsq{}}
         \PY{l+s+sd}{TP = np.sum(income) \PYZsh{} Counting the ones as this is the naive case. Note that \PYZsq{}income\PYZsq{} is the \PYZsq{}income\PYZus{}raw\PYZsq{} data }
         \PY{l+s+sd}{encoded to numerical values done in the data preprocessing step.}
         \PY{l+s+sd}{FP = income.count() \PYZhy{} TP \PYZsh{} Specific to the naive case}
         
         \PY{l+s+sd}{TN = 0 \PYZsh{} No predicted negatives in the naive case}
         \PY{l+s+sd}{FN = 0 \PYZsh{} No predicted negatives in the naive case}
         \PY{l+s+sd}{\PYZsq{}\PYZsq{}\PYZsq{}}
         \PY{c+c1}{\PYZsh{} TODO: Calculate accuracy, precision and recall}
         \PY{n}{accuracy} \PY{o}{=} \PY{n}{np}\PY{o}{.}\PY{n}{sum}\PY{p}{(}\PY{n}{income}\PY{p}{)} \PY{o}{/} \PY{p}{(}\PY{n}{np}\PY{o}{.}\PY{n}{sum}\PY{p}{(}\PY{n}{income}\PY{p}{)} \PY{o}{+} \PY{p}{(}\PY{n}{income}\PY{o}{.}\PY{n}{count}\PY{p}{(}\PY{p}{)} \PY{o}{\PYZhy{}} \PY{n}{np}\PY{o}{.}\PY{n}{sum}\PY{p}{(}\PY{n}{income}\PY{p}{)}\PY{p}{)}\PY{p}{)}
         \PY{n}{recall} \PY{o}{=} \PY{n}{np}\PY{o}{.}\PY{n}{sum}\PY{p}{(}\PY{n}{income}\PY{p}{)} \PY{o}{/} \PY{p}{(}\PY{n}{np}\PY{o}{.}\PY{n}{sum}\PY{p}{(}\PY{n}{income}\PY{p}{)} \PY{o}{+} \PY{l+m+mi}{0}\PY{p}{)}
         \PY{n}{precision} \PY{o}{=} \PY{n}{np}\PY{o}{.}\PY{n}{sum}\PY{p}{(}\PY{n}{income}\PY{p}{)} \PY{o}{/} \PY{p}{(}\PY{n}{np}\PY{o}{.}\PY{n}{sum}\PY{p}{(}\PY{n}{income}\PY{p}{)} \PY{o}{+} \PY{p}{(}\PY{n}{income}\PY{o}{.}\PY{n}{count}\PY{p}{(}\PY{p}{)} \PY{o}{\PYZhy{}} \PY{n}{np}\PY{o}{.}\PY{n}{sum}\PY{p}{(}\PY{n}{income}\PY{p}{)}\PY{p}{)}\PY{p}{)}
         
         \PY{c+c1}{\PYZsh{} TODO: Calculate F\PYZhy{}score using the formula above for beta = 0.5 and correct values for precision and recall.}
         \PY{n}{beta} \PY{o}{=} \PY{l+m+mf}{0.5}
         \PY{n}{fscore} \PY{o}{=} \PY{p}{(}\PY{l+m+mi}{1} \PY{o}{+} \PY{n}{beta} \PY{o}{*}\PY{o}{*} \PY{l+m+mi}{2}\PY{p}{)} \PY{o}{*} \PY{p}{(}\PY{p}{(}\PY{n}{precision} \PY{o}{*} \PY{n}{recall}\PY{p}{)} \PY{o}{/} \PY{p}{(}\PY{p}{(}\PY{p}{(}\PY{n}{beta} \PY{o}{*}\PY{o}{*} \PY{l+m+mi}{2}\PY{p}{)} \PY{o}{*} \PY{n}{precision}\PY{p}{)} \PY{o}{+} \PY{n}{recall}\PY{p}{)}\PY{p}{)}
         
         \PY{c+c1}{\PYZsh{} Print the results }
         \PY{n+nb}{print}\PY{p}{(}\PY{l+s+s2}{\PYZdq{}}\PY{l+s+s2}{Naive Predictor: [Accuracy score: }\PY{l+s+si}{\PYZob{}:.4f\PYZcb{}}\PY{l+s+s2}{, F\PYZhy{}score: }\PY{l+s+si}{\PYZob{}:.4f\PYZcb{}}\PY{l+s+s2}{]}\PY{l+s+s2}{\PYZdq{}}\PY{o}{.}\PY{n}{format}\PY{p}{(}\PY{n}{accuracy}\PY{p}{,} \PY{n}{fscore}\PY{p}{)}\PY{p}{)}
\end{Verbatim}


    \begin{Verbatim}[commandchars=\\\{\}]
Naive Predictor: [Accuracy score: 0.2478, F-score: 0.2917]

    \end{Verbatim}

    \hypertarget{supervised-learning-models}{%
\subsubsection{Supervised Learning
Models}\label{supervised-learning-models}}

\textbf{The following are some of the supervised learning models that
are currently available in}
\href{http://scikit-learn.org/stable/supervised_learning.html}{\texttt{scikit-learn}}
\textbf{that you may choose from:} - Gaussian Naive Bayes (GaussianNB) -
Decision Trees - Ensemble Methods (Bagging, AdaBoost, Random Forest,
Gradient Boosting) - K-Nearest Neighbors (KNeighbors) - Stochastic
Gradient Descent Classifier (SGDC) - Support Vector Machines (SVM) -
Logistic Regression

    \hypertarget{question-2---model-application}{%
\subsubsection{Question 2 - Model
Application}\label{question-2---model-application}}

List three of the supervised learning models above that are appropriate
for this problem that you will test on the census data. For each model
chosen

\begin{itemize}
\tightlist
\item
  Describe one real-world application in industry where the model can be
  applied.
\item
  What are the strengths of the model; when does it perform well?
\item
  What are the weaknesses of the model; when does it perform poorly?
\item
  What makes this model a good candidate for the problem, given what you
  know about the data?
\end{itemize}

** HINT: **

Structure your answer in the same format as above\^{}, with 4 parts for
each of the three models you pick. Please include references with your
answer.

    \textbf{Answer: }

\hypertarget{random-forest}{%
\subsubsection{Random Forest}\label{random-forest}}

** Describe one real-world application in industry where the model can
be applied. **

Random Forest can be used to predict the likely sales performance of a
new product by using comprehensive sales and attribute data for similar
products that were launched in the past.

** What are the strengths of the model; when does it perform well? **

It can be used for both classification and regression tasks. It produces
great results in most cases, even without hyper-parameter tuning. The
hyper-parameters are generally easy to understand compared to other
models. Also, generated forests can be saved for future use on other
data.

** What are the weaknesses of the model; when does it perform poorly? **

The main weakness of this model is the fact that it can take too long to
predict in situations where training speed is a major factor and there
are too many features.

** What makes this model a good candidate for the problem, given what
you know about the data? **

This model is a good candidate for the problem because the data has a
high number and variety of categorical and numerical features. Random
Forest implements a combination of random decision trees and averages
the results to produce close predictions. Additionally, it searches for
the best feature among each random subset of features, which contribute
to its renown ability to product satisfactory results.

** Sources: **

•
https://towardsdatascience.com/the-random-forest-algorithm-d457d499ffcd

•
https://www.stat.berkeley.edu/\textasciitilde{}breiman/RandomForests/cc\_home.htm

\hypertarget{support-vector-machines-svm}{%
\subsubsection{Support Vector Machines
(SVM)}\label{support-vector-machines-svm}}

** Describe one real-world application in industry where the model can
be applied. **

In bioinformatics, a Support Vector Machine can be used to classify
patients' likelihood of succumbing to illnesses on the basis of their
genes.

** What are the strengths of the model; when does it perform well? **

This algorithm outputs an optimal hyperplane that separates
classifications to correctly predict classifications when applied to new
data sets. In many scenarios, it can even successfully segregate
classifications in multi-dimensional space by transforming data using
linear algebra.

** What are the weaknesses of the model; when does it perform poorly? **

On large data sets, SVMs are known for having training times that are
multitudes slower than other models. Also, choosing the appropriate
kernel for a model proves difficult in many cases.

** What makes this model a good candidate for the problem, given what
you know about the data? **

The data set isn't unreasonably large to train with using SVM, although
the training time is noticeably extensive. Also, the label only has two
classes, which reduces the required time to train (one model is needed
per classification).

** Sources: **

• https://www.quora.com/What-are-the-disadvantages-of-SVM-algorithms

•
https://stats.stackexchange.com/questions/24437/advantages-and-disadvantages-of-svm

• https://data-flair.training/blogs/applications-of-svm/

•
https://medium.com/machine-learning-101/chapter-2-svm-support-vector-machine-theory-f0812effc72

•
https://machinelearningmastery.com/support-vector-machines-for-machine-learning/

\hypertarget{adaboost}{%
\subsubsection{AdaBoost}\label{adaboost}}

** Describe one real-world application in industry where the model can
be applied. **

Adaboost algorithms can be used to detect whether an object on an image
is a face. This is possible after a gray scale transformation is
conducted on an image and a thresholds are approximated to create face
boundaries.

** What are the strengths of the model; when does it perform well? **

Adaboost, like other models that fall under ``boosting,'' is an ensemble
method that creates a strong classifier from numerous weak classifiers.
To clarify, weak classifiers'' or ``weak learners'' are predictive
models that generate predictions with accuracies that are barely above
chance.

Adaboost is most suited when used with decision trees with one level
(decision stumps). Decision stumps are decision trees that learn by only
considering the attributes under one field when making a prediction
(i.e., male or female) in a dataset that would require multiple,
branching decisions involving the attributes under multiple fields to
derive a highly accurate prediction. Generating a prediction for the
dataset of this exercise by only considering gender would lead to a
highly inaccurate prediction.

Additionally, Adaboost was specifically developed for binary
classifications.

** What are the weaknesses of the model; when does it perform poorly? **

Adaboost is only used for classification, and not regression.

Also, Adaboost is prone to overfitting under the following conditions: •
When used with complicated weak learners such as decision trees over one
level and models involving hyperplanes in multi-dimensional space • When
used on noisy datasets (excessive outliers, etc.) • When used on
high-dimensional datasets

** What makes this model a good candidate for the problem, given what
you know about the data? **

Adaboost is a great candidate for this problem because the label of the
dataset contains binary classifications. As is often the case, this
dataset is generally ideal for decision trees. And decision trees are
paired with the Adaboost model when one level of a decision tree is
utilized.

** Sources: **

•
https://machinelearningmastery.com/boosting-and-adaboost-for-machine-learning/

• https://en.wikipedia.org/wiki/AdaBoost

• http://rob.schapire.net/papers/explaining-adaboost.pdf

•
https://stats.stackexchange.com/questions/20622/is-adaboost-less-or-more-prone-to-overfitting

•
https://www.analyticsvidhya.com/blog/2015/05/boosting-algorithms-simplified/

    \hypertarget{implementation---creating-a-training-and-predicting-pipeline}{%
\subsubsection{Implementation - Creating a Training and Predicting
Pipeline}\label{implementation---creating-a-training-and-predicting-pipeline}}

To properly evaluate the performance of each model you've chosen, it's
important that you create a training and predicting pipeline that allows
you to quickly and effectively train models using various sizes of
training data and perform predictions on the testing data. Your
implementation here will be used in the following section. In the code
block below, you will need to implement the following: - Import
\texttt{fbeta\_score} and \texttt{accuracy\_score} from
\href{http://scikit-learn.org/stable/modules/classes.html\#sklearn-metrics-metrics}{\texttt{sklearn.metrics}}.
- Fit the learner to the sampled training data and record the training
time. - Perform predictions on the test data \texttt{X\_test}, and also
on the first 300 training points \texttt{X\_train{[}:300{]}}. - Record
the total prediction time. - Calculate the accuracy score for both the
training subset and testing set. - Calculate the F-score for both the
training subset and testing set. - Make sure that you set the
\texttt{beta} parameter!

    \begin{Verbatim}[commandchars=\\\{\}]
{\color{incolor}In [{\color{incolor}37}]:} \PY{c+c1}{\PYZsh{} TODO: Import two metrics from sklearn \PYZhy{} fbeta\PYZus{}score and accuracy\PYZus{}score}
         \PY{k+kn}{from} \PY{n+nn}{sklearn}\PY{n+nn}{.}\PY{n+nn}{metrics} \PY{k}{import} \PY{n}{fbeta\PYZus{}score}\PY{p}{,} \PY{n}{accuracy\PYZus{}score}
         
         
         \PY{k}{def} \PY{n+nf}{train\PYZus{}predict}\PY{p}{(}\PY{n}{learner}\PY{p}{,} \PY{n}{sample\PYZus{}size}\PY{p}{,} \PY{n}{X\PYZus{}train}\PY{p}{,} \PY{n}{y\PYZus{}train}\PY{p}{,} \PY{n}{X\PYZus{}test}\PY{p}{,} \PY{n}{y\PYZus{}test}\PY{p}{)}\PY{p}{:} 
             \PY{l+s+sd}{\PYZsq{}\PYZsq{}\PYZsq{}}
         \PY{l+s+sd}{    inputs:}
         \PY{l+s+sd}{       \PYZhy{} learner: the learning algorithm to be trained and predicted on}
         \PY{l+s+sd}{       \PYZhy{} sample\PYZus{}size: the size of samples (number) to be drawn from training set}
         \PY{l+s+sd}{       \PYZhy{} X\PYZus{}train: features training set}
         \PY{l+s+sd}{       \PYZhy{} y\PYZus{}train: income training set}
         \PY{l+s+sd}{       \PYZhy{} X\PYZus{}test: features testing set}
         \PY{l+s+sd}{       \PYZhy{} y\PYZus{}test: income testing set}
         \PY{l+s+sd}{    \PYZsq{}\PYZsq{}\PYZsq{}}
             
             \PY{n}{results} \PY{o}{=} \PY{p}{\PYZob{}}\PY{p}{\PYZcb{}}
             
             \PY{c+c1}{\PYZsh{} TODO: Fit the learner to the training data using slicing with \PYZsq{}sample\PYZus{}size\PYZsq{} using .fit(training\PYZus{}features[:], training\PYZus{}labels[:])}
             \PY{n}{start} \PY{o}{=} \PY{n}{time}\PY{p}{(}\PY{p}{)} \PY{c+c1}{\PYZsh{} Get start time}
             \PY{n}{learner} \PY{o}{=} \PY{n}{learner}\PY{o}{.}\PY{n}{fit}\PY{p}{(}\PY{n}{X\PYZus{}train}\PY{p}{[}\PY{p}{:}\PY{n}{sample\PYZus{}size}\PY{p}{]}\PY{p}{,} \PY{n}{y\PYZus{}train}\PY{p}{[}\PY{p}{:}\PY{n}{sample\PYZus{}size}\PY{p}{]}\PY{p}{)}
             \PY{n}{end} \PY{o}{=} \PY{n}{time}\PY{p}{(}\PY{p}{)} \PY{c+c1}{\PYZsh{} Get end time}
             
             \PY{c+c1}{\PYZsh{} TODO: Calculate the training time}
             \PY{n}{results}\PY{p}{[}\PY{l+s+s1}{\PYZsq{}}\PY{l+s+s1}{train\PYZus{}time}\PY{l+s+s1}{\PYZsq{}}\PY{p}{]} \PY{o}{=} \PY{n}{end}\PY{o}{\PYZhy{}}\PY{n}{start}
                 
             \PY{c+c1}{\PYZsh{} TODO: Get the predictions on the test set(X\PYZus{}test),}
             \PY{c+c1}{\PYZsh{}       then get predictions on the first 300 training samples(X\PYZus{}train) using .predict()}
             \PY{n}{start} \PY{o}{=} \PY{n}{time}\PY{p}{(}\PY{p}{)} \PY{c+c1}{\PYZsh{} Get start time}
             \PY{n}{predictions\PYZus{}test} \PY{o}{=} \PY{n}{learner}\PY{o}{.}\PY{n}{predict}\PY{p}{(}\PY{n}{X\PYZus{}test}\PY{p}{)}
             \PY{n}{predictions\PYZus{}train} \PY{o}{=} \PY{n}{learner}\PY{o}{.}\PY{n}{predict}\PY{p}{(}\PY{n}{X\PYZus{}train}\PY{p}{[}\PY{p}{:}\PY{l+m+mi}{300}\PY{p}{]}\PY{p}{)}
             \PY{n}{end} \PY{o}{=} \PY{n}{time}\PY{p}{(}\PY{p}{)} \PY{c+c1}{\PYZsh{} Get end time}
             
             \PY{c+c1}{\PYZsh{} TODO: Calculate the total prediction time}
             \PY{n}{results}\PY{p}{[}\PY{l+s+s1}{\PYZsq{}}\PY{l+s+s1}{pred\PYZus{}time}\PY{l+s+s1}{\PYZsq{}}\PY{p}{]} \PY{o}{=} \PY{n}{end}\PY{o}{\PYZhy{}}\PY{n}{start}
                     
             \PY{c+c1}{\PYZsh{} TODO: Compute accuracy on the first 300 training samples which is y\PYZus{}train[:300]}
             \PY{n}{results}\PY{p}{[}\PY{l+s+s1}{\PYZsq{}}\PY{l+s+s1}{acc\PYZus{}train}\PY{l+s+s1}{\PYZsq{}}\PY{p}{]} \PY{o}{=} \PY{n}{accuracy\PYZus{}score}\PY{p}{(}\PY{n}{y\PYZus{}train}\PY{p}{[}\PY{p}{:}\PY{l+m+mi}{300}\PY{p}{]}\PY{p}{,} \PY{n}{predictions\PYZus{}train}\PY{p}{)}
                 
             \PY{c+c1}{\PYZsh{} TODO: Compute accuracy on test set using accuracy\PYZus{}score()}
             \PY{n}{results}\PY{p}{[}\PY{l+s+s1}{\PYZsq{}}\PY{l+s+s1}{acc\PYZus{}test}\PY{l+s+s1}{\PYZsq{}}\PY{p}{]} \PY{o}{=} \PY{n}{accuracy\PYZus{}score}\PY{p}{(}\PY{n}{y\PYZus{}test}\PY{p}{,} \PY{n}{predictions\PYZus{}test}\PY{p}{)}
             
             \PY{c+c1}{\PYZsh{} TODO: Compute F\PYZhy{}score on the the first 300 training samples using fbeta\PYZus{}score()}
             \PY{n}{results}\PY{p}{[}\PY{l+s+s1}{\PYZsq{}}\PY{l+s+s1}{f\PYZus{}train}\PY{l+s+s1}{\PYZsq{}}\PY{p}{]} \PY{o}{=} \PY{n}{fbeta\PYZus{}score}\PY{p}{(}\PY{n}{y\PYZus{}train}\PY{p}{[}\PY{p}{:}\PY{l+m+mi}{300}\PY{p}{]}\PY{p}{,} \PY{n}{predictions\PYZus{}train}\PY{p}{[}\PY{p}{:}\PY{l+m+mi}{300}\PY{p}{]}\PY{p}{,} \PY{n}{beta}\PY{o}{=}\PY{l+m+mf}{0.5}\PY{p}{)}
                 
             \PY{c+c1}{\PYZsh{} TODO: Compute F\PYZhy{}score on the test set which is y\PYZus{}test}
             \PY{n}{results}\PY{p}{[}\PY{l+s+s1}{\PYZsq{}}\PY{l+s+s1}{f\PYZus{}test}\PY{l+s+s1}{\PYZsq{}}\PY{p}{]} \PY{o}{=} \PY{n}{fbeta\PYZus{}score}\PY{p}{(}\PY{n}{y\PYZus{}test}\PY{p}{,} \PY{n}{predictions\PYZus{}test}\PY{p}{,} \PY{n}{beta}\PY{o}{=}\PY{l+m+mf}{0.5}\PY{p}{)}
                
             \PY{c+c1}{\PYZsh{} Success}
             \PY{n+nb}{print}\PY{p}{(}\PY{l+s+s2}{\PYZdq{}}\PY{l+s+si}{\PYZob{}\PYZcb{}}\PY{l+s+s2}{ trained on }\PY{l+s+si}{\PYZob{}\PYZcb{}}\PY{l+s+s2}{ samples.}\PY{l+s+s2}{\PYZdq{}}\PY{o}{.}\PY{n}{format}\PY{p}{(}\PY{n}{learner}\PY{o}{.}\PY{n+nv+vm}{\PYZus{}\PYZus{}class\PYZus{}\PYZus{}}\PY{o}{.}\PY{n+nv+vm}{\PYZus{}\PYZus{}name\PYZus{}\PYZus{}}\PY{p}{,} \PY{n}{sample\PYZus{}size}\PY{p}{)}\PY{p}{)}
                 
             \PY{c+c1}{\PYZsh{} Return the results}
             \PY{k}{return} \PY{n}{results}
\end{Verbatim}


    \hypertarget{implementation-initial-model-evaluation}{%
\subsubsection{Implementation: Initial Model
Evaluation}\label{implementation-initial-model-evaluation}}

In the code cell, you will need to implement the following: - Import the
three supervised learning models you've discussed in the previous
section. - Initialize the three models and store them in
\texttt{\textquotesingle{}clf\_A\textquotesingle{}},
\texttt{\textquotesingle{}clf\_B\textquotesingle{}}, and
\texttt{\textquotesingle{}clf\_C\textquotesingle{}}. - Use a
\texttt{\textquotesingle{}random\_state\textquotesingle{}} for each
model you use, if provided. - \textbf{Note:} Use the default settings
for each model --- you will tune one specific model in a later section.
- Calculate the number of records equal to 1\%, 10\%, and 100\% of the
training data. - Store those values in
\texttt{\textquotesingle{}samples\_1\textquotesingle{}},
\texttt{\textquotesingle{}samples\_10\textquotesingle{}}, and
\texttt{\textquotesingle{}samples\_100\textquotesingle{}} respectively.

\textbf{Note:} Depending on which algorithms you chose, the following
implementation may take some time to run!

    \begin{Verbatim}[commandchars=\\\{\}]
{\color{incolor}In [{\color{incolor}38}]:} \PY{c+c1}{\PYZsh{} TODO: Import the three supervised learning models from sklearn}
         \PY{k+kn}{from} \PY{n+nn}{sklearn}\PY{n+nn}{.}\PY{n+nn}{ensemble} \PY{k}{import} \PY{n}{RandomForestClassifier}
         \PY{k+kn}{from} \PY{n+nn}{sklearn} \PY{k}{import} \PY{n}{svm}
         \PY{k+kn}{from} \PY{n+nn}{sklearn}\PY{n+nn}{.}\PY{n+nn}{ensemble} \PY{k}{import} \PY{n}{AdaBoostClassifier}
         
         \PY{c+c1}{\PYZsh{} TODO: Initialize the three models}
         \PY{n}{clf\PYZus{}A} \PY{o}{=} \PY{n}{RandomForestClassifier}\PY{p}{(}\PY{n}{random\PYZus{}state}\PY{o}{=}\PY{l+m+mi}{6890}\PY{p}{)}
         \PY{n}{clf\PYZus{}B} \PY{o}{=} \PY{n}{svm}\PY{o}{.}\PY{n}{SVC}\PY{p}{(}\PY{n}{random\PYZus{}state}\PY{o}{=}\PY{l+m+mi}{740}\PY{p}{)}
         \PY{n}{clf\PYZus{}C} \PY{o}{=} \PY{n}{AdaBoostClassifier}\PY{p}{(}\PY{n}{random\PYZus{}state}\PY{o}{=}\PY{l+m+mi}{85}\PY{p}{)}
         
         \PY{c+c1}{\PYZsh{} TODO: Calculate the number of samples for 1\PYZpc{}, 10\PYZpc{}, and 100\PYZpc{} of the training data}
         \PY{c+c1}{\PYZsh{} HINT: samples\PYZus{}100 is the entire training set i.e. len(y\PYZus{}train)}
         \PY{c+c1}{\PYZsh{} HINT: samples\PYZus{}10 is 10\PYZpc{} of samples\PYZus{}100 (ensure to set the count of the values to be `int` and not `float`)}
         \PY{c+c1}{\PYZsh{} HINT: samples\PYZus{}1 is 1\PYZpc{} of samples\PYZus{}100 (ensure to set the count of the values to be `int` and not `float`)}
         \PY{c+c1}{\PYZsh{}samples\PYZus{}100 = int(len(y\PYZus{}train))}
         \PY{c+c1}{\PYZsh{}samples\PYZus{}10 = int(len(y\PYZus{}train) * 0.1)}
         \PY{c+c1}{\PYZsh{}samples\PYZus{}1 = int(len(y\PYZus{}train) * 0.01)}
         \PY{n}{samples\PYZus{}100} \PY{o}{=} \PY{n+nb}{len}\PY{p}{(}\PY{n}{y\PYZus{}train}\PY{p}{)}
         \PY{n}{samples\PYZus{}10} \PY{o}{=} \PY{n+nb}{int}\PY{p}{(}\PY{n}{samples\PYZus{}100} \PY{o}{*} \PY{l+m+mf}{0.1}\PY{p}{)}
         \PY{n}{samples\PYZus{}1} \PY{o}{=} \PY{n+nb}{int}\PY{p}{(}\PY{n}{samples\PYZus{}100} \PY{o}{*} \PY{l+m+mf}{0.01}\PY{p}{)}
         
         \PY{c+c1}{\PYZsh{} Collect results on the learners}
         \PY{n}{results} \PY{o}{=} \PY{p}{\PYZob{}}\PY{p}{\PYZcb{}}
         \PY{k}{for} \PY{n}{clf} \PY{o+ow}{in} \PY{p}{[}\PY{n}{clf\PYZus{}A}\PY{p}{,} \PY{n}{clf\PYZus{}B}\PY{p}{,} \PY{n}{clf\PYZus{}C}\PY{p}{]}\PY{p}{:}
             \PY{n}{clf\PYZus{}name} \PY{o}{=} \PY{n}{clf}\PY{o}{.}\PY{n+nv+vm}{\PYZus{}\PYZus{}class\PYZus{}\PYZus{}}\PY{o}{.}\PY{n+nv+vm}{\PYZus{}\PYZus{}name\PYZus{}\PYZus{}}
             \PY{n}{results}\PY{p}{[}\PY{n}{clf\PYZus{}name}\PY{p}{]} \PY{o}{=} \PY{p}{\PYZob{}}\PY{p}{\PYZcb{}}
             \PY{k}{for} \PY{n}{i}\PY{p}{,} \PY{n}{samples} \PY{o+ow}{in} \PY{n+nb}{enumerate}\PY{p}{(}\PY{p}{[}\PY{n}{samples\PYZus{}1}\PY{p}{,} \PY{n}{samples\PYZus{}10}\PY{p}{,} \PY{n}{samples\PYZus{}100}\PY{p}{]}\PY{p}{)}\PY{p}{:}
                 \PY{n}{results}\PY{p}{[}\PY{n}{clf\PYZus{}name}\PY{p}{]}\PY{p}{[}\PY{n}{i}\PY{p}{]} \PY{o}{=} \PYZbs{}
                 \PY{n}{train\PYZus{}predict}\PY{p}{(}\PY{n}{clf}\PY{p}{,} \PY{n}{samples}\PY{p}{,} \PY{n}{X\PYZus{}train}\PY{p}{,} \PY{n}{y\PYZus{}train}\PY{p}{,} \PY{n}{X\PYZus{}test}\PY{p}{,} \PY{n}{y\PYZus{}test}\PY{p}{)}
         
         \PY{c+c1}{\PYZsh{} Run metrics visualization for the three supervised learning models chosen}
         \PY{n}{vs}\PY{o}{.}\PY{n}{evaluate}\PY{p}{(}\PY{n}{results}\PY{p}{,} \PY{n}{accuracy}\PY{p}{,} \PY{n}{fscore}\PY{p}{)}
\end{Verbatim}


    \begin{Verbatim}[commandchars=\\\{\}]
RandomForestClassifier trained on 361 samples.
RandomForestClassifier trained on 3617 samples.
RandomForestClassifier trained on 36177 samples.

    \end{Verbatim}

    \begin{Verbatim}[commandchars=\\\{\}]
/opt/conda/lib/python3.6/site-packages/sklearn/metrics/classification.py:1135: UndefinedMetricWarning: F-score is ill-defined and being set to 0.0 due to no predicted samples.
  'precision', 'predicted', average, warn\_for)

    \end{Verbatim}

    \begin{Verbatim}[commandchars=\\\{\}]
SVC trained on 361 samples.
SVC trained on 3617 samples.
SVC trained on 36177 samples.
AdaBoostClassifier trained on 361 samples.
AdaBoostClassifier trained on 3617 samples.
AdaBoostClassifier trained on 36177 samples.

    \end{Verbatim}

    \begin{center}
    \adjustimage{max size={0.9\linewidth}{0.9\paperheight}}{output_29_3.png}
    \end{center}
    { \hspace*{\fill} \\}
    
    \begin{center}\rule{0.5\linewidth}{\linethickness}\end{center}

\hypertarget{improving-results}{%
\subsection{Improving Results}\label{improving-results}}

In this final section, you will choose from the three supervised
learning models the \emph{best} model to use on the student data. You
will then perform a grid search optimization for the model over the
entire training set (\texttt{X\_train} and \texttt{y\_train}) by tuning
at least one parameter to improve upon the untuned model's F-score.

    \hypertarget{question-3---choosing-the-best-model}{%
\subsubsection{Question 3 - Choosing the Best
Model}\label{question-3---choosing-the-best-model}}

\begin{itemize}
\tightlist
\item
  Based on the evaluation you performed earlier, in one to two
  paragraphs, explain to \emph{CharityML} which of the three models you
  believe to be most appropriate for the task of identifying individuals
  that make more than \$50,000.
\end{itemize}

** HINT: ** Look at the graph at the bottom left from the cell above(the
visualization created by
\texttt{vs.evaluate(results,\ accuracy,\ fscore)}) and check the F score
for the testing set when 100\% of the training set is used. Which model
has the highest score? Your answer should include discussion of the: *
metrics - F score on the testing when 100\% of the training data is
used, * prediction/training time * the algorithm's suitability for the
data.

    \textbf{Answer: }

Based on the results, the AdaBoost model is most appropriate for the
task of identifying individuals that make more than \$50,000. I reached
this conclusion based on the following reasons: • Out of the three
models, the AdaBoost model has the highest F-score on the testing set
when 100\% of the training data is used. • The AdaBoost model has a low
prediction/training time, especially when compared to the SVC model. •
AdaBoost is highly suitable for the data since the label is comprised of
two binary classifications.

    \hypertarget{question-4---describing-the-model-in-laymans-terms}{%
\subsubsection{Question 4 - Describing the Model in Layman's
Terms}\label{question-4---describing-the-model-in-laymans-terms}}

\begin{itemize}
\tightlist
\item
  In one to two paragraphs, explain to \emph{CharityML}, in layman's
  terms, how the final model chosen is supposed to work. Be sure that
  you are describing the major qualities of the model, such as how the
  model is trained and how the model makes a prediction. Avoid using
  advanced mathematical jargon, such as describing equations.
\end{itemize}

** HINT: **

When explaining your model, if using external resources please include
all citations.

    \textbf{Answer: }

AdaBoost involves sequentially constructing a strong model by combining
multiple weak models. It is often used in conjunction with many other
types of learning algorithms to improve results. Also, AdaBoost is
adaptive because as it trains, subsequent models attempt to correct the
errors of prior models. Weak models are sequentially added and trained
within this adaptive process. The process continues until a pre-set
number of weak learners have been created and no further improvement can
be made based on the input criteria. Ultimately, predictions are made by
calculating the weighted average of the weak classifiers.

AdaBoost is best used to boost the performance of decision trees on
binary classification problems.

Source:
https://machinelearningmastery.com/boosting-and-adaboost-for-machine-learning/

    \hypertarget{implementation-model-tuning}{%
\subsubsection{Implementation: Model
Tuning}\label{implementation-model-tuning}}

Fine tune the chosen model. Use grid search (\texttt{GridSearchCV}) with
at least one important parameter tuned with at least 3 different values.
You will need to use the entire training set for this. In the code cell
below, you will need to implement the following: - Import
\href{http://scikit-learn.org/0.17/modules/generated/sklearn.grid_search.GridSearchCV.html}{\texttt{sklearn.grid\_search.GridSearchCV}}
and
\href{http://scikit-learn.org/stable/modules/generated/sklearn.metrics.make_scorer.html}{\texttt{sklearn.metrics.make\_scorer}}.
- Initialize the classifier you've chosen and store it in \texttt{clf}.
- Set a \texttt{random\_state} if one is available to the same state you
set before. - Create a dictionary of parameters you wish to tune for the
chosen model. - Example:
\texttt{parameters\ =\ \{\textquotesingle{}parameter\textquotesingle{}\ :\ {[}list\ of\ values{]}\}}.
- \textbf{Note:} Avoid tuning the \texttt{max\_features} parameter of
your learner if that parameter is available! - Use \texttt{make\_scorer}
to create an \texttt{fbeta\_score} scoring object (with
\(\beta = 0.5\)). - Perform grid search on the classifier \texttt{clf}
using the \texttt{\textquotesingle{}scorer\textquotesingle{}}, and store
it in \texttt{grid\_obj}. - Fit the grid search object to the training
data (\texttt{X\_train}, \texttt{y\_train}), and store it in
\texttt{grid\_fit}.

\textbf{Note:} Depending on the algorithm chosen and the parameter list,
the following implementation may take some time to run!

    \begin{Verbatim}[commandchars=\\\{\}]
{\color{incolor}In [{\color{incolor}39}]:} \PY{c+c1}{\PYZsh{} TODO: Import \PYZsq{}GridSearchCV\PYZsq{}, \PYZsq{}make\PYZus{}scorer\PYZsq{}, and any other necessary libraries}
         \PY{k+kn}{from} \PY{n+nn}{sklearn} \PY{k}{import} \PY{n}{grid\PYZus{}search}
         \PY{k+kn}{from} \PY{n+nn}{sklearn}\PY{n+nn}{.}\PY{n+nn}{metrics} \PY{k}{import} \PY{n}{make\PYZus{}scorer}
         
         \PY{c+c1}{\PYZsh{} TODO: Initialize the classifier}
         \PY{n}{clf} \PY{o}{=} \PY{n}{RandomForestClassifier}\PY{p}{(}\PY{n}{random\PYZus{}state}\PY{o}{=}\PY{l+m+mi}{6890}\PY{p}{)}
         
         \PY{c+c1}{\PYZsh{} TODO: Create the parameters list you wish to tune, using a dictionary if needed.}
         \PY{c+c1}{\PYZsh{} HINT: parameters = \PYZob{}\PYZsq{}parameter\PYZus{}1\PYZsq{}: [value1, value2], \PYZsq{}parameter\PYZus{}2\PYZsq{}: [value1, value2]\PYZcb{}}
         \PY{n}{parameters} \PY{o}{=} \PY{p}{\PYZob{}}\PY{l+s+s1}{\PYZsq{}}\PY{l+s+s1}{bootstrap}\PY{l+s+s1}{\PYZsq{}}\PY{p}{:} \PY{p}{[}\PY{k+kc}{False}\PY{p}{]}\PY{p}{,} \PY{l+s+s1}{\PYZsq{}}\PY{l+s+s1}{criterion}\PY{l+s+s1}{\PYZsq{}}\PY{p}{:} \PY{p}{[}\PY{l+s+s1}{\PYZsq{}}\PY{l+s+s1}{gini}\PY{l+s+s1}{\PYZsq{}}\PY{p}{]}\PY{p}{,} \PY{l+s+s1}{\PYZsq{}}\PY{l+s+s1}{max\PYZus{}depth}\PY{l+s+s1}{\PYZsq{}}\PY{p}{:} \PY{p}{[}\PY{l+m+mi}{20}\PY{p}{,} \PY{l+m+mi}{30}\PY{p}{,} \PY{l+m+mi}{40}\PY{p}{]}\PY{p}{,} \PY{l+s+s1}{\PYZsq{}}\PY{l+s+s1}{max\PYZus{}features}\PY{l+s+s1}{\PYZsq{}}\PY{p}{:} \PY{p}{[}\PY{l+m+mi}{2}\PY{p}{,} \PY{l+m+mi}{3}\PY{p}{]}\PY{p}{,} \PY{l+s+s1}{\PYZsq{}}\PY{l+s+s1}{min\PYZus{}samples\PYZus{}leaf}\PY{l+s+s1}{\PYZsq{}}\PY{p}{:} \PY{p}{[}\PY{l+m+mi}{1}\PY{p}{]}\PY{p}{,} \PY{l+s+s1}{\PYZsq{}}\PY{l+s+s1}{min\PYZus{}samples\PYZus{}split}\PY{l+s+s1}{\PYZsq{}}\PY{p}{:} \PY{p}{[}\PY{l+m+mi}{2}\PY{p}{]}\PY{p}{\PYZcb{}}
         
         \PY{c+c1}{\PYZsh{} TODO: Make an fbeta\PYZus{}score scoring object using make\PYZus{}scorer()}
         \PY{n}{scorer} \PY{o}{=} \PY{n}{make\PYZus{}scorer}\PY{p}{(}\PY{n}{fbeta\PYZus{}score}\PY{p}{,} \PY{n}{beta} \PY{o}{=} \PY{l+m+mf}{0.5}\PY{p}{)}
         
         \PY{c+c1}{\PYZsh{} TODO: Perform grid search on the classifier using \PYZsq{}scorer\PYZsq{} as the scoring method using GridSearchCV()}
         \PY{n}{grid\PYZus{}obj} \PY{o}{=} \PY{n}{grid\PYZus{}search}\PY{o}{.}\PY{n}{GridSearchCV}\PY{p}{(}\PY{n}{clf}\PY{p}{,} \PY{n}{parameters}\PY{p}{,} \PY{n}{scoring}\PY{o}{=}\PY{n}{scorer}\PY{p}{)}
         
         \PY{c+c1}{\PYZsh{} TODO: Fit the grid search object to the training data and find the optimal parameters using fit()}
         \PY{n}{grid\PYZus{}fit} \PY{o}{=} \PY{n}{grid\PYZus{}obj}\PY{o}{.}\PY{n}{fit}\PY{p}{(}\PY{n}{X\PYZus{}train}\PY{p}{,} \PY{n}{y\PYZus{}train}\PY{o}{.}\PY{n}{values}\PY{o}{.}\PY{n}{ravel}\PY{p}{(}\PY{p}{)}\PY{p}{)}
         
         \PY{c+c1}{\PYZsh{} Get the estimator}
         \PY{n}{best\PYZus{}clf} \PY{o}{=} \PY{n}{grid\PYZus{}fit}\PY{o}{.}\PY{n}{best\PYZus{}estimator\PYZus{}}
         
         \PY{c+c1}{\PYZsh{} Make predictions using the unoptimized and model}
         \PY{n}{predictions} \PY{o}{=} \PY{p}{(}\PY{n}{clf}\PY{o}{.}\PY{n}{fit}\PY{p}{(}\PY{n}{X\PYZus{}train}\PY{p}{,} \PY{n}{y\PYZus{}train}\PY{p}{)}\PY{p}{)}\PY{o}{.}\PY{n}{predict}\PY{p}{(}\PY{n}{X\PYZus{}test}\PY{p}{)}
         \PY{n}{best\PYZus{}predictions} \PY{o}{=} \PY{n}{best\PYZus{}clf}\PY{o}{.}\PY{n}{predict}\PY{p}{(}\PY{n}{X\PYZus{}test}\PY{p}{)}
         
         \PY{c+c1}{\PYZsh{} Report the before\PYZhy{}and\PYZhy{}afterscores}
         \PY{n+nb}{print}\PY{p}{(}\PY{l+s+s2}{\PYZdq{}}\PY{l+s+s2}{Unoptimized model}\PY{l+s+se}{\PYZbs{}n}\PY{l+s+s2}{\PYZhy{}\PYZhy{}\PYZhy{}\PYZhy{}\PYZhy{}\PYZhy{}}\PY{l+s+s2}{\PYZdq{}}\PY{p}{)}
         \PY{n+nb}{print}\PY{p}{(}\PY{l+s+s2}{\PYZdq{}}\PY{l+s+s2}{Accuracy score on testing data: }\PY{l+s+si}{\PYZob{}:.4f\PYZcb{}}\PY{l+s+s2}{\PYZdq{}}\PY{o}{.}\PY{n}{format}\PY{p}{(}\PY{n}{accuracy\PYZus{}score}\PY{p}{(}\PY{n}{y\PYZus{}test}\PY{p}{,} \PY{n}{predictions}\PY{p}{)}\PY{p}{)}\PY{p}{)}
         \PY{n+nb}{print}\PY{p}{(}\PY{l+s+s2}{\PYZdq{}}\PY{l+s+s2}{F\PYZhy{}score on testing data: }\PY{l+s+si}{\PYZob{}:.4f\PYZcb{}}\PY{l+s+s2}{\PYZdq{}}\PY{o}{.}\PY{n}{format}\PY{p}{(}\PY{n}{fbeta\PYZus{}score}\PY{p}{(}\PY{n}{y\PYZus{}test}\PY{p}{,} \PY{n}{predictions}\PY{p}{,} \PY{n}{beta} \PY{o}{=} \PY{l+m+mf}{0.5}\PY{p}{)}\PY{p}{)}\PY{p}{)}
         \PY{n+nb}{print}\PY{p}{(}\PY{l+s+s2}{\PYZdq{}}\PY{l+s+se}{\PYZbs{}n}\PY{l+s+s2}{Optimized Model}\PY{l+s+se}{\PYZbs{}n}\PY{l+s+s2}{\PYZhy{}\PYZhy{}\PYZhy{}\PYZhy{}\PYZhy{}\PYZhy{}}\PY{l+s+s2}{\PYZdq{}}\PY{p}{)}
         \PY{n+nb}{print}\PY{p}{(}\PY{l+s+s2}{\PYZdq{}}\PY{l+s+s2}{Final accuracy score on the testing data: }\PY{l+s+si}{\PYZob{}:.4f\PYZcb{}}\PY{l+s+s2}{\PYZdq{}}\PY{o}{.}\PY{n}{format}\PY{p}{(}\PY{n}{accuracy\PYZus{}score}\PY{p}{(}\PY{n}{y\PYZus{}test}\PY{p}{,} \PY{n}{best\PYZus{}predictions}\PY{p}{)}\PY{p}{)}\PY{p}{)}
         \PY{n+nb}{print}\PY{p}{(}\PY{l+s+s2}{\PYZdq{}}\PY{l+s+s2}{Final F\PYZhy{}score on the testing data: }\PY{l+s+si}{\PYZob{}:.4f\PYZcb{}}\PY{l+s+s2}{\PYZdq{}}\PY{o}{.}\PY{n}{format}\PY{p}{(}\PY{n}{fbeta\PYZus{}score}\PY{p}{(}\PY{n}{y\PYZus{}test}\PY{p}{,} \PY{n}{best\PYZus{}predictions}\PY{p}{,} \PY{n}{beta} \PY{o}{=} \PY{l+m+mf}{0.5}\PY{p}{)}\PY{p}{)}\PY{p}{)}
\end{Verbatim}


    \begin{Verbatim}[commandchars=\\\{\}]
Unoptimized model
------
Accuracy score on testing data: 0.8378
F-score on testing data: 0.6721

Optimized Model
------
Final accuracy score on the testing data: 0.8495
Final F-score on the testing data: 0.7130

    \end{Verbatim}

    \hypertarget{question-5---final-model-evaluation}{%
\subsubsection{Question 5 - Final Model
Evaluation}\label{question-5---final-model-evaluation}}

\begin{itemize}
\tightlist
\item
  What is your optimized model's accuracy and F-score on the testing
  data?
\item
  Are these scores better or worse than the unoptimized model?
\item
  How do the results from your optimized model compare to the naive
  predictor benchmarks you found earlier in \textbf{Question 1}?\_
\end{itemize}

\textbf{Note:} Fill in the table below with your results, and then
provide discussion in the \textbf{Answer} box.

    \hypertarget{results}{%
\paragraph{Results:}\label{results}}

\begin{longtable}[]{@{}ccc@{}}
\toprule
Metric & Unoptimized Model & Optimized Model\tabularnewline
\midrule
\endhead
Accuracy Score & 84\% & 85\%\tabularnewline
F-score & 67.2\% & 71.3\%\tabularnewline
\bottomrule
\end{longtable}

    \textbf{Answer: }

** What is your optimized model's accuracy and F-score on the testing
data? **

The results were typed above in the table.

** Are these scores better or worse than the unoptimized model? **

The scores for the optimized model are better than the scores for the
unoptimized model. Also, the results from the optimized model are far
superior to the naïve predictor benchmarks in Question 1.

** How do the results from your optimized model compare to the naive
predictor benchmarks you found earlier in Question 1? **

The optimized model's accuracy is 85\% while the naïve predictor's
accuracy is 4.8\%. The optimized model's F-score is 71.3\% while the
naïve predictor's F-score is 9.1\%.

    \begin{center}\rule{0.5\linewidth}{\linethickness}\end{center}

\hypertarget{feature-importance}{%
\subsection{Feature Importance}\label{feature-importance}}

An important task when performing supervised learning on a dataset like
the census data we study here is determining which features provide the
most predictive power. By focusing on the relationship between only a
few crucial features and the target label we simplify our understanding
of the phenomenon, which is most always a useful thing to do. In the
case of this project, that means we wish to identify a small number of
features that most strongly predict whether an individual makes at most
or more than \$50,000.

Choose a scikit-learn classifier (e.g., adaboost, random forests) that
has a \texttt{feature\_importance\_} attribute, which is a function that
ranks the importance of features according to the chosen classifier. In
the next python cell fit this classifier to training set and use this
attribute to determine the top 5 most important features for the census
dataset.

    \hypertarget{question-6---feature-relevance-observation}{%
\subsubsection{Question 6 - Feature Relevance
Observation}\label{question-6---feature-relevance-observation}}

When \textbf{Exploring the Data}, it was shown there are thirteen
available features for each individual on record in the census data. Of
these thirteen records, which five features do you believe to be most
important for prediction, and in what order would you rank them and why?

    \textbf{Answer:}

I think it would be reasonable to assume that the following five
features are most important for prediction in this order: 1. **
occupation\_ Prof-specialty ** -- Individuals in fields of professional
specialty (technical, medical, law, etc.) tend to have higher wages than
people in other fields. 2. ** education\_level\_ Doctorate ** -- People
with higher levels of education tend to have higher wages than those
with lower levels of education. 3. ** age ** -- People with more years
of experience in the workforce tend to have higher wages. 4. **
relationship\_Husband ** -- Husbands are likely to have higher wages
than other types of people, maybe because of the general mindset that
marriage and leading a family encourages; there are also favorable
political considerations involved. 5. ** sex \_Male ** -- Unfortunately,
males may have the upper hand in this area; trends on salary data seem
to indicate this.

    \hypertarget{implementation---extracting-feature-importance}{%
\subsubsection{Implementation - Extracting Feature
Importance}\label{implementation---extracting-feature-importance}}

Choose a \texttt{scikit-learn} supervised learning algorithm that has a
\texttt{feature\_importance\_} attribute availble for it. This attribute
is a function that ranks the importance of each feature when making
predictions based on the chosen algorithm.

In the code cell below, you will need to implement the following: -
Import a supervised learning model from sklearn if it is different from
the three used earlier. - Train the supervised model on the entire
training set. - Extract the feature importances using
\texttt{\textquotesingle{}.feature\_importances\_\textquotesingle{}}.

    \begin{Verbatim}[commandchars=\\\{\}]
{\color{incolor}In [{\color{incolor}40}]:} \PY{c+c1}{\PYZsh{} TODO: Import a supervised learning model that has \PYZsq{}feature\PYZus{}importances\PYZus{}\PYZsq{}}
         \PY{k+kn}{from} \PY{n+nn}{sklearn}\PY{n+nn}{.}\PY{n+nn}{ensemble} \PY{k}{import} \PY{n}{RandomForestClassifier}
         
         \PY{c+c1}{\PYZsh{} TODO: Train the supervised model on the training set using .fit(X\PYZus{}train, y\PYZus{}train)}
         \PY{n}{model} \PY{o}{=} \PY{n}{RandomForestClassifier}\PY{p}{(}\PY{n}{random\PYZus{}state}\PY{o}{=}\PY{l+m+mi}{7}\PY{p}{)}\PY{o}{.}\PY{n}{fit}\PY{p}{(}\PY{n}{X\PYZus{}train}\PY{p}{,} \PY{n}{y\PYZus{}train}\PY{p}{)}
         
         \PY{c+c1}{\PYZsh{} TODO: Extract the feature importances using .feature\PYZus{}importances\PYZus{} }
         \PY{n}{importances} \PY{o}{=} \PY{n}{model}\PY{o}{.}\PY{n}{feature\PYZus{}importances\PYZus{}}
         
         \PY{c+c1}{\PYZsh{} Plot}
         \PY{n}{vs}\PY{o}{.}\PY{n}{feature\PYZus{}plot}\PY{p}{(}\PY{n}{importances}\PY{p}{,} \PY{n}{X\PYZus{}train}\PY{p}{,} \PY{n}{y\PYZus{}train}\PY{p}{)}
\end{Verbatim}


    \begin{center}
    \adjustimage{max size={0.9\linewidth}{0.9\paperheight}}{output_44_0.png}
    \end{center}
    { \hspace*{\fill} \\}
    
    \hypertarget{question-7---extracting-feature-importance}{%
\subsubsection{Question 7 - Extracting Feature
Importance}\label{question-7---extracting-feature-importance}}

Observe the visualization created above which displays the five most
relevant features for predicting if an individual makes at most or above
\$50,000.\\
* How do these five features compare to the five features you discussed
in \textbf{Question 6}? * If you were close to the same answer, how does
this visualization confirm your thoughts? * If you were not close, why
do you think these features are more relevant?

    \textbf{Answer:}

** How do these five features compare to the five features you discussed
in Question 6? **

I chose two of the five correct features. I did not choose
hours-per-week, capital-gain, and marital-status\_Married. I did not
believe hours-per-week and marital-status made such a difference. I
should have considered capital-gain more closely because this feature
can generally indicate the ownership of property.

** If you were close to the same answer, how does this visualization
confirm your thoughts? **

I was certain that age would have a high correlation with income because
numerous charts and analysis on economic websites support this
conclusion. The feature ``relationship\_Husband'' seemed obvious because
this type of person usually has the highest wealth in every related
observation I've seen.

** If you were not close, why do you think these features are more
relevant? **

The feature ``hours-per-week'' probably ranked high because workers with
hourly wages in some professions can make significant amounts of money
with overtime pay.

The feature ``capital-gain'' likely ranks high since it indicates that
an individual recently sold property and that individual likely has
other assets to manage. It seems unlikely that a person would sell all
of his or her property.

The feature ``marital-status\_Married'' is highly predictive, probably
because marriage encourages a mindset and societal perception conductive
to setting career goals, financial planning, leadership, and personal
branding.

    \hypertarget{feature-selection}{%
\subsubsection{Feature Selection}\label{feature-selection}}

How does a model perform if we only use a subset of all the available
features in the data? With less features required to train, the
expectation is that training and prediction time is much lower --- at
the cost of performance metrics. From the visualization above, we see
that the top five most important features contribute more than half of
the importance of \textbf{all} features present in the data. This hints
that we can attempt to \emph{reduce the feature space} and simplify the
information required for the model to learn. The code cell below will
use the same optimized model you found earlier, and train it on the same
training set \emph{with only the top five important features}.

    \begin{Verbatim}[commandchars=\\\{\}]
{\color{incolor}In [{\color{incolor}41}]:} \PY{c+c1}{\PYZsh{} Import functionality for cloning a model}
         \PY{k+kn}{from} \PY{n+nn}{sklearn}\PY{n+nn}{.}\PY{n+nn}{base} \PY{k}{import} \PY{n}{clone}
         
         \PY{c+c1}{\PYZsh{} Reduce the feature space}
         \PY{n}{X\PYZus{}train\PYZus{}reduced} \PY{o}{=} \PY{n}{X\PYZus{}train}\PY{p}{[}\PY{n}{X\PYZus{}train}\PY{o}{.}\PY{n}{columns}\PY{o}{.}\PY{n}{values}\PY{p}{[}\PY{p}{(}\PY{n}{np}\PY{o}{.}\PY{n}{argsort}\PY{p}{(}\PY{n}{importances}\PY{p}{)}\PY{p}{[}\PY{p}{:}\PY{p}{:}\PY{o}{\PYZhy{}}\PY{l+m+mi}{1}\PY{p}{]}\PY{p}{)}\PY{p}{[}\PY{p}{:}\PY{l+m+mi}{5}\PY{p}{]}\PY{p}{]}\PY{p}{]}
         \PY{n}{X\PYZus{}test\PYZus{}reduced} \PY{o}{=} \PY{n}{X\PYZus{}test}\PY{p}{[}\PY{n}{X\PYZus{}test}\PY{o}{.}\PY{n}{columns}\PY{o}{.}\PY{n}{values}\PY{p}{[}\PY{p}{(}\PY{n}{np}\PY{o}{.}\PY{n}{argsort}\PY{p}{(}\PY{n}{importances}\PY{p}{)}\PY{p}{[}\PY{p}{:}\PY{p}{:}\PY{o}{\PYZhy{}}\PY{l+m+mi}{1}\PY{p}{]}\PY{p}{)}\PY{p}{[}\PY{p}{:}\PY{l+m+mi}{5}\PY{p}{]}\PY{p}{]}\PY{p}{]}
         
         \PY{c+c1}{\PYZsh{} Train on the \PYZdq{}best\PYZdq{} model found from grid search earlier}
         \PY{n}{clf} \PY{o}{=} \PY{p}{(}\PY{n}{clone}\PY{p}{(}\PY{n}{best\PYZus{}clf}\PY{p}{)}\PY{p}{)}\PY{o}{.}\PY{n}{fit}\PY{p}{(}\PY{n}{X\PYZus{}train\PYZus{}reduced}\PY{p}{,} \PY{n}{y\PYZus{}train}\PY{p}{)}
         
         \PY{c+c1}{\PYZsh{} Make new predictions}
         \PY{n}{reduced\PYZus{}predictions} \PY{o}{=} \PY{n}{clf}\PY{o}{.}\PY{n}{predict}\PY{p}{(}\PY{n}{X\PYZus{}test\PYZus{}reduced}\PY{p}{)}
         
         \PY{c+c1}{\PYZsh{} Report scores from the final model using both versions of data}
         \PY{n+nb}{print}\PY{p}{(}\PY{l+s+s2}{\PYZdq{}}\PY{l+s+s2}{Final Model trained on full data}\PY{l+s+se}{\PYZbs{}n}\PY{l+s+s2}{\PYZhy{}\PYZhy{}\PYZhy{}\PYZhy{}\PYZhy{}\PYZhy{}}\PY{l+s+s2}{\PYZdq{}}\PY{p}{)}
         \PY{n+nb}{print}\PY{p}{(}\PY{l+s+s2}{\PYZdq{}}\PY{l+s+s2}{Accuracy on testing data: }\PY{l+s+si}{\PYZob{}:.4f\PYZcb{}}\PY{l+s+s2}{\PYZdq{}}\PY{o}{.}\PY{n}{format}\PY{p}{(}\PY{n}{accuracy\PYZus{}score}\PY{p}{(}\PY{n}{y\PYZus{}test}\PY{p}{,} \PY{n}{best\PYZus{}predictions}\PY{p}{)}\PY{p}{)}\PY{p}{)}
         \PY{n+nb}{print}\PY{p}{(}\PY{l+s+s2}{\PYZdq{}}\PY{l+s+s2}{F\PYZhy{}score on testing data: }\PY{l+s+si}{\PYZob{}:.4f\PYZcb{}}\PY{l+s+s2}{\PYZdq{}}\PY{o}{.}\PY{n}{format}\PY{p}{(}\PY{n}{fbeta\PYZus{}score}\PY{p}{(}\PY{n}{y\PYZus{}test}\PY{p}{,} \PY{n}{best\PYZus{}predictions}\PY{p}{,} \PY{n}{beta} \PY{o}{=} \PY{l+m+mf}{0.5}\PY{p}{)}\PY{p}{)}\PY{p}{)}
         \PY{n+nb}{print}\PY{p}{(}\PY{l+s+s2}{\PYZdq{}}\PY{l+s+se}{\PYZbs{}n}\PY{l+s+s2}{Final Model trained on reduced data}\PY{l+s+se}{\PYZbs{}n}\PY{l+s+s2}{\PYZhy{}\PYZhy{}\PYZhy{}\PYZhy{}\PYZhy{}\PYZhy{}}\PY{l+s+s2}{\PYZdq{}}\PY{p}{)}
         \PY{n+nb}{print}\PY{p}{(}\PY{l+s+s2}{\PYZdq{}}\PY{l+s+s2}{Accuracy on testing data: }\PY{l+s+si}{\PYZob{}:.4f\PYZcb{}}\PY{l+s+s2}{\PYZdq{}}\PY{o}{.}\PY{n}{format}\PY{p}{(}\PY{n}{accuracy\PYZus{}score}\PY{p}{(}\PY{n}{y\PYZus{}test}\PY{p}{,} \PY{n}{reduced\PYZus{}predictions}\PY{p}{)}\PY{p}{)}\PY{p}{)}
         \PY{n+nb}{print}\PY{p}{(}\PY{l+s+s2}{\PYZdq{}}\PY{l+s+s2}{F\PYZhy{}score on testing data: }\PY{l+s+si}{\PYZob{}:.4f\PYZcb{}}\PY{l+s+s2}{\PYZdq{}}\PY{o}{.}\PY{n}{format}\PY{p}{(}\PY{n}{fbeta\PYZus{}score}\PY{p}{(}\PY{n}{y\PYZus{}test}\PY{p}{,} \PY{n}{reduced\PYZus{}predictions}\PY{p}{,} \PY{n}{beta} \PY{o}{=} \PY{l+m+mf}{0.5}\PY{p}{)}\PY{p}{)}\PY{p}{)}
\end{Verbatim}


    \begin{Verbatim}[commandchars=\\\{\}]
Final Model trained on full data
------
Accuracy on testing data: 0.8495
F-score on testing data: 0.7130

Final Model trained on reduced data
------
Accuracy on testing data: 0.8125
F-score on testing data: 0.6120

    \end{Verbatim}

    \hypertarget{question-8---effects-of-feature-selection}{%
\subsubsection{Question 8 - Effects of Feature
Selection}\label{question-8---effects-of-feature-selection}}

\begin{itemize}
\tightlist
\item
  How does the final model's F-score and accuracy score on the reduced
  data using only five features compare to those same scores when all
  features are used?
\item
  If training time was a factor, would you consider using the reduced
  data as your training set?
\end{itemize}

    \textbf{Answer:}

** How does the final model's F-score and accuracy score on the reduced
data using only five features compare to those same scores when all
features are used? **

The F-score on the reduced data is much lower than on the full data
(61.2\% as opposed to 71.3\%, respectively). The difference is
substantial enough to raise concerns.

The accuracy on the reduced data is slightly lower than on the full data
(81.3\% as opposed to 85\%, respectively). The accuracy numbers
definitely seem close.

** If training time was a factor, would you consider using the reduced
data as your training set? **

Since the F-score is so much lower on the reduced data, I would not
consider using the reduced data as my training set. If I added two to
four more features, perhaps the F-score would be satisfactorily closer.

    \begin{quote}
\textbf{Note}: Once you have completed all of the code implementations
and successfully answered each question above, you may finalize your
work by exporting the iPython Notebook as an HTML document. You can do
this by using the menu above and navigating to\\
\textbf{File -\textgreater{} Download as -\textgreater{} HTML (.html)}.
Include the finished document along with this notebook as your
submission.
\end{quote}

    \#\#Before You Submit You will also need run the following in order to
convert the Jupyter notebook into HTML, so that your submission will
include both files.

    \begin{Verbatim}[commandchars=\\\{\}]
{\color{incolor}In [{\color{incolor}42}]:} \PY{o}{!!}jupyter nbconvert *.ipynb
\end{Verbatim}


\begin{Verbatim}[commandchars=\\\{\}]
{\color{outcolor}Out[{\color{outcolor}42}]:} ['[NbConvertApp] Converting notebook finding\_donors.ipynb to html',
          '[NbConvertApp] Writing 402306 bytes to finding\_donors.html']
\end{Verbatim}
            

    % Add a bibliography block to the postdoc
    
    
    
    \end{document}
