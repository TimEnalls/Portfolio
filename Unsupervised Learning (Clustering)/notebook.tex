
% Default to the notebook output style

    


% Inherit from the specified cell style.




    
\documentclass[11pt]{article}

    
    
    \usepackage[T1]{fontenc}
    % Nicer default font (+ math font) than Computer Modern for most use cases
    \usepackage{mathpazo}

    % Basic figure setup, for now with no caption control since it's done
    % automatically by Pandoc (which extracts ![](path) syntax from Markdown).
    \usepackage{graphicx}
    % We will generate all images so they have a width \maxwidth. This means
    % that they will get their normal width if they fit onto the page, but
    % are scaled down if they would overflow the margins.
    \makeatletter
    \def\maxwidth{\ifdim\Gin@nat@width>\linewidth\linewidth
    \else\Gin@nat@width\fi}
    \makeatother
    \let\Oldincludegraphics\includegraphics
    % Set max figure width to be 80% of text width, for now hardcoded.
    \renewcommand{\includegraphics}[1]{\Oldincludegraphics[width=.8\maxwidth]{#1}}
    % Ensure that by default, figures have no caption (until we provide a
    % proper Figure object with a Caption API and a way to capture that
    % in the conversion process - todo).
    \usepackage{caption}
    \DeclareCaptionLabelFormat{nolabel}{}
    \captionsetup{labelformat=nolabel}

    \usepackage{adjustbox} % Used to constrain images to a maximum size 
    \usepackage{xcolor} % Allow colors to be defined
    \usepackage{enumerate} % Needed for markdown enumerations to work
    \usepackage{geometry} % Used to adjust the document margins
    \usepackage{amsmath} % Equations
    \usepackage{amssymb} % Equations
    \usepackage{textcomp} % defines textquotesingle
    % Hack from http://tex.stackexchange.com/a/47451/13684:
    \AtBeginDocument{%
        \def\PYZsq{\textquotesingle}% Upright quotes in Pygmentized code
    }
    \usepackage{upquote} % Upright quotes for verbatim code
    \usepackage{eurosym} % defines \euro
    \usepackage[mathletters]{ucs} % Extended unicode (utf-8) support
    \usepackage[utf8x]{inputenc} % Allow utf-8 characters in the tex document
    \usepackage{fancyvrb} % verbatim replacement that allows latex
    \usepackage{grffile} % extends the file name processing of package graphics 
                         % to support a larger range 
    % The hyperref package gives us a pdf with properly built
    % internal navigation ('pdf bookmarks' for the table of contents,
    % internal cross-reference links, web links for URLs, etc.)
    \usepackage{hyperref}
    \usepackage{longtable} % longtable support required by pandoc >1.10
    \usepackage{booktabs}  % table support for pandoc > 1.12.2
    \usepackage[inline]{enumitem} % IRkernel/repr support (it uses the enumerate* environment)
    \usepackage[normalem]{ulem} % ulem is needed to support strikethroughs (\sout)
                                % normalem makes italics be italics, not underlines
    

    
    
    % Colors for the hyperref package
    \definecolor{urlcolor}{rgb}{0,.145,.698}
    \definecolor{linkcolor}{rgb}{.71,0.21,0.01}
    \definecolor{citecolor}{rgb}{.12,.54,.11}

    % ANSI colors
    \definecolor{ansi-black}{HTML}{3E424D}
    \definecolor{ansi-black-intense}{HTML}{282C36}
    \definecolor{ansi-red}{HTML}{E75C58}
    \definecolor{ansi-red-intense}{HTML}{B22B31}
    \definecolor{ansi-green}{HTML}{00A250}
    \definecolor{ansi-green-intense}{HTML}{007427}
    \definecolor{ansi-yellow}{HTML}{DDB62B}
    \definecolor{ansi-yellow-intense}{HTML}{B27D12}
    \definecolor{ansi-blue}{HTML}{208FFB}
    \definecolor{ansi-blue-intense}{HTML}{0065CA}
    \definecolor{ansi-magenta}{HTML}{D160C4}
    \definecolor{ansi-magenta-intense}{HTML}{A03196}
    \definecolor{ansi-cyan}{HTML}{60C6C8}
    \definecolor{ansi-cyan-intense}{HTML}{258F8F}
    \definecolor{ansi-white}{HTML}{C5C1B4}
    \definecolor{ansi-white-intense}{HTML}{A1A6B2}

    % commands and environments needed by pandoc snippets
    % extracted from the output of `pandoc -s`
    \providecommand{\tightlist}{%
      \setlength{\itemsep}{0pt}\setlength{\parskip}{0pt}}
    \DefineVerbatimEnvironment{Highlighting}{Verbatim}{commandchars=\\\{\}}
    % Add ',fontsize=\small' for more characters per line
    \newenvironment{Shaded}{}{}
    \newcommand{\KeywordTok}[1]{\textcolor[rgb]{0.00,0.44,0.13}{\textbf{{#1}}}}
    \newcommand{\DataTypeTok}[1]{\textcolor[rgb]{0.56,0.13,0.00}{{#1}}}
    \newcommand{\DecValTok}[1]{\textcolor[rgb]{0.25,0.63,0.44}{{#1}}}
    \newcommand{\BaseNTok}[1]{\textcolor[rgb]{0.25,0.63,0.44}{{#1}}}
    \newcommand{\FloatTok}[1]{\textcolor[rgb]{0.25,0.63,0.44}{{#1}}}
    \newcommand{\CharTok}[1]{\textcolor[rgb]{0.25,0.44,0.63}{{#1}}}
    \newcommand{\StringTok}[1]{\textcolor[rgb]{0.25,0.44,0.63}{{#1}}}
    \newcommand{\CommentTok}[1]{\textcolor[rgb]{0.38,0.63,0.69}{\textit{{#1}}}}
    \newcommand{\OtherTok}[1]{\textcolor[rgb]{0.00,0.44,0.13}{{#1}}}
    \newcommand{\AlertTok}[1]{\textcolor[rgb]{1.00,0.00,0.00}{\textbf{{#1}}}}
    \newcommand{\FunctionTok}[1]{\textcolor[rgb]{0.02,0.16,0.49}{{#1}}}
    \newcommand{\RegionMarkerTok}[1]{{#1}}
    \newcommand{\ErrorTok}[1]{\textcolor[rgb]{1.00,0.00,0.00}{\textbf{{#1}}}}
    \newcommand{\NormalTok}[1]{{#1}}
    
    % Additional commands for more recent versions of Pandoc
    \newcommand{\ConstantTok}[1]{\textcolor[rgb]{0.53,0.00,0.00}{{#1}}}
    \newcommand{\SpecialCharTok}[1]{\textcolor[rgb]{0.25,0.44,0.63}{{#1}}}
    \newcommand{\VerbatimStringTok}[1]{\textcolor[rgb]{0.25,0.44,0.63}{{#1}}}
    \newcommand{\SpecialStringTok}[1]{\textcolor[rgb]{0.73,0.40,0.53}{{#1}}}
    \newcommand{\ImportTok}[1]{{#1}}
    \newcommand{\DocumentationTok}[1]{\textcolor[rgb]{0.73,0.13,0.13}{\textit{{#1}}}}
    \newcommand{\AnnotationTok}[1]{\textcolor[rgb]{0.38,0.63,0.69}{\textbf{\textit{{#1}}}}}
    \newcommand{\CommentVarTok}[1]{\textcolor[rgb]{0.38,0.63,0.69}{\textbf{\textit{{#1}}}}}
    \newcommand{\VariableTok}[1]{\textcolor[rgb]{0.10,0.09,0.49}{{#1}}}
    \newcommand{\ControlFlowTok}[1]{\textcolor[rgb]{0.00,0.44,0.13}{\textbf{{#1}}}}
    \newcommand{\OperatorTok}[1]{\textcolor[rgb]{0.40,0.40,0.40}{{#1}}}
    \newcommand{\BuiltInTok}[1]{{#1}}
    \newcommand{\ExtensionTok}[1]{{#1}}
    \newcommand{\PreprocessorTok}[1]{\textcolor[rgb]{0.74,0.48,0.00}{{#1}}}
    \newcommand{\AttributeTok}[1]{\textcolor[rgb]{0.49,0.56,0.16}{{#1}}}
    \newcommand{\InformationTok}[1]{\textcolor[rgb]{0.38,0.63,0.69}{\textbf{\textit{{#1}}}}}
    \newcommand{\WarningTok}[1]{\textcolor[rgb]{0.38,0.63,0.69}{\textbf{\textit{{#1}}}}}
    
    
    % Define a nice break command that doesn't care if a line doesn't already
    % exist.
    \def\br{\hspace*{\fill} \\* }
    % Math Jax compatability definitions
    \def\gt{>}
    \def\lt{<}
    % Document parameters
    \title{Identify\_Customer\_Segments}
    
    
    

    % Pygments definitions
    
\makeatletter
\def\PY@reset{\let\PY@it=\relax \let\PY@bf=\relax%
    \let\PY@ul=\relax \let\PY@tc=\relax%
    \let\PY@bc=\relax \let\PY@ff=\relax}
\def\PY@tok#1{\csname PY@tok@#1\endcsname}
\def\PY@toks#1+{\ifx\relax#1\empty\else%
    \PY@tok{#1}\expandafter\PY@toks\fi}
\def\PY@do#1{\PY@bc{\PY@tc{\PY@ul{%
    \PY@it{\PY@bf{\PY@ff{#1}}}}}}}
\def\PY#1#2{\PY@reset\PY@toks#1+\relax+\PY@do{#2}}

\expandafter\def\csname PY@tok@w\endcsname{\def\PY@tc##1{\textcolor[rgb]{0.73,0.73,0.73}{##1}}}
\expandafter\def\csname PY@tok@c\endcsname{\let\PY@it=\textit\def\PY@tc##1{\textcolor[rgb]{0.25,0.50,0.50}{##1}}}
\expandafter\def\csname PY@tok@cp\endcsname{\def\PY@tc##1{\textcolor[rgb]{0.74,0.48,0.00}{##1}}}
\expandafter\def\csname PY@tok@k\endcsname{\let\PY@bf=\textbf\def\PY@tc##1{\textcolor[rgb]{0.00,0.50,0.00}{##1}}}
\expandafter\def\csname PY@tok@kp\endcsname{\def\PY@tc##1{\textcolor[rgb]{0.00,0.50,0.00}{##1}}}
\expandafter\def\csname PY@tok@kt\endcsname{\def\PY@tc##1{\textcolor[rgb]{0.69,0.00,0.25}{##1}}}
\expandafter\def\csname PY@tok@o\endcsname{\def\PY@tc##1{\textcolor[rgb]{0.40,0.40,0.40}{##1}}}
\expandafter\def\csname PY@tok@ow\endcsname{\let\PY@bf=\textbf\def\PY@tc##1{\textcolor[rgb]{0.67,0.13,1.00}{##1}}}
\expandafter\def\csname PY@tok@nb\endcsname{\def\PY@tc##1{\textcolor[rgb]{0.00,0.50,0.00}{##1}}}
\expandafter\def\csname PY@tok@nf\endcsname{\def\PY@tc##1{\textcolor[rgb]{0.00,0.00,1.00}{##1}}}
\expandafter\def\csname PY@tok@nc\endcsname{\let\PY@bf=\textbf\def\PY@tc##1{\textcolor[rgb]{0.00,0.00,1.00}{##1}}}
\expandafter\def\csname PY@tok@nn\endcsname{\let\PY@bf=\textbf\def\PY@tc##1{\textcolor[rgb]{0.00,0.00,1.00}{##1}}}
\expandafter\def\csname PY@tok@ne\endcsname{\let\PY@bf=\textbf\def\PY@tc##1{\textcolor[rgb]{0.82,0.25,0.23}{##1}}}
\expandafter\def\csname PY@tok@nv\endcsname{\def\PY@tc##1{\textcolor[rgb]{0.10,0.09,0.49}{##1}}}
\expandafter\def\csname PY@tok@no\endcsname{\def\PY@tc##1{\textcolor[rgb]{0.53,0.00,0.00}{##1}}}
\expandafter\def\csname PY@tok@nl\endcsname{\def\PY@tc##1{\textcolor[rgb]{0.63,0.63,0.00}{##1}}}
\expandafter\def\csname PY@tok@ni\endcsname{\let\PY@bf=\textbf\def\PY@tc##1{\textcolor[rgb]{0.60,0.60,0.60}{##1}}}
\expandafter\def\csname PY@tok@na\endcsname{\def\PY@tc##1{\textcolor[rgb]{0.49,0.56,0.16}{##1}}}
\expandafter\def\csname PY@tok@nt\endcsname{\let\PY@bf=\textbf\def\PY@tc##1{\textcolor[rgb]{0.00,0.50,0.00}{##1}}}
\expandafter\def\csname PY@tok@nd\endcsname{\def\PY@tc##1{\textcolor[rgb]{0.67,0.13,1.00}{##1}}}
\expandafter\def\csname PY@tok@s\endcsname{\def\PY@tc##1{\textcolor[rgb]{0.73,0.13,0.13}{##1}}}
\expandafter\def\csname PY@tok@sd\endcsname{\let\PY@it=\textit\def\PY@tc##1{\textcolor[rgb]{0.73,0.13,0.13}{##1}}}
\expandafter\def\csname PY@tok@si\endcsname{\let\PY@bf=\textbf\def\PY@tc##1{\textcolor[rgb]{0.73,0.40,0.53}{##1}}}
\expandafter\def\csname PY@tok@se\endcsname{\let\PY@bf=\textbf\def\PY@tc##1{\textcolor[rgb]{0.73,0.40,0.13}{##1}}}
\expandafter\def\csname PY@tok@sr\endcsname{\def\PY@tc##1{\textcolor[rgb]{0.73,0.40,0.53}{##1}}}
\expandafter\def\csname PY@tok@ss\endcsname{\def\PY@tc##1{\textcolor[rgb]{0.10,0.09,0.49}{##1}}}
\expandafter\def\csname PY@tok@sx\endcsname{\def\PY@tc##1{\textcolor[rgb]{0.00,0.50,0.00}{##1}}}
\expandafter\def\csname PY@tok@m\endcsname{\def\PY@tc##1{\textcolor[rgb]{0.40,0.40,0.40}{##1}}}
\expandafter\def\csname PY@tok@gh\endcsname{\let\PY@bf=\textbf\def\PY@tc##1{\textcolor[rgb]{0.00,0.00,0.50}{##1}}}
\expandafter\def\csname PY@tok@gu\endcsname{\let\PY@bf=\textbf\def\PY@tc##1{\textcolor[rgb]{0.50,0.00,0.50}{##1}}}
\expandafter\def\csname PY@tok@gd\endcsname{\def\PY@tc##1{\textcolor[rgb]{0.63,0.00,0.00}{##1}}}
\expandafter\def\csname PY@tok@gi\endcsname{\def\PY@tc##1{\textcolor[rgb]{0.00,0.63,0.00}{##1}}}
\expandafter\def\csname PY@tok@gr\endcsname{\def\PY@tc##1{\textcolor[rgb]{1.00,0.00,0.00}{##1}}}
\expandafter\def\csname PY@tok@ge\endcsname{\let\PY@it=\textit}
\expandafter\def\csname PY@tok@gs\endcsname{\let\PY@bf=\textbf}
\expandafter\def\csname PY@tok@gp\endcsname{\let\PY@bf=\textbf\def\PY@tc##1{\textcolor[rgb]{0.00,0.00,0.50}{##1}}}
\expandafter\def\csname PY@tok@go\endcsname{\def\PY@tc##1{\textcolor[rgb]{0.53,0.53,0.53}{##1}}}
\expandafter\def\csname PY@tok@gt\endcsname{\def\PY@tc##1{\textcolor[rgb]{0.00,0.27,0.87}{##1}}}
\expandafter\def\csname PY@tok@err\endcsname{\def\PY@bc##1{\setlength{\fboxsep}{0pt}\fcolorbox[rgb]{1.00,0.00,0.00}{1,1,1}{\strut ##1}}}
\expandafter\def\csname PY@tok@kc\endcsname{\let\PY@bf=\textbf\def\PY@tc##1{\textcolor[rgb]{0.00,0.50,0.00}{##1}}}
\expandafter\def\csname PY@tok@kd\endcsname{\let\PY@bf=\textbf\def\PY@tc##1{\textcolor[rgb]{0.00,0.50,0.00}{##1}}}
\expandafter\def\csname PY@tok@kn\endcsname{\let\PY@bf=\textbf\def\PY@tc##1{\textcolor[rgb]{0.00,0.50,0.00}{##1}}}
\expandafter\def\csname PY@tok@kr\endcsname{\let\PY@bf=\textbf\def\PY@tc##1{\textcolor[rgb]{0.00,0.50,0.00}{##1}}}
\expandafter\def\csname PY@tok@bp\endcsname{\def\PY@tc##1{\textcolor[rgb]{0.00,0.50,0.00}{##1}}}
\expandafter\def\csname PY@tok@fm\endcsname{\def\PY@tc##1{\textcolor[rgb]{0.00,0.00,1.00}{##1}}}
\expandafter\def\csname PY@tok@vc\endcsname{\def\PY@tc##1{\textcolor[rgb]{0.10,0.09,0.49}{##1}}}
\expandafter\def\csname PY@tok@vg\endcsname{\def\PY@tc##1{\textcolor[rgb]{0.10,0.09,0.49}{##1}}}
\expandafter\def\csname PY@tok@vi\endcsname{\def\PY@tc##1{\textcolor[rgb]{0.10,0.09,0.49}{##1}}}
\expandafter\def\csname PY@tok@vm\endcsname{\def\PY@tc##1{\textcolor[rgb]{0.10,0.09,0.49}{##1}}}
\expandafter\def\csname PY@tok@sa\endcsname{\def\PY@tc##1{\textcolor[rgb]{0.73,0.13,0.13}{##1}}}
\expandafter\def\csname PY@tok@sb\endcsname{\def\PY@tc##1{\textcolor[rgb]{0.73,0.13,0.13}{##1}}}
\expandafter\def\csname PY@tok@sc\endcsname{\def\PY@tc##1{\textcolor[rgb]{0.73,0.13,0.13}{##1}}}
\expandafter\def\csname PY@tok@dl\endcsname{\def\PY@tc##1{\textcolor[rgb]{0.73,0.13,0.13}{##1}}}
\expandafter\def\csname PY@tok@s2\endcsname{\def\PY@tc##1{\textcolor[rgb]{0.73,0.13,0.13}{##1}}}
\expandafter\def\csname PY@tok@sh\endcsname{\def\PY@tc##1{\textcolor[rgb]{0.73,0.13,0.13}{##1}}}
\expandafter\def\csname PY@tok@s1\endcsname{\def\PY@tc##1{\textcolor[rgb]{0.73,0.13,0.13}{##1}}}
\expandafter\def\csname PY@tok@mb\endcsname{\def\PY@tc##1{\textcolor[rgb]{0.40,0.40,0.40}{##1}}}
\expandafter\def\csname PY@tok@mf\endcsname{\def\PY@tc##1{\textcolor[rgb]{0.40,0.40,0.40}{##1}}}
\expandafter\def\csname PY@tok@mh\endcsname{\def\PY@tc##1{\textcolor[rgb]{0.40,0.40,0.40}{##1}}}
\expandafter\def\csname PY@tok@mi\endcsname{\def\PY@tc##1{\textcolor[rgb]{0.40,0.40,0.40}{##1}}}
\expandafter\def\csname PY@tok@il\endcsname{\def\PY@tc##1{\textcolor[rgb]{0.40,0.40,0.40}{##1}}}
\expandafter\def\csname PY@tok@mo\endcsname{\def\PY@tc##1{\textcolor[rgb]{0.40,0.40,0.40}{##1}}}
\expandafter\def\csname PY@tok@ch\endcsname{\let\PY@it=\textit\def\PY@tc##1{\textcolor[rgb]{0.25,0.50,0.50}{##1}}}
\expandafter\def\csname PY@tok@cm\endcsname{\let\PY@it=\textit\def\PY@tc##1{\textcolor[rgb]{0.25,0.50,0.50}{##1}}}
\expandafter\def\csname PY@tok@cpf\endcsname{\let\PY@it=\textit\def\PY@tc##1{\textcolor[rgb]{0.25,0.50,0.50}{##1}}}
\expandafter\def\csname PY@tok@c1\endcsname{\let\PY@it=\textit\def\PY@tc##1{\textcolor[rgb]{0.25,0.50,0.50}{##1}}}
\expandafter\def\csname PY@tok@cs\endcsname{\let\PY@it=\textit\def\PY@tc##1{\textcolor[rgb]{0.25,0.50,0.50}{##1}}}

\def\PYZbs{\char`\\}
\def\PYZus{\char`\_}
\def\PYZob{\char`\{}
\def\PYZcb{\char`\}}
\def\PYZca{\char`\^}
\def\PYZam{\char`\&}
\def\PYZlt{\char`\<}
\def\PYZgt{\char`\>}
\def\PYZsh{\char`\#}
\def\PYZpc{\char`\%}
\def\PYZdl{\char`\$}
\def\PYZhy{\char`\-}
\def\PYZsq{\char`\'}
\def\PYZdq{\char`\"}
\def\PYZti{\char`\~}
% for compatibility with earlier versions
\def\PYZat{@}
\def\PYZlb{[}
\def\PYZrb{]}
\makeatother


    % Exact colors from NB
    \definecolor{incolor}{rgb}{0.0, 0.0, 0.5}
    \definecolor{outcolor}{rgb}{0.545, 0.0, 0.0}



    
    % Prevent overflowing lines due to hard-to-break entities
    \sloppy 
    % Setup hyperref package
    \hypersetup{
      breaklinks=true,  % so long urls are correctly broken across lines
      colorlinks=true,
      urlcolor=urlcolor,
      linkcolor=linkcolor,
      citecolor=citecolor,
      }
    % Slightly bigger margins than the latex defaults
    
    \geometry{verbose,tmargin=1in,bmargin=1in,lmargin=1in,rmargin=1in}
    
    

    \begin{document}
    
    
    \maketitle
    
    

    
    \hypertarget{project-identify-customer-segments}{%
\section{Project: Identify Customer
Segments}\label{project-identify-customer-segments}}

In this project, you will apply unsupervised learning techniques to
identify segments of the population that form the core customer base for
a mail-order sales company in Germany. These segments can then be used
to direct marketing campaigns towards audiences that will have the
highest expected rate of returns. The data that you will use has been
provided by our partners at Bertelsmann Arvato Analytics, and represents
a real-life data science task.

This notebook will help you complete this task by providing a framework
within which you will perform your analysis steps. In each step of the
project, you will see some text describing the subtask that you will
perform, followed by one or more code cells for you to complete your
work. \textbf{Feel free to add additional code and markdown cells as you
go along so that you can explore everything in precise chunks.} The code
cells provided in the base template will outline only the major tasks,
and will usually not be enough to cover all of the minor tasks that
comprise it.

It should be noted that while there will be precise guidelines on how
you should handle certain tasks in the project, there will also be
places where an exact specification is not provided. \textbf{There will
be times in the project where you will need to make and justify your own
decisions on how to treat the data.} These are places where there may
not be only one way to handle the data. In real-life tasks, there may be
many valid ways to approach an analysis task. One of the most important
things you can do is clearly document your approach so that other
scientists can understand the decisions you've made.

At the end of most sections, there will be a Markdown cell labeled
\textbf{Discussion}. In these cells, you will report your findings for
the completed section, as well as document the decisions that you made
in your approach to each subtask. \textbf{Your project will be evaluated
not just on the code used to complete the tasks outlined, but also your
communication about your observations and conclusions at each stage.}

    \begin{Verbatim}[commandchars=\\\{\}]
{\color{incolor}In [{\color{incolor}6}]:} \PY{c+c1}{\PYZsh{} import libraries here; add more as necessary}
        \PY{k+kn}{import} \PY{n+nn}{numpy} \PY{k}{as} \PY{n+nn}{np}
        \PY{k+kn}{import} \PY{n+nn}{pandas} \PY{k}{as} \PY{n+nn}{pd}
        \PY{k+kn}{import} \PY{n+nn}{matplotlib}\PY{n+nn}{.}\PY{n+nn}{pyplot} \PY{k}{as} \PY{n+nn}{plt}
        \PY{k+kn}{import} \PY{n+nn}{seaborn} \PY{k}{as} \PY{n+nn}{sns}
        
        \PY{c+c1}{\PYZsh{} magic word for producing visualizations in notebook}
        \PY{o}{\PYZpc{}}\PY{k}{matplotlib} inline
\end{Verbatim}


    \hypertarget{step-0-load-the-data}{%
\subsubsection{Step 0: Load the Data}\label{step-0-load-the-data}}

There are four files associated with this project (not including this
one):

\begin{itemize}
\tightlist
\item
  \texttt{Udacity\_AZDIAS\_Subset.csv}: Demographics data for the
  general population of Germany; 891211 persons (rows) x 85 features
  (columns).
\item
  \texttt{Udacity\_CUSTOMERS\_Subset.csv}: Demographics data for
  customers of a mail-order company; 191652 persons (rows) x 85 features
  (columns).
\item
  \texttt{Data\_Dictionary.md}: Detailed information file about the
  features in the provided datasets.
\item
  \texttt{AZDIAS\_Feature\_Summary.csv}: Summary of feature attributes
  for demographics data; 85 features (rows) x 4 columns
\end{itemize}

Each row of the demographics files represents a single person, but also
includes information outside of individuals, including information about
their household, building, and neighborhood. You will use this
information to cluster the general population into groups with similar
demographic properties. Then, you will see how the people in the
customers dataset fit into those created clusters. The hope here is that
certain clusters are over-represented in the customers data, as compared
to the general population; those over-represented clusters will be
assumed to be part of the core userbase. This information can then be
used for further applications, such as targeting for a marketing
campaign.

To start off with, load in the demographics data for the general
population into a pandas DataFrame, and do the same for the feature
attributes summary. Note for all of the \texttt{.csv} data files in this
project: they're semicolon (\texttt{;}) delimited, so you'll need an
additional argument in your
\href{https://pandas.pydata.org/pandas-docs/stable/generated/pandas.read_csv.html}{\texttt{read\_csv()}}
call to read in the data properly. Also, considering the size of the
main dataset, it may take some time for it to load completely.

Once the dataset is loaded, it's recommended that you take a little bit
of time just browsing the general structure of the dataset and feature
summary file. You'll be getting deep into the innards of the cleaning in
the first major step of the project, so gaining some general familiarity
can help you get your bearings.

    \begin{Verbatim}[commandchars=\\\{\}]
{\color{incolor}In [{\color{incolor}7}]:} \PY{c+c1}{\PYZsh{} Load in the general demographics data.}
        \PY{n}{azdias} \PY{o}{=} \PY{n}{pd}\PY{o}{.}\PY{n}{read\PYZus{}csv}\PY{p}{(}\PY{l+s+s1}{\PYZsq{}}\PY{l+s+s1}{Udacity\PYZus{}AZDIAS\PYZus{}Subset.csv}\PY{l+s+s1}{\PYZsq{}}\PY{p}{,} \PY{n}{sep}\PY{o}{=}\PY{l+s+s1}{\PYZsq{}}\PY{l+s+s1}{;}\PY{l+s+s1}{\PYZsq{}}\PY{p}{)}
        
        \PY{c+c1}{\PYZsh{} Load in the feature summary file.}
        \PY{n}{feat\PYZus{}info} \PY{o}{=} \PY{n}{pd}\PY{o}{.}\PY{n}{read\PYZus{}csv}\PY{p}{(}\PY{l+s+s1}{\PYZsq{}}\PY{l+s+s1}{AZDIAS\PYZus{}Feature\PYZus{}Summary.csv}\PY{l+s+s1}{\PYZsq{}}\PY{p}{,} \PY{n}{sep}\PY{o}{=}\PY{l+s+s1}{\PYZsq{}}\PY{l+s+s1}{;}\PY{l+s+s1}{\PYZsq{}}\PY{p}{)}
\end{Verbatim}


    \begin{Verbatim}[commandchars=\\\{\}]
{\color{incolor}In [{\color{incolor}8}]:} \PY{c+c1}{\PYZsh{} Check the structure of the data after it\PYZsq{}s loaded (e.g. print the number of}
        \PY{c+c1}{\PYZsh{} rows and columns, print the first few rows).}
        
        \PY{c+c1}{\PYZsh{} Number of rows in azidias}
        \PY{n}{azidias\PYZus{}nrows} \PY{o}{=} \PY{p}{(}\PY{n+nb}{len}\PY{p}{(}\PY{n}{azdias}\PY{p}{)}\PY{p}{)}
        
        \PY{c+c1}{\PYZsh{} Number of columns in azidias}
        \PY{n}{azidias\PYZus{}ncolumns} \PY{o}{=} \PY{p}{(}\PY{n+nb}{len}\PY{p}{(}\PY{n}{azdias}\PY{o}{.}\PY{n}{columns}\PY{p}{)}\PY{p}{)}
        
        \PY{c+c1}{\PYZsh{} Number of rows in feat\PYZus{}info}
        \PY{n}{feat\PYZus{}info\PYZus{}nrows} \PY{o}{=} \PY{p}{(}\PY{n+nb}{len}\PY{p}{(}\PY{n}{feat\PYZus{}info}\PY{p}{)}\PY{p}{)}
        
        \PY{c+c1}{\PYZsh{} Number of columns in feat\PYZus{}info}
        \PY{n}{feat\PYZus{}info\PYZus{}ncolumns} \PY{o}{=} \PY{p}{(}\PY{n+nb}{len}\PY{p}{(}\PY{n}{feat\PYZus{}info}\PY{o}{.}\PY{n}{columns}\PY{p}{)}\PY{p}{)}
        
        \PY{c+c1}{\PYZsh{} Print the number of rows and columns in azidias and feat\PYZus{}info}
        \PY{n+nb}{print}\PY{p}{(}\PY{l+s+s2}{\PYZdq{}}\PY{l+s+s2}{The total number of rows in azidias: }\PY{l+s+si}{\PYZob{}\PYZcb{}}\PY{l+s+s2}{\PYZdq{}}\PY{o}{.}\PY{n}{format}\PY{p}{(}\PY{n}{azidias\PYZus{}nrows}\PY{p}{)}\PY{p}{)}
        \PY{n+nb}{print}\PY{p}{(}\PY{l+s+s2}{\PYZdq{}}\PY{l+s+s2}{The total number of columns in azidias: }\PY{l+s+si}{\PYZob{}\PYZcb{}}\PY{l+s+s2}{\PYZdq{}}\PY{o}{.}\PY{n}{format}\PY{p}{(}\PY{n}{azidias\PYZus{}ncolumns}\PY{p}{)}\PY{p}{)}
        \PY{n+nb}{print}\PY{p}{(}\PY{l+s+s2}{\PYZdq{}}\PY{l+s+s2}{The total number of rows in feat\PYZus{}info: }\PY{l+s+si}{\PYZob{}\PYZcb{}}\PY{l+s+s2}{\PYZdq{}}\PY{o}{.}\PY{n}{format}\PY{p}{(}\PY{n}{feat\PYZus{}info\PYZus{}nrows}\PY{p}{)}\PY{p}{)}
        \PY{n+nb}{print}\PY{p}{(}\PY{l+s+s2}{\PYZdq{}}\PY{l+s+s2}{The total number of columns in feat\PYZus{}info: }\PY{l+s+si}{\PYZob{}\PYZcb{}}\PY{l+s+s2}{\PYZdq{}}\PY{o}{.}\PY{n}{format}\PY{p}{(}\PY{n}{feat\PYZus{}info\PYZus{}ncolumns}\PY{p}{)}\PY{p}{)}
\end{Verbatim}


    \begin{Verbatim}[commandchars=\\\{\}]
The total number of rows in azidias: 891221
The total number of columns in azidias: 85
The total number of rows in feat\_info: 85
The total number of columns in feat\_info: 4

    \end{Verbatim}

    \begin{Verbatim}[commandchars=\\\{\}]
{\color{incolor}In [{\color{incolor}9}]:} \PY{c+c1}{\PYZsh{} First 5 records of azidias}
        \PY{n}{display}\PY{p}{(}\PY{n}{azdias}\PY{o}{.}\PY{n}{head}\PY{p}{(}\PY{n}{n}\PY{o}{=}\PY{l+m+mi}{20}\PY{p}{)}\PY{p}{)}
        
        \PY{c+c1}{\PYZsh{} First 5 records of feat\PYZus{}info}
        \PY{n}{display}\PY{p}{(}\PY{n}{feat\PYZus{}info}\PY{o}{.}\PY{n}{head}\PY{p}{(}\PY{n}{n}\PY{o}{=}\PY{l+m+mi}{20}\PY{p}{)}\PY{p}{)}
\end{Verbatim}


    
    \begin{verbatim}
    AGER_TYP  ALTERSKATEGORIE_GROB  ANREDE_KZ  CJT_GESAMTTYP  \
0         -1                     2          1            2.0   
1         -1                     1          2            5.0   
2         -1                     3          2            3.0   
3          2                     4          2            2.0   
4         -1                     3          1            5.0   
5          3                     1          2            2.0   
6         -1                     2          2            5.0   
7         -1                     1          1            3.0   
8         -1                     3          1            3.0   
9         -1                     3          2            4.0   
10         0                     3          2            1.0   
11        -1                     2          1            6.0   
12        -1                     3          1            6.0   
13        -1                     1          2            5.0   
14        -1                     3          1            6.0   
15         1                     4          2            4.0   
16        -1                     1          2            1.0   
17        -1                     2          1            6.0   
18        -1                     2          2            6.0   
19        -1                     3          1            3.0   

    FINANZ_MINIMALIST  FINANZ_SPARER  FINANZ_VORSORGER  FINANZ_ANLEGER  \
0                   3              4                 3               5   
1                   1              5                 2               5   
2                   1              4                 1               2   
3                   4              2                 5               2   
4                   4              3                 4               1   
5                   3              1                 5               2   
6                   1              5                 1               5   
7                   3              3                 4               1   
8                   4              4                 2               4   
9                   2              4                 2               3   
10                  2              2                 5               3   
11                  3              4                 3               5   
12                  5              3                 4               2   
13                  1              4                 3               5   
14                  3              4                 3               5   
15                  4              1                 5               1   
16                  4              3                 1               4   
17                  3              4                 3               5   
18                  2              4                 1               5   
19                  5              2                 3               1   

    FINANZ_UNAUFFAELLIGER  FINANZ_HAUSBAUER    ...     PLZ8_ANTG1  PLZ8_ANTG2  \
0                       5                 3    ...            NaN         NaN   
1                       4                 5    ...            2.0         3.0   
2                       3                 5    ...            3.0         3.0   
3                       1                 2    ...            2.0         2.0   
4                       3                 2    ...            2.0         4.0   
5                       2                 5    ...            2.0         3.0   
6                       4                 3    ...            3.0         3.0   
7                       3                 2    ...            3.0         3.0   
8                       2                 2    ...            2.0         3.0   
9                       5                 4    ...            2.0         3.0   
10                      1                 5    ...            2.0         4.0   
11                      5                 3    ...            NaN         NaN   
12                      4                 1    ...            3.0         3.0   
13                      5                 2    ...            2.0         1.0   
14                      5                 3    ...            NaN         NaN   
15                      1                 4    ...            NaN         NaN   
16                      5                 1    ...            3.0         3.0   
17                      5                 3    ...            NaN         NaN   
18                      4                 1    ...            2.0         3.0   
19                      3                 1    ...            2.0         4.0   

    PLZ8_ANTG3  PLZ8_ANTG4  PLZ8_BAUMAX  PLZ8_HHZ  PLZ8_GBZ  ARBEIT  \
0          NaN         NaN          NaN       NaN       NaN     NaN   
1          2.0         1.0          1.0       5.0       4.0     3.0   
2          1.0         0.0          1.0       4.0       4.0     3.0   
3          2.0         0.0          1.0       3.0       4.0     2.0   
4          2.0         1.0          2.0       3.0       3.0     4.0   
5          1.0         1.0          1.0       5.0       5.0     2.0   
6          1.0         0.0          1.0       5.0       5.0     4.0   
7          1.0         0.0          1.0       4.0       4.0     2.0   
8          2.0         1.0          1.0       3.0       3.0     2.0   
9          2.0         1.0          1.0       3.0       3.0     2.0   
10         2.0         0.0          2.0       3.0       3.0     4.0   
11         NaN         NaN          NaN       NaN       NaN     NaN   
12         1.0         0.0          1.0       5.0       5.0     3.0   
13         1.0         1.0          1.0       3.0       3.0     3.0   
14         NaN         NaN          NaN       NaN       NaN     NaN   
15         NaN         NaN          NaN       NaN       NaN     4.0   
16         1.0         0.0          1.0       3.0       4.0     1.0   
17         NaN         NaN          NaN       NaN       NaN     NaN   
18         2.0         1.0          1.0       3.0       3.0     3.0   
19         2.0         1.0          2.0       5.0       4.0     4.0   

    ORTSGR_KLS9  RELAT_AB  
0           NaN       NaN  
1           5.0       4.0  
2           5.0       2.0  
3           3.0       3.0  
4           6.0       5.0  
5           3.0       3.0  
6           6.0       3.0  
7           5.0       2.0  
8           4.0       3.0  
9           3.0       1.0  
10          6.0       5.0  
11          NaN       NaN  
12          6.0       4.0  
13          6.0       4.0  
14          NaN       NaN  
15          8.0       5.0  
16          2.0       1.0  
17          NaN       NaN  
18          4.0       3.0  
19          6.0       3.0  

[20 rows x 85 columns]
    \end{verbatim}

    
    
    \begin{verbatim}
                attribute information_level         type missing_or_unknown
0                AGER_TYP            person  categorical             [-1,0]
1    ALTERSKATEGORIE_GROB            person      ordinal           [-1,0,9]
2               ANREDE_KZ            person  categorical             [-1,0]
3           CJT_GESAMTTYP            person  categorical                [0]
4       FINANZ_MINIMALIST            person      ordinal               [-1]
5           FINANZ_SPARER            person      ordinal               [-1]
6        FINANZ_VORSORGER            person      ordinal               [-1]
7          FINANZ_ANLEGER            person      ordinal               [-1]
8   FINANZ_UNAUFFAELLIGER            person      ordinal               [-1]
9        FINANZ_HAUSBAUER            person      ordinal               [-1]
10              FINANZTYP            person  categorical               [-1]
11            GEBURTSJAHR            person      numeric                [0]
12        GFK_URLAUBERTYP            person  categorical                 []
13       GREEN_AVANTGARDE            person  categorical                 []
14             HEALTH_TYP            person      ordinal             [-1,0]
15    LP_LEBENSPHASE_FEIN            person        mixed                [0]
16    LP_LEBENSPHASE_GROB            person        mixed                [0]
17        LP_FAMILIE_FEIN            person  categorical                [0]
18        LP_FAMILIE_GROB            person  categorical                [0]
19         LP_STATUS_FEIN            person  categorical                [0]
    \end{verbatim}

    
    \begin{quote}
\textbf{Tip}: Add additional cells to keep everything in
reasonably-sized chunks! Keyboard shortcut
\texttt{esc\ -\/-\textgreater{}\ a} (press escape to enter command mode,
then press the `A' key) adds a new cell before the active cell, and
\texttt{esc\ -\/-\textgreater{}\ b} adds a new cell after the active
cell. If you need to convert an active cell to a markdown cell, use
\texttt{esc\ -\/-\textgreater{}\ m} and to convert to a code cell, use
\texttt{esc\ -\/-\textgreater{}\ y}.
\end{quote}

\hypertarget{step-1-preprocessing}{%
\subsection{Step 1: Preprocessing}\label{step-1-preprocessing}}

\hypertarget{step-1.1-assess-missing-data}{%
\subsubsection{Step 1.1: Assess Missing
Data}\label{step-1.1-assess-missing-data}}

The feature summary file contains a summary of properties for each
demographics data column. You will use this file to help you make
cleaning decisions during this stage of the project. First of all, you
should assess the demographics data in terms of missing data. Pay
attention to the following points as you perform your analysis, and take
notes on what you observe. Make sure that you fill in the
\textbf{Discussion} cell with your findings and decisions at the end of
each step that has one!

\hypertarget{step-1.1.1-convert-missing-value-codes-to-nans}{%
\paragraph{Step 1.1.1: Convert Missing Value Codes to
NaNs}\label{step-1.1.1-convert-missing-value-codes-to-nans}}

The fourth column of the feature attributes summary (loaded in above as
\texttt{feat\_info}) documents the codes from the data dictionary that
indicate missing or unknown data. While the file encodes this as a list
(e.g. \texttt{{[}-1,0{]}}), this will get read in as a string object.
You'll need to do a little bit of parsing to make use of it to identify
and clean the data. Convert data that matches a `missing' or `unknown'
value code into a numpy NaN value. You might want to see how much data
takes on a `missing' or `unknown' code, and how much data is naturally
missing, as a point of interest.

\textbf{As one more reminder, you are encouraged to add additional cells
to break up your analysis into manageable chunks.}

    \begin{Verbatim}[commandchars=\\\{\}]
{\color{incolor}In [{\color{incolor}15}]:} \PY{c+c1}{\PYZsh{} Identify missing or unknown data values and convert them to NaNs.}
         
         \PY{c+c1}{\PYZsh{} Number of attributes in dataset that are 0}
         \PY{c+c1}{\PYZsh{} azdias.size \PYZhy{} np.count\PYZus{}nonzero(azdias)}
         
         \PY{c+c1}{\PYZsh{} feat\PYZus{}info[\PYZsq{}missing\PYZus{}or\PYZus{}unknown\PYZsq{}].value\PYZus{}counts()}
         
         \PY{c+c1}{\PYZsh{}a = feat\PYZus{}info}
                             
         \PY{c+c1}{\PYZsh{} Create a copy of \PYZsq{}missing\PYZus{}or\PYZus{}unknown\PYZsq{} column that is converted to floats result = []}
         
         
         \PY{n}{feat\PYZus{}info}\PY{o}{.}\PY{n}{head}\PY{p}{(}\PY{p}{)}
         
         
         \PY{n}{feat\PYZus{}info} \PY{o}{=} \PY{n}{feat\PYZus{}info}\PY{o}{.}\PY{n}{replace}\PY{p}{(}\PY{l+s+s1}{\PYZsq{}}\PY{l+s+s1}{[\PYZhy{}1,0]}\PY{l+s+s1}{\PYZsq{}}\PY{p}{,} \PY{n}{np}\PY{o}{.}\PY{n}{nan}\PY{p}{)}
         \PY{n}{feat\PYZus{}info}
\end{Verbatim}


\begin{Verbatim}[commandchars=\\\{\}]
{\color{outcolor}Out[{\color{outcolor}15}]:}                 attribute information\_level         type missing\_or\_unknown
         0                AGER\_TYP            person  categorical                NaN
         1    ALTERSKATEGORIE\_GROB            person      ordinal           [-1,0,9]
         2               ANREDE\_KZ            person  categorical                NaN
         3           CJT\_GESAMTTYP            person  categorical                [0]
         4       FINANZ\_MINIMALIST            person      ordinal               [-1]
         5           FINANZ\_SPARER            person      ordinal               [-1]
         6        FINANZ\_VORSORGER            person      ordinal               [-1]
         7          FINANZ\_ANLEGER            person      ordinal               [-1]
         8   FINANZ\_UNAUFFAELLIGER            person      ordinal               [-1]
         9        FINANZ\_HAUSBAUER            person      ordinal               [-1]
         10              FINANZTYP            person  categorical               [-1]
         11            GEBURTSJAHR            person      numeric                [0]
         12        GFK\_URLAUBERTYP            person  categorical                 []
         13       GREEN\_AVANTGARDE            person  categorical                 []
         14             HEALTH\_TYP            person      ordinal                NaN
         15    LP\_LEBENSPHASE\_FEIN            person        mixed                [0]
         16    LP\_LEBENSPHASE\_GROB            person        mixed                [0]
         17        LP\_FAMILIE\_FEIN            person  categorical                [0]
         18        LP\_FAMILIE\_GROB            person  categorical                [0]
         19         LP\_STATUS\_FEIN            person  categorical                [0]
         20         LP\_STATUS\_GROB            person  categorical                [0]
         21       NATIONALITAET\_KZ            person  categorical                NaN
         22  PRAEGENDE\_JUGENDJAHRE            person        mixed                NaN
         23         RETOURTYP\_BK\_S            person      ordinal                [0]
         24              SEMIO\_SOZ            person      ordinal             [-1,9]
         25              SEMIO\_FAM            person      ordinal             [-1,9]
         26              SEMIO\_REL            person      ordinal             [-1,9]
         27              SEMIO\_MAT            person      ordinal             [-1,9]
         28             SEMIO\_VERT            person      ordinal             [-1,9]
         29             SEMIO\_LUST            person      ordinal             [-1,9]
         ..                    {\ldots}               {\ldots}          {\ldots}                {\ldots}
         55            OST\_WEST\_KZ          building  categorical               [-1]
         56               WOHNLAGE          building        mixed               [-1]
         57        CAMEO\_DEUG\_2015     microcell\_rr4  categorical             [-1,X]
         58         CAMEO\_DEU\_2015     microcell\_rr4  categorical               [XX]
         59        CAMEO\_INTL\_2015     microcell\_rr4        mixed            [-1,XX]
         60            KBA05\_ANTG1     microcell\_rr3      ordinal               [-1]
         61            KBA05\_ANTG2     microcell\_rr3      ordinal               [-1]
         62            KBA05\_ANTG3     microcell\_rr3      ordinal               [-1]
         63            KBA05\_ANTG4     microcell\_rr3      ordinal               [-1]
         64           KBA05\_BAUMAX     microcell\_rr3        mixed                NaN
         65              KBA05\_GBZ     microcell\_rr3      ordinal                NaN
         66               BALLRAUM          postcode      ordinal               [-1]
         67               EWDICHTE          postcode      ordinal               [-1]
         68             INNENSTADT          postcode      ordinal               [-1]
         69     GEBAEUDETYP\_RASTER        region\_rr1      ordinal                 []
         70                    KKK        region\_rr1      ordinal                NaN
         71             MOBI\_REGIO        region\_rr1      ordinal                 []
         72      ONLINE\_AFFINITAET        region\_rr1      ordinal                 []
         73               REGIOTYP        region\_rr1      ordinal                NaN
         74       KBA13\_ANZAHL\_PKW    macrocell\_plz8      numeric                 []
         75             PLZ8\_ANTG1    macrocell\_plz8      ordinal               [-1]
         76             PLZ8\_ANTG2    macrocell\_plz8      ordinal               [-1]
         77             PLZ8\_ANTG3    macrocell\_plz8      ordinal               [-1]
         78             PLZ8\_ANTG4    macrocell\_plz8      ordinal               [-1]
         79            PLZ8\_BAUMAX    macrocell\_plz8        mixed                NaN
         80               PLZ8\_HHZ    macrocell\_plz8      ordinal               [-1]
         81               PLZ8\_GBZ    macrocell\_plz8      ordinal               [-1]
         82                 ARBEIT         community      ordinal             [-1,9]
         83            ORTSGR\_KLS9         community      ordinal                NaN
         84               RELAT\_AB         community      ordinal             [-1,9]
         
         [85 rows x 4 columns]
\end{Verbatim}
            
    \begin{Verbatim}[commandchars=\\\{\}]
{\color{incolor}In [{\color{incolor}3}]:} 
\end{Verbatim}


    \begin{Verbatim}[commandchars=\\\{\}]

          File "<ipython-input-3-a22715dcebb7>", line 3
        for i,V in enumerate(azdias.iteritems()): m\_u = a['missing\_or\_unknown'][i] column\_name = V[0] m\_u = m\_u[1:-1].split(',') if m\_u != ['']: hold = [] for c in m\_u: if c in ['X','XX']: hold.append(c) else: hold.append(int(c)) azdias[column\_name] = azdias[column\_name].replace(hold, np.nan)
                                                                                             \^{}
    SyntaxError: invalid syntax


    \end{Verbatim}

    \begin{Verbatim}[commandchars=\\\{\}]
{\color{incolor}In [{\color{incolor}14}]:} \PY{c+c1}{\PYZsh{} azdias = azdias.astype(float)}
         \PY{c+c1}{\PYZsh{}azdias = azdias.replace(\PYZsq{}\PYZhy{}1\PYZsq{}, np.nan)}
         
         \PY{c+c1}{\PYZsh{}print(type(azdias.AGER\PYZus{}TYP))}
         
         \PY{n}{azdias} \PY{o}{=} \PY{n}{azdias}\PY{o}{.}\PY{n}{replace}\PY{p}{(}\PY{l+s+s1}{\PYZsq{}}\PY{l+s+s1}{\PYZhy{}1}\PY{l+s+s1}{\PYZsq{}}\PY{p}{,} \PY{l+s+s1}{\PYZsq{}}\PY{l+s+s1}{NAN}\PY{l+s+s1}{\PYZsq{}}\PY{p}{)}
         
         \PY{n}{display}\PY{p}{(}\PY{n}{azdias}\PY{o}{.}\PY{n}{head}\PY{p}{(}\PY{n}{n}\PY{o}{=}\PY{l+m+mi}{20}\PY{p}{)}\PY{p}{)}
         
         \PY{c+c1}{\PYZsh{}for col in azdias:}
         \PY{c+c1}{\PYZsh{}    print (azdias[col].apply(type))}
\end{Verbatim}


    
    \begin{verbatim}
    AGER_TYP  ALTERSKATEGORIE_GROB  ANREDE_KZ  CJT_GESAMTTYP  \
0         -1                     2          1            2.0   
1         -1                     1          2            5.0   
2         -1                     3          2            3.0   
3          2                     4          2            2.0   
4         -1                     3          1            5.0   
5          3                     1          2            2.0   
6         -1                     2          2            5.0   
7         -1                     1          1            3.0   
8         -1                     3          1            3.0   
9         -1                     3          2            4.0   
10         0                     3          2            1.0   
11        -1                     2          1            6.0   
12        -1                     3          1            6.0   
13        -1                     1          2            5.0   
14        -1                     3          1            6.0   
15         1                     4          2            4.0   
16        -1                     1          2            1.0   
17        -1                     2          1            6.0   
18        -1                     2          2            6.0   
19        -1                     3          1            3.0   

    FINANZ_MINIMALIST  FINANZ_SPARER  FINANZ_VORSORGER  FINANZ_ANLEGER  \
0                   3              4                 3               5   
1                   1              5                 2               5   
2                   1              4                 1               2   
3                   4              2                 5               2   
4                   4              3                 4               1   
5                   3              1                 5               2   
6                   1              5                 1               5   
7                   3              3                 4               1   
8                   4              4                 2               4   
9                   2              4                 2               3   
10                  2              2                 5               3   
11                  3              4                 3               5   
12                  5              3                 4               2   
13                  1              4                 3               5   
14                  3              4                 3               5   
15                  4              1                 5               1   
16                  4              3                 1               4   
17                  3              4                 3               5   
18                  2              4                 1               5   
19                  5              2                 3               1   

    FINANZ_UNAUFFAELLIGER  FINANZ_HAUSBAUER    ...     PLZ8_ANTG1  PLZ8_ANTG2  \
0                       5                 3    ...            NaN         NaN   
1                       4                 5    ...            2.0         3.0   
2                       3                 5    ...            3.0         3.0   
3                       1                 2    ...            2.0         2.0   
4                       3                 2    ...            2.0         4.0   
5                       2                 5    ...            2.0         3.0   
6                       4                 3    ...            3.0         3.0   
7                       3                 2    ...            3.0         3.0   
8                       2                 2    ...            2.0         3.0   
9                       5                 4    ...            2.0         3.0   
10                      1                 5    ...            2.0         4.0   
11                      5                 3    ...            NaN         NaN   
12                      4                 1    ...            3.0         3.0   
13                      5                 2    ...            2.0         1.0   
14                      5                 3    ...            NaN         NaN   
15                      1                 4    ...            NaN         NaN   
16                      5                 1    ...            3.0         3.0   
17                      5                 3    ...            NaN         NaN   
18                      4                 1    ...            2.0         3.0   
19                      3                 1    ...            2.0         4.0   

    PLZ8_ANTG3  PLZ8_ANTG4  PLZ8_BAUMAX  PLZ8_HHZ  PLZ8_GBZ  ARBEIT  \
0          NaN         NaN          NaN       NaN       NaN     NaN   
1          2.0         1.0          1.0       5.0       4.0     3.0   
2          1.0         0.0          1.0       4.0       4.0     3.0   
3          2.0         0.0          1.0       3.0       4.0     2.0   
4          2.0         1.0          2.0       3.0       3.0     4.0   
5          1.0         1.0          1.0       5.0       5.0     2.0   
6          1.0         0.0          1.0       5.0       5.0     4.0   
7          1.0         0.0          1.0       4.0       4.0     2.0   
8          2.0         1.0          1.0       3.0       3.0     2.0   
9          2.0         1.0          1.0       3.0       3.0     2.0   
10         2.0         0.0          2.0       3.0       3.0     4.0   
11         NaN         NaN          NaN       NaN       NaN     NaN   
12         1.0         0.0          1.0       5.0       5.0     3.0   
13         1.0         1.0          1.0       3.0       3.0     3.0   
14         NaN         NaN          NaN       NaN       NaN     NaN   
15         NaN         NaN          NaN       NaN       NaN     4.0   
16         1.0         0.0          1.0       3.0       4.0     1.0   
17         NaN         NaN          NaN       NaN       NaN     NaN   
18         2.0         1.0          1.0       3.0       3.0     3.0   
19         2.0         1.0          2.0       5.0       4.0     4.0   

    ORTSGR_KLS9  RELAT_AB  
0           NaN       NaN  
1           5.0       4.0  
2           5.0       2.0  
3           3.0       3.0  
4           6.0       5.0  
5           3.0       3.0  
6           6.0       3.0  
7           5.0       2.0  
8           4.0       3.0  
9           3.0       1.0  
10          6.0       5.0  
11          NaN       NaN  
12          6.0       4.0  
13          6.0       4.0  
14          NaN       NaN  
15          8.0       5.0  
16          2.0       1.0  
17          NaN       NaN  
18          4.0       3.0  
19          6.0       3.0  

[20 rows x 85 columns]
    \end{verbatim}

    
    \hypertarget{step-1.1.2-assess-missing-data-in-each-column}{%
\paragraph{Step 1.1.2: Assess Missing Data in Each
Column}\label{step-1.1.2-assess-missing-data-in-each-column}}

How much missing data is present in each column? There are a few columns
that are outliers in terms of the proportion of values that are missing.
You will want to use matplotlib's
\href{https://matplotlib.org/api/_as_gen/matplotlib.pyplot.hist.html}{\texttt{hist()}}
function to visualize the distribution of missing value counts to find
these columns. Identify and document these columns. While some of these
columns might have justifications for keeping or re-encoding the data,
for this project you should just remove them from the dataframe. (Feel
free to make remarks about these outlier columns in the discussion,
however!)

For the remaining features, are there any patterns in which columns
have, or share, missing data?

    \begin{Verbatim}[commandchars=\\\{\}]
{\color{incolor}In [{\color{incolor}9}]:} \PY{c+c1}{\PYZsh{} Perform an assessment of how much missing data there is in each column of the}
        \PY{c+c1}{\PYZsh{} dataset.}
        \PY{n}{azdias}\PY{o}{.}\PY{n}{isnull}\PY{p}{(}\PY{p}{)}\PY{o}{.}\PY{n}{sum}\PY{p}{(}\PY{p}{)}
\end{Verbatim}


\begin{Verbatim}[commandchars=\\\{\}]
{\color{outcolor}Out[{\color{outcolor}9}]:} AGER\_TYP                      0
        ALTERSKATEGORIE\_GROB          0
        ANREDE\_KZ                     0
        CJT\_GESAMTTYP              4854
        FINANZ\_MINIMALIST             0
        FINANZ\_SPARER                 0
        FINANZ\_VORSORGER              0
        FINANZ\_ANLEGER                0
        FINANZ\_UNAUFFAELLIGER         0
        FINANZ\_HAUSBAUER              0
        FINANZTYP                     0
        GEBURTSJAHR                   0
        GFK\_URLAUBERTYP            4854
        GREEN\_AVANTGARDE              0
        HEALTH\_TYP                    0
        LP\_LEBENSPHASE\_FEIN        4854
        LP\_LEBENSPHASE\_GROB        4854
        LP\_FAMILIE\_FEIN            4854
        LP\_FAMILIE\_GROB            4854
        LP\_STATUS\_FEIN             4854
        LP\_STATUS\_GROB             4854
        NATIONALITAET\_KZ              0
        PRAEGENDE\_JUGENDJAHRE         0
        RETOURTYP\_BK\_S             4854
        SEMIO\_SOZ                     0
        SEMIO\_FAM                     0
        SEMIO\_REL                     0
        SEMIO\_MAT                     0
        SEMIO\_VERT                    0
        SEMIO\_LUST                    0
                                  {\ldots}  
        OST\_WEST\_KZ               93148
        WOHNLAGE                  93148
        CAMEO\_DEUG\_2015           98979
        CAMEO\_DEU\_2015            98979
        CAMEO\_INTL\_2015           98979
        KBA05\_ANTG1              133324
        KBA05\_ANTG2              133324
        KBA05\_ANTG3              133324
        KBA05\_ANTG4              133324
        KBA05\_BAUMAX             133324
        KBA05\_GBZ                133324
        BALLRAUM                  93740
        EWDICHTE                  93740
        INNENSTADT                93740
        GEBAEUDETYP\_RASTER        93155
        KKK                      121196
        MOBI\_REGIO               133324
        ONLINE\_AFFINITAET          4854
        REGIOTYP                 121196
        KBA13\_ANZAHL\_PKW         105800
        PLZ8\_ANTG1               116515
        PLZ8\_ANTG2               116515
        PLZ8\_ANTG3               116515
        PLZ8\_ANTG4               116515
        PLZ8\_BAUMAX              116515
        PLZ8\_HHZ                 116515
        PLZ8\_GBZ                 116515
        ARBEIT                    97216
        ORTSGR\_KLS9               97216
        RELAT\_AB                  97216
        Length: 85, dtype: int64
\end{Verbatim}
            
    \begin{Verbatim}[commandchars=\\\{\}]
{\color{incolor}In [{\color{incolor} }]:} \PY{c+c1}{\PYZsh{} Investigate patterns in the amount of missing data in each column.}
\end{Verbatim}


    \begin{Verbatim}[commandchars=\\\{\}]
{\color{incolor}In [{\color{incolor} }]:} \PY{c+c1}{\PYZsh{} Remove the outlier columns from the dataset. (You\PYZsq{}ll perform other data}
        \PY{c+c1}{\PYZsh{} engineering tasks such as re\PYZhy{}encoding and imputation later.)}
\end{Verbatim}


    \hypertarget{discussion-1.1.2-assess-missing-data-in-each-column}{%
\paragraph{Discussion 1.1.2: Assess Missing Data in Each
Column}\label{discussion-1.1.2-assess-missing-data-in-each-column}}

(Double click this cell and replace this text with your own text,
reporting your observations regarding the amount of missing data in each
column. Are there any patterns in missing values? Which columns were
removed from the dataset?)

    \hypertarget{step-1.1.3-assess-missing-data-in-each-row}{%
\paragraph{Step 1.1.3: Assess Missing Data in Each
Row}\label{step-1.1.3-assess-missing-data-in-each-row}}

Now, you'll perform a similar assessment for the rows of the dataset.
How much data is missing in each row? As with the columns, you should
see some groups of points that have a very different numbers of missing
values. Divide the data into two subsets: one for data points that are
above some threshold for missing values, and a second subset for points
below that threshold.

In order to know what to do with the outlier rows, we should see if the
distribution of data values on columns that are not missing data (or are
missing very little data) are similar or different between the two
groups. Select at least five of these columns and compare the
distribution of values. - You can use seaborn's
\href{https://seaborn.pydata.org/generated/seaborn.countplot.html}{\texttt{countplot()}}
function to create a bar chart of code frequencies and matplotlib's
\href{https://matplotlib.org/api/_as_gen/matplotlib.pyplot.subplot.html}{\texttt{subplot()}}
function to put bar charts for the two subplots side by side. - To
reduce repeated code, you might want to write a function that can
perform this comparison, taking as one of its arguments a column to be
compared.

Depending on what you observe in your comparison, this will have
implications on how you approach your conclusions later in the analysis.
If the distributions of non-missing features look similar between the
data with many missing values and the data with few or no missing
values, then we could argue that simply dropping those points from the
analysis won't present a major issue. On the other hand, if the data
with many missing values looks very different from the data with few or
no missing values, then we should make a note on those data as special.
We'll revisit these data later on. \textbf{Either way, you should
continue your analysis for now using just the subset of the data with
few or no missing values.}

    \begin{Verbatim}[commandchars=\\\{\}]
{\color{incolor}In [{\color{incolor} }]:} \PY{c+c1}{\PYZsh{} How much data is missing in each row of the dataset?}
\end{Verbatim}


    \begin{Verbatim}[commandchars=\\\{\}]
{\color{incolor}In [{\color{incolor} }]:} \PY{c+c1}{\PYZsh{} Write code to divide the data into two subsets based on the number of missing}
        \PY{c+c1}{\PYZsh{} values in each row.}
\end{Verbatim}


    \begin{Verbatim}[commandchars=\\\{\}]
{\color{incolor}In [{\color{incolor} }]:} \PY{c+c1}{\PYZsh{} Compare the distribution of values for at least five columns where there are}
        \PY{c+c1}{\PYZsh{} no or few missing values, between the two subsets.}
\end{Verbatim}


    \hypertarget{discussion-1.1.3-assess-missing-data-in-each-row}{%
\paragraph{Discussion 1.1.3: Assess Missing Data in Each
Row}\label{discussion-1.1.3-assess-missing-data-in-each-row}}

(Double-click this cell and replace this text with your own text,
reporting your observations regarding missing data in rows. Are the data
with lots of missing values are qualitatively different from data with
few or no missing values?)

    \hypertarget{step-1.2-select-and-re-encode-features}{%
\subsubsection{Step 1.2: Select and Re-Encode
Features}\label{step-1.2-select-and-re-encode-features}}

Checking for missing data isn't the only way in which you can prepare a
dataset for analysis. Since the unsupervised learning techniques to be
used will only work on data that is encoded numerically, you need to
make a few encoding changes or additional assumptions to be able to make
progress. In addition, while almost all of the values in the dataset are
encoded using numbers, not all of them represent numeric values. Check
the third column of the feature summary (\texttt{feat\_info}) for a
summary of types of measurement. - For numeric and interval data, these
features can be kept without changes. - Most of the variables in the
dataset are ordinal in nature. While ordinal values may technically be
non-linear in spacing, make the simplifying assumption that the ordinal
variables can be treated as being interval in nature (that is, kept
without any changes). - Special handling may be necessary for the
remaining two variable types: categorical, and `mixed'.

In the first two parts of this sub-step, you will perform an
investigation of the categorical and mixed-type features and make a
decision on each of them, whether you will keep, drop, or re-encode
each. Then, in the last part, you will create a new data frame with only
the selected and engineered columns.

Data wrangling is often the trickiest part of the data analysis process,
and there's a lot of it to be done here. But stick with it: once you're
done with this step, you'll be ready to get to the machine learning
parts of the project!

    \begin{Verbatim}[commandchars=\\\{\}]
{\color{incolor}In [{\color{incolor} }]:} \PY{c+c1}{\PYZsh{} How many features are there of each data type?}
\end{Verbatim}


    \hypertarget{step-1.2.1-re-encode-categorical-features}{%
\paragraph{Step 1.2.1: Re-Encode Categorical
Features}\label{step-1.2.1-re-encode-categorical-features}}

For categorical data, you would ordinarily need to encode the levels as
dummy variables. Depending on the number of categories, perform one of
the following: - For binary (two-level) categoricals that take numeric
values, you can keep them without needing to do anything. - There is one
binary variable that takes on non-numeric values. For this one, you need
to re-encode the values as numbers or create a dummy variable. - For
multi-level categoricals (three or more values), you can choose to
encode the values using multiple dummy variables (e.g.~via
\href{http://scikit-learn.org/stable/modules/generated/sklearn.preprocessing.OneHotEncoder.html}{OneHotEncoder}),
or (to keep things straightforward) just drop them from the analysis. As
always, document your choices in the Discussion section.

    \begin{Verbatim}[commandchars=\\\{\}]
{\color{incolor}In [{\color{incolor} }]:} \PY{c+c1}{\PYZsh{} Assess categorical variables: which are binary, which are multi\PYZhy{}level, and}
        \PY{c+c1}{\PYZsh{} which one needs to be re\PYZhy{}encoded?}
\end{Verbatim}


    \begin{Verbatim}[commandchars=\\\{\}]
{\color{incolor}In [{\color{incolor} }]:} \PY{c+c1}{\PYZsh{} Re\PYZhy{}encode categorical variable(s) to be kept in the analysis.}
\end{Verbatim}


    \hypertarget{discussion-1.2.1-re-encode-categorical-features}{%
\paragraph{Discussion 1.2.1: Re-Encode Categorical
Features}\label{discussion-1.2.1-re-encode-categorical-features}}

(Double-click this cell and replace this text with your own text,
reporting your findings and decisions regarding categorical features.
Which ones did you keep, which did you drop, and what engineering steps
did you perform?)

    \hypertarget{step-1.2.2-engineer-mixed-type-features}{%
\paragraph{Step 1.2.2: Engineer Mixed-Type
Features}\label{step-1.2.2-engineer-mixed-type-features}}

There are a handful of features that are marked as ``mixed'' in the
feature summary that require special treatment in order to be included
in the analysis. There are two in particular that deserve attention; the
handling of the rest are up to your own choices: -
``PRAEGENDE\_JUGENDJAHRE'' combines information on three dimensions:
generation by decade, movement (mainstream vs.~avantgarde), and nation
(east vs.~west). While there aren't enough levels to disentangle east
from west, you should create two new variables to capture the other two
dimensions: an interval-type variable for decade, and a binary variable
for movement. - ``CAMEO\_INTL\_2015'' combines information on two axes:
wealth and life stage. Break up the two-digit codes by their
`tens'-place and `ones'-place digits into two new ordinal variables
(which, for the purposes of this project, is equivalent to just treating
them as their raw numeric values). - If you decide to keep or engineer
new features around the other mixed-type features, make sure you note
your steps in the Discussion section.

Be sure to check \texttt{Data\_Dictionary.md} for the details needed to
finish these tasks.

    \begin{Verbatim}[commandchars=\\\{\}]
{\color{incolor}In [{\color{incolor} }]:} \PY{c+c1}{\PYZsh{} Investigate \PYZdq{}PRAEGENDE\PYZus{}JUGENDJAHRE\PYZdq{} and engineer two new variables.}
\end{Verbatim}


    \begin{Verbatim}[commandchars=\\\{\}]
{\color{incolor}In [{\color{incolor} }]:} \PY{c+c1}{\PYZsh{} Investigate \PYZdq{}CAMEO\PYZus{}INTL\PYZus{}2015\PYZdq{} and engineer two new variables.}
\end{Verbatim}


    \hypertarget{discussion-1.2.2-engineer-mixed-type-features}{%
\paragraph{Discussion 1.2.2: Engineer Mixed-Type
Features}\label{discussion-1.2.2-engineer-mixed-type-features}}

(Double-click this cell and replace this text with your own text,
reporting your findings and decisions regarding mixed-value features.
Which ones did you keep, which did you drop, and what engineering steps
did you perform?)

    \hypertarget{step-1.2.3-complete-feature-selection}{%
\paragraph{Step 1.2.3: Complete Feature
Selection}\label{step-1.2.3-complete-feature-selection}}

In order to finish this step up, you need to make sure that your data
frame now only has the columns that you want to keep. To summarize, the
dataframe should consist of the following: - All numeric, interval, and
ordinal type columns from the original dataset. - Binary categorical
features (all numerically-encoded). - Engineered features from other
multi-level categorical features and mixed features.

Make sure that for any new columns that you have engineered, that you've
excluded the original columns from the final dataset. Otherwise, their
values will interfere with the analysis later on the project. For
example, you should not keep ``PRAEGENDE\_JUGENDJAHRE'', since its
values won't be useful for the algorithm: only the values derived from
it in the engineered features you created should be retained. As a
reminder, your data should only be from \textbf{the subset with few or
no missing values}.

    \begin{Verbatim}[commandchars=\\\{\}]
{\color{incolor}In [{\color{incolor} }]:} \PY{c+c1}{\PYZsh{} If there are other re\PYZhy{}engineering tasks you need to perform, make sure you}
        \PY{c+c1}{\PYZsh{} take care of them here. (Dealing with missing data will come in step 2.1.)}
\end{Verbatim}


    \begin{Verbatim}[commandchars=\\\{\}]
{\color{incolor}In [{\color{incolor} }]:} \PY{c+c1}{\PYZsh{} Do whatever you need to in order to ensure that the dataframe only contains}
        \PY{c+c1}{\PYZsh{} the columns that should be passed to the algorithm functions.}
\end{Verbatim}


    \hypertarget{step-1.3-create-a-cleaning-function}{%
\subsubsection{Step 1.3: Create a Cleaning
Function}\label{step-1.3-create-a-cleaning-function}}

Even though you've finished cleaning up the general population
demographics data, it's important to look ahead to the future and
realize that you'll need to perform the same cleaning steps on the
customer demographics data. In this substep, complete the function below
to execute the main feature selection, encoding, and re-engineering
steps you performed above. Then, when it comes to looking at the
customer data in Step 3, you can just run this function on that
DataFrame to get the trimmed dataset in a single step.

    \begin{Verbatim}[commandchars=\\\{\}]
{\color{incolor}In [{\color{incolor}1}]:} \PY{k}{def} \PY{n+nf}{clean\PYZus{}data}\PY{p}{(}\PY{n}{df}\PY{p}{)}\PY{p}{:}
            \PY{l+s+sd}{\PYZdq{}\PYZdq{}\PYZdq{}}
        \PY{l+s+sd}{    Perform feature trimming, re\PYZhy{}encoding, and engineering for demographics}
        \PY{l+s+sd}{    data}
        \PY{l+s+sd}{    }
        \PY{l+s+sd}{    INPUT: Demographics DataFrame}
        \PY{l+s+sd}{    OUTPUT: Trimmed and cleaned demographics DataFrame}
        \PY{l+s+sd}{    \PYZdq{}\PYZdq{}\PYZdq{}}
            
            \PY{c+c1}{\PYZsh{} Put in code here to execute all main cleaning steps:}
            \PY{c+c1}{\PYZsh{} convert missing value codes into NaNs, ...}
            
            
            \PY{c+c1}{\PYZsh{} remove selected columns and rows, ...}
        
            
            \PY{c+c1}{\PYZsh{} select, re\PYZhy{}encode, and engineer column values.}
        
            
            \PY{c+c1}{\PYZsh{} Return the cleaned dataframe.}
            
            
\end{Verbatim}


    \hypertarget{step-2-feature-transformation}{%
\subsection{Step 2: Feature
Transformation}\label{step-2-feature-transformation}}

\hypertarget{step-2.1-apply-feature-scaling}{%
\subsubsection{Step 2.1: Apply Feature
Scaling}\label{step-2.1-apply-feature-scaling}}

Before we apply dimensionality reduction techniques to the data, we need
to perform feature scaling so that the principal component vectors are
not influenced by the natural differences in scale for features.
Starting from this part of the project, you'll want to keep an eye on
the \href{http://scikit-learn.org/stable/modules/classes.html}{API
reference page for sklearn} to help you navigate to all of the classes
and functions that you'll need. In this substep, you'll need to check
the following:

\begin{itemize}
\tightlist
\item
  sklearn requires that data not have missing values in order for its
  estimators to work properly. So, before applying the scaler to your
  data, make sure that you've cleaned the DataFrame of the remaining
  missing values. This can be as simple as just removing all data points
  with missing data, or applying an
  \href{http://scikit-learn.org/stable/modules/generated/sklearn.preprocessing.Imputer.html}{Imputer}
  to replace all missing values. You might also try a more complicated
  procedure where you temporarily remove missing values in order to
  compute the scaling parameters before re-introducing those missing
  values and applying imputation. Think about how much missing data you
  have and what possible effects each approach might have on your
  analysis, and justify your decision in the discussion section below.
\item
  For the actual scaling function, a
  \href{http://scikit-learn.org/stable/modules/generated/sklearn.preprocessing.StandardScaler.html}{StandardScaler}
  instance is suggested, scaling each feature to mean 0 and standard
  deviation 1.
\item
  For these classes, you can make use of the \texttt{.fit\_transform()}
  method to both fit a procedure to the data as well as apply the
  transformation to the data at the same time. Don't forget to keep the
  fit sklearn objects handy, since you'll be applying them to the
  customer demographics data towards the end of the project.
\end{itemize}

    \begin{Verbatim}[commandchars=\\\{\}]
{\color{incolor}In [{\color{incolor} }]:} \PY{c+c1}{\PYZsh{} If you\PYZsq{}ve not yet cleaned the dataset of all NaN values, then investigate and}
        \PY{c+c1}{\PYZsh{} do that now.}
\end{Verbatim}


    \begin{Verbatim}[commandchars=\\\{\}]
{\color{incolor}In [{\color{incolor} }]:} \PY{c+c1}{\PYZsh{} Apply feature scaling to the general population demographics data.}
\end{Verbatim}


    \hypertarget{discussion-2.1-apply-feature-scaling}{%
\subsubsection{Discussion 2.1: Apply Feature
Scaling}\label{discussion-2.1-apply-feature-scaling}}

(Double-click this cell and replace this text with your own text,
reporting your decisions regarding feature scaling.)

    \hypertarget{step-2.2-perform-dimensionality-reduction}{%
\subsubsection{Step 2.2: Perform Dimensionality
Reduction}\label{step-2.2-perform-dimensionality-reduction}}

On your scaled data, you are now ready to apply dimensionality reduction
techniques.

\begin{itemize}
\tightlist
\item
  Use sklearn's
  \href{http://scikit-learn.org/stable/modules/generated/sklearn.decomposition.PCA.html}{PCA}
  class to apply principal component analysis on the data, thus finding
  the vectors of maximal variance in the data. To start, you should not
  set any parameters (so all components are computed) or set a number of
  components that is at least half the number of features (so there's
  enough features to see the general trend in variability).
\item
  Check out the ratio of variance explained by each principal component
  as well as the cumulative variance explained. Try plotting the
  cumulative or sequential values using matplotlib's
  \href{https://matplotlib.org/api/_as_gen/matplotlib.pyplot.plot.html}{\texttt{plot()}}
  function. Based on what you find, select a value for the number of
  transformed features you'll retain for the clustering part of the
  project.
\item
  Once you've made a choice for the number of components to keep, make
  sure you re-fit a PCA instance to perform the decided-on
  transformation.
\end{itemize}

    \begin{Verbatim}[commandchars=\\\{\}]
{\color{incolor}In [{\color{incolor} }]:} \PY{c+c1}{\PYZsh{} Apply PCA to the data.}
\end{Verbatim}


    \begin{Verbatim}[commandchars=\\\{\}]
{\color{incolor}In [{\color{incolor} }]:} \PY{c+c1}{\PYZsh{} Investigate the variance accounted for by each principal component.}
\end{Verbatim}


    \begin{Verbatim}[commandchars=\\\{\}]
{\color{incolor}In [{\color{incolor} }]:} \PY{c+c1}{\PYZsh{} Re\PYZhy{}apply PCA to the data while selecting for number of components to retain.}
\end{Verbatim}


    \hypertarget{discussion-2.2-perform-dimensionality-reduction}{%
\subsubsection{Discussion 2.2: Perform Dimensionality
Reduction}\label{discussion-2.2-perform-dimensionality-reduction}}

(Double-click this cell and replace this text with your own text,
reporting your findings and decisions regarding dimensionality
reduction. How many principal components / transformed features are you
retaining for the next step of the analysis?)

    \hypertarget{step-2.3-interpret-principal-components}{%
\subsubsection{Step 2.3: Interpret Principal
Components}\label{step-2.3-interpret-principal-components}}

Now that we have our transformed principal components, it's a nice idea
to check out the weight of each variable on the first few components to
see if they can be interpreted in some fashion.

As a reminder, each principal component is a unit vector that points in
the direction of highest variance (after accounting for the variance
captured by earlier principal components). The further a weight is from
zero, the more the principal component is in the direction of the
corresponding feature. If two features have large weights of the same
sign (both positive or both negative), then increases in one tend expect
to be associated with increases in the other. To contrast, features with
different signs can be expected to show a negative correlation:
increases in one variable should result in a decrease in the other.

\begin{itemize}
\tightlist
\item
  To investigate the features, you should map each weight to their
  corresponding feature name, then sort the features according to
  weight. The most interesting features for each principal component,
  then, will be those at the beginning and end of the sorted list. Use
  the data dictionary document to help you understand these most
  prominent features, their relationships, and what a positive or
  negative value on the principal component might indicate.
\item
  You should investigate and interpret feature associations from the
  first three principal components in this substep. To help facilitate
  this, you should write a function that you can call at any time to
  print the sorted list of feature weights, for the \emph{i}-th
  principal component. This might come in handy in the next step of the
  project, when you interpret the tendencies of the discovered clusters.
\end{itemize}

    \begin{Verbatim}[commandchars=\\\{\}]
{\color{incolor}In [{\color{incolor} }]:} \PY{c+c1}{\PYZsh{} Map weights for the first principal component to corresponding feature names}
        \PY{c+c1}{\PYZsh{} and then print the linked values, sorted by weight.}
        \PY{c+c1}{\PYZsh{} HINT: Try defining a function here or in a new cell that you can reuse in the}
        \PY{c+c1}{\PYZsh{} other cells.}
\end{Verbatim}


    \begin{Verbatim}[commandchars=\\\{\}]
{\color{incolor}In [{\color{incolor} }]:} \PY{c+c1}{\PYZsh{} Map weights for the second principal component to corresponding feature names}
        \PY{c+c1}{\PYZsh{} and then print the linked values, sorted by weight.}
\end{Verbatim}


    \begin{Verbatim}[commandchars=\\\{\}]
{\color{incolor}In [{\color{incolor} }]:} \PY{c+c1}{\PYZsh{} Map weights for the third principal component to corresponding feature names}
        \PY{c+c1}{\PYZsh{} and then print the linked values, sorted by weight.}
\end{Verbatim}


    \hypertarget{discussion-2.3-interpret-principal-components}{%
\subsubsection{Discussion 2.3: Interpret Principal
Components}\label{discussion-2.3-interpret-principal-components}}

(Double-click this cell and replace this text with your own text,
reporting your observations from detailed investigation of the first few
principal components generated. Can we interpret positive and negative
values from them in a meaningful way?)

    \hypertarget{step-3-clustering}{%
\subsection{Step 3: Clustering}\label{step-3-clustering}}

\hypertarget{step-3.1-apply-clustering-to-general-population}{%
\subsubsection{Step 3.1: Apply Clustering to General
Population}\label{step-3.1-apply-clustering-to-general-population}}

You've assessed and cleaned the demographics data, then scaled and
transformed them. Now, it's time to see how the data clusters in the
principal components space. In this substep, you will apply k-means
clustering to the dataset and use the average within-cluster distances
from each point to their assigned cluster's centroid to decide on a
number of clusters to keep.

\begin{itemize}
\tightlist
\item
  Use sklearn's
  \href{http://scikit-learn.org/stable/modules/generated/sklearn.cluster.KMeans.html\#sklearn.cluster.KMeans}{KMeans}
  class to perform k-means clustering on the PCA-transformed data.
\item
  Then, compute the average difference from each point to its assigned
  cluster's center. \textbf{Hint}: The KMeans object's \texttt{.score()}
  method might be useful here, but note that in sklearn, scores tend to
  be defined so that larger is better. Try applying it to a small, toy
  dataset, or use an internet search to help your understanding.
\item
  Perform the above two steps for a number of different cluster counts.
  You can then see how the average distance decreases with an increasing
  number of clusters. However, each additional cluster provides a
  smaller net benefit. Use this fact to select a final number of
  clusters in which to group the data. \textbf{Warning}: because of the
  large size of the dataset, it can take a long time for the algorithm
  to resolve. The more clusters to fit, the longer the algorithm will
  take. You should test for cluster counts through at least 10 clusters
  to get the full picture, but you shouldn't need to test for a number
  of clusters above about 30.
\item
  Once you've selected a final number of clusters to use, re-fit a
  KMeans instance to perform the clustering operation. Make sure that
  you also obtain the cluster assignments for the general demographics
  data, since you'll be using them in the final Step 3.3.
\end{itemize}

    \begin{Verbatim}[commandchars=\\\{\}]
{\color{incolor}In [{\color{incolor} }]:} \PY{c+c1}{\PYZsh{} Over a number of different cluster counts...}
        
        
            \PY{c+c1}{\PYZsh{} run k\PYZhy{}means clustering on the data and...}
            
            
            \PY{c+c1}{\PYZsh{} compute the average within\PYZhy{}cluster distances.}
            
            
\end{Verbatim}


    \begin{Verbatim}[commandchars=\\\{\}]
{\color{incolor}In [{\color{incolor} }]:} \PY{c+c1}{\PYZsh{} Investigate the change in within\PYZhy{}cluster distance across number of clusters.}
        \PY{c+c1}{\PYZsh{} HINT: Use matplotlib\PYZsq{}s plot function to visualize this relationship.}
\end{Verbatim}


    \begin{Verbatim}[commandchars=\\\{\}]
{\color{incolor}In [{\color{incolor} }]:} \PY{c+c1}{\PYZsh{} Re\PYZhy{}fit the k\PYZhy{}means model with the selected number of clusters and obtain}
        \PY{c+c1}{\PYZsh{} cluster predictions for the general population demographics data.}
\end{Verbatim}


    \hypertarget{discussion-3.1-apply-clustering-to-general-population}{%
\subsubsection{Discussion 3.1: Apply Clustering to General
Population}\label{discussion-3.1-apply-clustering-to-general-population}}

(Double-click this cell and replace this text with your own text,
reporting your findings and decisions regarding clustering. Into how
many clusters have you decided to segment the population?)

    \hypertarget{step-3.2-apply-all-steps-to-the-customer-data}{%
\subsubsection{Step 3.2: Apply All Steps to the Customer
Data}\label{step-3.2-apply-all-steps-to-the-customer-data}}

Now that you have clusters and cluster centers for the general
population, it's time to see how the customer data maps on to those
clusters. Take care to not confuse this for re-fitting all of the models
to the customer data. Instead, you're going to use the fits from the
general population to clean, transform, and cluster the customer data.
In the last step of the project, you will interpret how the general
population fits apply to the customer data.

\begin{itemize}
\tightlist
\item
  Don't forget when loading in the customers data, that it is semicolon
  (\texttt{;}) delimited.
\item
  Apply the same feature wrangling, selection, and engineering steps to
  the customer demographics using the \texttt{clean\_data()} function
  you created earlier. (You can assume that the customer demographics
  data has similar meaning behind missing data patterns as the general
  demographics data.)
\item
  Use the sklearn objects from the general demographics data, and apply
  their transformations to the customers data. That is, you should not
  be using a \texttt{.fit()} or \texttt{.fit\_transform()} method to
  re-fit the old objects, nor should you be creating new sklearn
  objects! Carry the data through the feature scaling, PCA, and
  clustering steps, obtaining cluster assignments for all of the data in
  the customer demographics data.
\end{itemize}

    \begin{Verbatim}[commandchars=\\\{\}]
{\color{incolor}In [{\color{incolor} }]:} \PY{c+c1}{\PYZsh{} Load in the customer demographics data.}
        \PY{n}{customers} \PY{o}{=} 
\end{Verbatim}


    \begin{Verbatim}[commandchars=\\\{\}]
{\color{incolor}In [{\color{incolor} }]:} \PY{c+c1}{\PYZsh{} Apply preprocessing, feature transformation, and clustering from the general}
        \PY{c+c1}{\PYZsh{} demographics onto the customer data, obtaining cluster predictions for the}
        \PY{c+c1}{\PYZsh{} customer demographics data.}
\end{Verbatim}


    \hypertarget{step-3.3-compare-customer-data-to-demographics-data}{%
\subsubsection{Step 3.3: Compare Customer Data to Demographics
Data}\label{step-3.3-compare-customer-data-to-demographics-data}}

At this point, you have clustered data based on demographics of the
general population of Germany, and seen how the customer data for a
mail-order sales company maps onto those demographic clusters. In this
final substep, you will compare the two cluster distributions to see
where the strongest customer base for the company is.

Consider the proportion of persons in each cluster for the general
population, and the proportions for the customers. If we think the
company's customer base to be universal, then the cluster assignment
proportions should be fairly similar between the two. If there are only
particular segments of the population that are interested in the
company's products, then we should see a mismatch from one to the other.
If there is a higher proportion of persons in a cluster for the customer
data compared to the general population (e.g.~5\% of persons are
assigned to a cluster for the general population, but 15\% of the
customer data is closest to that cluster's centroid) then that suggests
the people in that cluster to be a target audience for the company. On
the other hand, the proportion of the data in a cluster being larger in
the general population than the customer data (e.g.~only 2\% of
customers closest to a population centroid that captures 6\% of the
data) suggests that group of persons to be outside of the target
demographics.

Take a look at the following points in this step:

\begin{itemize}
\tightlist
\item
  Compute the proportion of data points in each cluster for the general
  population and the customer data. Visualizations will be useful here:
  both for the individual dataset proportions, but also to visualize the
  ratios in cluster representation between groups. Seaborn's
  \href{https://seaborn.pydata.org/generated/seaborn.countplot.html}{\texttt{countplot()}}
  or
  \href{https://seaborn.pydata.org/generated/seaborn.barplot.html}{\texttt{barplot()}}
  function could be handy.

  \begin{itemize}
  \tightlist
  \item
    Recall the analysis you performed in step 1.1.3 of the project,
    where you separated out certain data points from the dataset if they
    had more than a specified threshold of missing values. If you found
    that this group was qualitatively different from the main bulk of
    the data, you should treat this as an additional data cluster in
    this analysis. Make sure that you account for the number of data
    points in this subset, for both the general population and customer
    datasets, when making your computations!
  \end{itemize}
\item
  Which cluster or clusters are overrepresented in the customer dataset
  compared to the general population? Select at least one such cluster
  and infer what kind of people might be represented by that cluster.
  Use the principal component interpretations from step 2.3 or look at
  additional components to help you make this inference. Alternatively,
  you can use the \texttt{.inverse\_transform()} method of the PCA and
  StandardScaler objects to transform centroids back to the original
  data space and interpret the retrieved values directly.
\item
  Perform a similar investigation for the underrepresented clusters.
  Which cluster or clusters are underrepresented in the customer dataset
  compared to the general population, and what kinds of people are
  typified by these clusters?
\end{itemize}

    \begin{Verbatim}[commandchars=\\\{\}]
{\color{incolor}In [{\color{incolor} }]:} \PY{c+c1}{\PYZsh{} Compare the proportion of data in each cluster for the customer data to the}
        \PY{c+c1}{\PYZsh{} proportion of data in each cluster for the general population.}
\end{Verbatim}


    \begin{Verbatim}[commandchars=\\\{\}]
{\color{incolor}In [{\color{incolor} }]:} \PY{c+c1}{\PYZsh{} What kinds of people are part of a cluster that is overrepresented in the}
        \PY{c+c1}{\PYZsh{} customer data compared to the general population?}
\end{Verbatim}


    \begin{Verbatim}[commandchars=\\\{\}]
{\color{incolor}In [{\color{incolor} }]:} \PY{c+c1}{\PYZsh{} What kinds of people are part of a cluster that is underrepresented in the}
        \PY{c+c1}{\PYZsh{} customer data compared to the general population?}
\end{Verbatim}


    \hypertarget{discussion-3.3-compare-customer-data-to-demographics-data}{%
\subsubsection{Discussion 3.3: Compare Customer Data to Demographics
Data}\label{discussion-3.3-compare-customer-data-to-demographics-data}}

(Double-click this cell and replace this text with your own text,
reporting findings and conclusions from the clustering analysis. Can we
describe segments of the population that are relatively popular with the
mail-order company, or relatively unpopular with the company?)

    \begin{quote}
Congratulations on making it this far in the project! Before you finish,
make sure to check through the entire notebook from top to bottom to
make sure that your analysis follows a logical flow and all of your
findings are documented in \textbf{Discussion} cells. Once you've
checked over all of your work, you should export the notebook as an HTML
document to submit for evaluation. You can do this from the menu,
navigating to \textbf{File -\textgreater{} Download as -\textgreater{}
HTML (.html)}. You will submit both that document and this notebook for
your project submission.
\end{quote}


    % Add a bibliography block to the postdoc
    
    
    
    \end{document}
